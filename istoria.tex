
%title
%page1

П. В. Каменский

История Одного Дня

Достоверное Сказание

Ликар-Стоматолог ВДОВ

Валентин Дмитрович

ЕКАТЕРИНОСЛАВ.

Типо-литография М. С. Копылова. Проспект, собств. дом

1900.

%page2

Дозволено цензурою. Харьков, 25 сентября 1900 г.

%page3

ИСТОРИЯ ОДНОГО ДНЯ

(ДОСТОВЕРНОЕ СКАЗАНИЕ)

Около 20-х чисел марта 1868 года в г. Бердянске,
в гостиннице ``Белого Лебедя'' ореховский 3-й гильдии купец Поддубня,
торговавший чаем, дегтем и салом, рассказывал двум своим знакомым, Мазину и Колосовскому,
о приключившихся ему злоключениях в г. Мариуполе. В числе слушателей
находился еще третий слушатель, неизвестный Поддубне, но также внимавший рассказчику;
назывался он \textbf{\em Григорий Власов Ильяшенко}. Это был молодой человек 33 лет, блондин, внешней 
фигурой ничего из себя особенного не представлявший, никаких, как говорится в паспортах, особых примет не имевший.
Он был уроженец гор. Николаева: в описываемое время проживал с женой в г. Бердянске, где занимался частно
чертежными работами у местного архитектора. До прибытия в Бердянск Ильяшенко
находился на службе в севастопольской инженерной команде морской строительной части
чертежником и был награжден, как это удостоверяется
официальными документами, бронзовой медалью на Андреевской ленте.

Разеказ Поддубни не был связный. Он начал с объяснения, в чем состояло в г. Мариуполе
особое греческое управление, но его объяснения были мало уяснительны,
%page4
ибо они сводились на повторение одних и тех же слов: каторжные греческие порядки,
проклятое греческое царство и т. п. Понятно, что эти повторные фразы,
переплетаемыя с ругательствами, никак не знакомили 
с учреждениями, которым подчинялось в то время греческое население.
На самом же деле, ознакомление с этими учреждения не представляется делом сложным.

Переселившиеся в конце прошлого столетия из Крымского полуострова греки
основали город Мариуполь и 353 греческих села.  По позднейшим законодательным
актам они составили особый греческий округ, и населявшие его греки подчинялись
исключительно созданному для них учреждению, называвшемуся греческим судом. Это
было учреждение одновременно судебное, административное и полицейское.  Его
компетенция распространялась только на греков, другие национальности ведению
этого суда не подлежали. Состав суда состоял из председателя, и трех членовъ,
которые назывались заседателями, секретаря и подчиненных ему столоначальников.
Веё дела решались коллегиально, и только по обычаю, но не по закону;
практиковалось, что на одного заседателя возлагались полицейские обязанности,
на другого обязанности следователя, а третий заведывал хозяйственной частью.
Исключая секретаря и подчиненных ему столоначальников, остальной состав
избирался греческими поселенцами на трехлетний период.

Для этого один раз в трехлётний период каждое
из 28-х греческих поселений посылало в гор. Мариуполь
двух уполномоченных, которые, вместе с представителями города, избирали часть состава
греческого суда, т. е. председателя и трех заседателей. По обычаю, принявшему вследствие своей неизменной
повторности силу закона, избирался всегда тот кандидат, который ставил избирателям больше вина.
%page5
Так как в состав суда попадали большей частью люди, не только не блиставшие административным и судебным опытом,
но даже недостаточно грамотные, понятия которых не шли дальше вопросов о ``паленице и льне'', то,
для наставления их на путь законности и достодолжного уразумения сих путей, правительством назначался
и присылался из губернского города Екатеринослава нарочито к сему приспособленный секретарь, именовавшийся
еще стряпчим и обладавший прокурорскими правами.

Из дальнейшего разговора, происходившего между посетителями гостиницы ``Белого
Лебедя'', можно было узнать, что в конце 50-х и начале 60-х годов членом
греческого суда состоял некий обыватель гор. Мариуполя, Логафетов; он ведаль
полицейскую часть, был, таким образом, администратором, имёвшим своей задачей
споспешествовать благоустроению мариупольского греческого округа и его
обывателей. В этакой своей роли он ничего памятного потомству не оставил. Но за
то, как полицейский чин, руководивший обходами во время ярмарок в Мариуполе,
Логафетов приобрел широкую известность; молва соединяла его имя с целым рядом
мелких и крупных преступлений. Знаменитые логафетовские обходы, надолго
сохранившиеся в памяти потомства, обычно состояли в следующем: под своим
начальством Логафетов составлял отряд из сторожей греческого суда и тех молодых
людей из города, которые во время ярмарок поступали в его отряд в качестве
волонтеров. С таким отрядом, заседатель греческого суда Логафетов отправлялся
на ярмарку ``смотреть за порядком''. На самом же деле вся эта банда предавалась
разврату, кутежам и обжорству, благо, в ярмарочных кабаках и трактирах этих
охранителей порядка безвозмездно кормили и поили, в виду
%page6
их начальственного значения. Расправы же и суда искать
было негде, по дальности расстояния от высшего начальства, губернатора и по безплодности
результатов расследований, производимых присылавшимися чиновниками-ревизорами.

Однажды, во время такого начальствования на ярмарке со своим обходом, Логафетов повстречал казака
Капсуленко, который резко ответил на требование не курить.
Произошла свалка; Капсуленко пал, получив смертельную рану в бок, нанесенную ему дротиком,
который Логафетов носил вместо палки. Хотя смертельный удар и не был нанесен лично
Логафетовым, а одним из его споспешников, тем не менее, в гор. Мариуполе
все передавали за достоверное, что заседатель - убийца казака
давали за достовфрное, что заеЪдатель — убмйца казака, что
разследование по делу об этом преступлении не дало никаких результатов, вследствие того, что
Логафетов успел склонить людей своего обхода к даче ложных показаний, которыми его обелили.
Молва указывала на изворотливость Логафетова перед старой следственной властью и добавляла, что Бог
покарал некоторых нечестивцев, лжесвидетельствовавших после присяги: уверяли, будто
лжесвидетели были поражены внезапной смертью сейчас,
после дачи ложных показаний.

Обо всех этих изложенных обстоятельствах разсказывал Подлубня своим слушателям. 
Одновременно передал он и им и то, что лично над ним проделал Логафетов в 1861 году.
``Приехал я, значит, с товаром на ярмарку. - 
говорит Поддубня, — расторговался на первый сорт,
спустил всякого товару. Выручил чистоганом на деньги
5,015 рублей, запрятал деньги под жилетку и, конечно,
напилея пьянъ, потребовал музыку''...
%page7
Когда в загрязненном сором ярмарочном трактире три оборванныхе музыканта,
вооруженных двумя иструментами, похожими на скрипки и бубном, услаждали пьяный
слух пьяного Поддубни, в другом углу того же трактира, Логафетов во своим
отрядом предавался даровому обжорству и пьянству. Отуманенный чрезмёрной едой и
излишне выпитым, Логафетов сталъ буйствовать и потребовал, чтобы музыка,
услаждавшая Поддубню, немедленно стала бы играть перед ним. Как ни был бы пьян
Поддубня, но он выступил защитником своих прав, говоря, что он музыке заплатил
вперед, и что он не отпустит, ``пока она не отыграет своего''.

Произошло недоразумение, в результате которого Поддубня оказался выброшенным с
завязанными назад руками в какой-то полутемный чулан, названный арестантской
камерой.  Здесь Поддубня заснул крепким пьяным сном, от которого очнулся часов
чрез пять, когда какой-то неизвестный человек стал выталкивать его на свободу.
Ужас охватил узника, когда он, ощупав у себя под жилетом, замтиль отсутствие
денег. Он сразу отрезвелъ и понял, что в конец разорен, так как там под жилетом
находилась выручка за весь полностью проданный товар, он понял, что теперь не
было ни товара, ни денег. Осталась нищета. Поддубня обратился за помощью к
местным властям, но это оказалось бесполезным. ``Там проклятое греческое
царство'' - объяснил он своим слушателям, - ``Логафетов богат, у него большая
родня, знакомства, наш брат ничего не поделает''...  Писал Поддубня жалобу и
губернатору, что вызвало разследование особо командированного чиновника,
который, исписав много бумаги, ни к какому результату не пришел. В общем же
жалобы на Логафетова Поддубни и других лиц, вызвали
%page8
устранение его от исполнения обязанностей члена греческого суда. От этого,
конечно, не было легче Поддубне: деньги его пропали. ``По миру пустил,
проклятый... грабитель'' ... - заключил свой разсказ Поддубня, уснастив свое
заключительное слово теми богатыми по своим оттенкам ругательствами, которые,
(хотя еще и теперь нередко слышатся на улицах в больших даже городах) нетерпимы
в печати.  Когда Ноддубня кончил, Ильяшенко вмешался в разговор. Он предложил
Поддубне ``взяться'' за его дело и пояснил, что раньше с успехом вел пред
начальством дела молокан.  Действительно, безспорными официальными данными
устанавливается, что Ильяшенко посещал колонии молокан, осматривал их на манер
ревизора-чиновника, выражал им иногда свое благоволение, похвалы и обещания
предстательства пред высшим начальством в Петербурге, какового, конечно, он
никогда не исполнял и исполнять не мог.  иеполняль и исполнять не могъ.  При
этом достойно замечания, что Ильяшенко в этом случае действовал ``без всякой
корыстной цели и даже без особой побудительной причины''. (См. в архиве
екатеринославского окружного суда: журнал решений за 1965 г. л. 29 л. д. 14
об).

На предложение Ильяшенко взяться за дело, Поддубня ответил отказов, заметив...
``с чем же вести дело, когда я остался гол как сокол!'' ``ну так знайте'', возразил Ильяшенко, ``я вам этого самого 
Логафетова в кандалах чрез Бердянск отправлю в екатеринославскую тюрьму!''
До сих пор разговор происходил сравнительно
спокойно и на вы, но, после слов, сказанных Ильяшенко,
Поддубня впал въ патетический тон и перешел на ты:
``будь благодетель, запри его в тюрьму, последнюю сорочку сниму, отдам!''... завопил Поддубня. ``Ничего не надо,
только поставишь могарыч'', великодушничал Ильяшенко, 
%page9
после чего внезапно познакомившиеся заключили друг друга в объятия, 
и стали пить воду из вновь принесенной бутылки. Возле этого сосуда объединилась вся компания из
четырех человек, и дружно вела громким шопотом какой-то длинный разговор.

День 5 апреля 1863 года являлся одним из тех весенних дней, когда зима делаетъ
последнюю вылазку против наступающего лета.  Было ``сиверко'', как говорит наш
простой народ; небо заволокло серыми тучами, которых не мог прогнать
пронизывающий, ни на одну минуту не ослаблявший своей порывистости
северо-воеточный ветер; степь не успела еще покрыться зеленью, и вместе с серым
небом, представляла как бы сплошную серовато-бурую массу. Если-бы не длинные
дни, да не особое весеннее освещение, которое, несмотря на закрывшие солнце
тучи, таки ощущалось, можно было бы подумать, что времена Года повернули
вспять, что настал ноябрь месяц. Едва образовавшийся ``накат'' по дорогам
испортился, благодаря безпрерывно сыпавшемуся на землю мелкому дождю.  Среди
такой обстановки, в указанный день 5 апреля, по дороге из Бердянска в
Мариуполь, верстах в двадцати от последнего города, тащилась одноконная
подвода, на которой сидели уже известные нам лица: купець Мазин, мещанин
Колосовский и отставной чертежник Григорий Власов Ильяшенко. (См. в архиве
екатеринославского окружного суда дело, вступившее в палату из александровского
уездного суда обще с городовой ратушей об отставном чертежнике из
обер-офицерских детей Гр. Вл. Ильяшенко л. д. 12).

Верст более 50 по невеселой дороге ехали они; им оставалось до Мариуполя еще верст 25 такого же невеселого пути. 
Пред путниками простиралась все та же степь
%page10
однородная, серая, без деревца, без какого то ни было пейзажа, на котором можно
остановить глаз. Но наши путешественники относились к этому равнодушно и сидели
на своей невзрачной повозке, проникнутые той апатией, которая по необходимости
овладевает путником, хорошо наперед знающим, что никакими усилями, хитростями,
мольбами, едущему не сократить, ни изменить долгих томительных часов пути среди
безмолвных степей.

Около четырех часов пополудни Ильяшенко и его товарищи подъехали к Мариуполю и
остановились в предместьи города, на Марьинской стороне, в одной из нензрачных
изб; гостиниць тогда в Мариуполе не было и приезжающие останавливались в
частных квартирах. Того же дня вечером, когда уже стемнело, Ильяшенко разыскал
дом Логафетова и явился к этому последнему.
(журнал решения екатеринославской палаты уголовного суда за 1865 г., л. 29). Как мы уже знаем, Логафетов в
то время не был властью: он был устранен от должности члена греческого суда. Тем не менее он жил
припеваючи: был холост, богат, не особенно стар, ему
было около 50 лет, недугами не страдал, в городе имел
богатых и сильных родственников, среди которых он
был свой и дорогой их сердцу человек. Словом жилось
ему хорошо, беззаботно, спокойно; его служебные подвиги
ето не тревожили, ибо все расследования о его прошлых деяниях 
кончились совсем благополучно: ни о каких судебных 
преследованиях не было и речи. Вот почему Логафетов свысока отнесся к Ильяшенко, когда тот стал
ему объяснять, что ему грозят неприятности по жалобам
Поддубни, и когда Ильяшенко предложил ему свое содействие для улаживания этого дела. В конце концов
Логафетов грубо выпроводил Ильяшенко и закрывая за ним 
%page11
дверь, пропустил мимо ушей обращенное к нему восклицание: ``ты меня попомнишь!''...
Понятно, что этим словам Логафетов не придал ровно никакого значения. 

\par\noindent\rule{\textwidth}{0.4pt}

\textbf{\em Mihi non pudet fateri nescire, quod nesciam} и мне не стыдно сознаться, что у
меня нет умения писать творчески, создавая художественные типы, частично
разбросанные в жизни.  Если бы я обладал таким умением, я предпочел бы
изобразить описываемые события в форме комедии или драмы, и в этом месте
изображения действительности опустил бы занавес, закончив I-й акт.
Затем я начал бы 2-й и последний акт, так внесено было бы более интереса и жизни в настоящую работу;
но, к сожалению, это не под силам мне; я только смиренный летописец-протоколист, и восстанавливаю
забытую и, на мой взгляд поучительную историю по тому способу и приему, как составляют обыкновенный судебный или
полицейский протокол. При этом я пользовался бесспорными документами, хранящимися в архиве екатеринославского
окружного суда, просмотр коих любезно был мне предоставлен почтенным председателем
названного суда И. Ц. Патоном; затем я имею в своем распоряжении разысканную моим уважаемым 
товарищем уездным членом таганрогского окружного суда В. С. Станкевичем кореспонденцию 
``Сына Отечества'' от 20 мая 1869 года и переданную ими мне; наконец я широко воспользовался свидетельскими 
показаниями некоторых старожилов г. Мариуполя, которые были не только непосредственными 
свидетелями, но даже отчасти участниками описываемых мною событий.
%page12
Пользуюсь случаем принести мою глубокую благодарность этим старожилам, из
которых некоторые не пожелали, чтобы я упомянул их имена; я их благодарю за их
длинные, подробные показания и главным образом за поразительную верноеть и
точность их свидетельств, в чем я убедилея сопоставляя их показания
документальными данными, хранящимися в архиве скатеринославского окружного
суда. После этого объяснения продолжаю прерванный протокол.

6 апреля 1863 года писарь, заведывавший канцелярией 
\textbf{\em ``Начальника Мариупольской команды внутренней стражи''},
по заведенному порядку, в 8 час. утра, явился в небольшую комнату, на дверях которой был прибит полу-лист
бумагит с криво выведенною большими буквами надписью: ``Канцелярия''. Писарь был из молодых; эту должность он занимал только с 1862 года.
Раньше он входил в состав команды, сформированной в г. Екатеринославе из 40 рядовых; команда была сформирована в 1858 году, и в тот же год была прислана в город Мариуполь под начальством ее постоянного начальника капитана Лисенко. 
В команду, о которой идет речь, входили солдаты лучшей репутации, имевшие нашивки. По свидетельству старожилов
в то время господствовала строжайшая военная
дисциплина: ``достаточно было чихнуть во фронте, чтобы
получить в морду от начальства'', привожу подлинные
слова одного бывшего солдата команды. Замечательно при
этом, что, давая такие показанля, старожилы из бывших
солдат тем не менее добавляют: ``жить было прекрасно''.
Далее они объясняют что все знали друг друга, что 
``воровства, грабежа, мошенничества, извозчиков и гостиниц
не было''. Очевилно, значит, что проделки и преступления Логафетова в счет не шли: вероятно, их
игнорировали из уважения к его начальственному состоянию.
%page13
Писарь войсковой команды отличался как образцовый ``службист'' и за образцовое
исполнение дисциплины, в 1860 году, был произведен в унтер-офицеры. Обязанности
писаря он исполнял с тем же неизменным усердием, с тем же строгим исполнением
дисциилины, как и обязанности рядового.

Прийдя 6 апреля в свою канцелярию, писарь подошел к большому, неуклюжему, грубо сделанному,
косому шкафу, достал из него несколько толстых тетрадей в потертом переплете и положил их на тут же
стоявший большой стол, такой же неуклюжий, как и шкаф.

Затем, обмакнув в пузырек с чернилом гусиное перо, он стал этим скрипучим
орудием выводить буквы на серой, немного мохнатой и похожей на войлок бумаге,
из которой состояли книги для входящих и исходящих бумаг.

``В то время'' (так показывает бывший писарь, ныне благополучно проживающий в гор. Мариуполе), 
\begin{quote}
\bfseries\em
твердым шагом, не спеша, но и не медля, входит приличный, 
солидный, с бравым видом господин лет тридцати. Роста он был выше среднего, блондин, одет в плохонькое пальто
светло коричневого цвета, заметно потертое; в такие пальто обыкновенно наряжены приказчики, стоящие за прилавком
\end{quote}

Неизвестный, войдя в канцелярию, поздоровался с писарем; последний встал,
вытянулся, поклонился, молча сел и стал опять усердно выводить буквы, но душа
его была уже охвачена какой-то неясной тревогой.

Писарь почувствовал, что необычайное появление нензвестного господина
неспроста. 

Между неизвестным и писарем в это время завязался следующий диалог.

\textbf{\em Неизвестный}: - Кто начальник?
%page14
\textbf{\em Писарь}. - Штабс-капитан Лисенко...

\textbf{\em Неизвестный.} - Хорошо-ли обращается с солдатами?

\textbf{\em Писарь.} - Так точно, все обстоит благополучно, позвольте доложиться начальн...

\textbf{\em Неизвестный} (резко обрывая). - Не надо останься...  Есть писчая бумага?

Уже после второго вопроса писарь понять, что предчувствие его не обмануло; 
ему показалось очевидным, что пред ним какой-то большой начальник, \textbf{\em ``ибо никто иной,
как только начальник станет спрашивать, как обращаются с солдатами''}; от этой мысли он заволновался,
\textbf{\em ``весь затряеся''} и дрожащею рукою положил на стол 6 листков белой бумаги.

Между тем, ``большой начальник'' уже отдал сухо приказ:

— Напиши на этой бумаге, что я скажу...

Оторопь охватила писаря, в голове мелькнула мысль: удрать, убежать, но это была
только мысль без решимости; на самом деле писарь боялся тронуться с места. Он
только смог произнести тихим, умоляющим голосом:

— Позвольте доложить начальнику команды.

Но на эту мольбу неизвестный еще более сухо, повысив голос, ответил:

— Не надо... Пиши!

Писарь покорно селъ перед листом белой бумаги.  Неизвестный, поглядывая в
пямятную книжку, вынутую им из внутреннего кармана своего верхнего пальто, стал
диктовать, а писарь записывать следующее:

\begin{quote}
\em\bfseries

По данной мне власти Государя Императора Всероссийского Александра Николаевича,
в Царском Селе, 18 мая 1862 года, лицом и именем которого повелеваю:
бывшего заседателя сего суда Николая Логафетова за грабеж и
смертоубийство
\end{quote}

%page15

Но силы писаря ему изменили. Уже когда он выводил слово: \enquote{Государя
Императора}, ему сперло дух и он чувствовал, как земля уходит из под его ног.
Когда же были упомянуты преступления Логафетова, о которых шопотом и почему-то
со страхом говорилось в городе, как будто-бы все были соучастниками его
преступлений, писарь всем своим существом постиг, что перед нимъ - великую
власть имущая персона, и так заволновался, что его руки с гусиным пером среди
стиснувшихся пальцев запрыгали на бумаге. Начатый лист был испорчен появившейся
на нем чернильной кляксой и до неузнаваемости безобразно выведенными буквами
последних слов, отражавшими пляску руки на бумаге.

Неизвестный прекратил диктант и повелительно сказал писарю:

— Ветань, оправься, пройдись по комнате, не надо дрефить!

Пока писарь оправлялся, неизвестный стал около двери, отрезав путь к бегству. 

— Ну, теперь пиши, - опять приказал ноизвфетный, и, после вписания на чистый
лист уже написанного, продолжал диктовать следующее:
\begin{quote}
\em\bfseries
и вообще за все злоупотребления лишить всех прав состояния со ссылкою на Алтайские заводы 
в вечные работники, имение продать с публичного торга и удовлетворить всех должников и претендентов; а остальное затем
должно поступить в казну.
\end{quote}

Просмотрев написанное неизвестный собственноручно ``подписует'' его так: 

\textbf{\em ``Полномоченный Государя моего, верноподанный Григорий Власов Ильяшенко. Мариуполь, апреля 6 дня, 1865 года''}
Подпись эта не произвела впечатления
%page16
на писаря, который, после перенесенных тревог, перестал временно реагировать и
только пассивно продолжал под диктовку, еще два приказа, текст коих я приведу в
точности несколько далее.

Все 3 приказа, написанные на 3-х отдельных листах, Ильяшенко спрятал во
внутренний карман своего пальто: затем, став вплотпую пред писарем и сказав
``cмотри'' расстегнул верхние пуговицы жилета и вынул изъ под него складной
медальон из красной меди, висевший на Андреевской ленте.  В этом медальоне был
портрет Государя и под крышкой кусочек белой бумаги на коей имелась надпись:
\textbf{\em ``быть по сему, Александр 2-й, Царское село, 17 мая 1862 года''}.
Эти слова были написаны обыкновенным почерком Ильяшепко, как это удостоверено
судебной экспертизой и как это было очевидно впоследствии для всякого
обозревавшего настоящую подпись. (Журнал решений Екатер. Палаты угол. суда за
1865 года, л. 29)

``Смотри'', говорил Ильяшенко; наступая на писаря:
\begin{quote}
\em\bfseries	
ты знаешь, кто я таков, видить, чем я награжден от
Государя. Знай, что если ты известишь начальника команды до ревизии мною суда,
сегодня же будешь повешен!
\end{quote}

С этими словами, произнесенными с зловещим грозным шепотом, Ильяшенко быстро
вышел из канцелярии.  У писаря отнялся дух, опять потемнело в глазах, застучало
в голове: не донести начальнику - беда, донести еще хуже, мелькнуло в его
сознании; недвижимый, словно окаменелый, остался он в своей канцелярии, как
будто тысяча железных цепей приковали его к месту.

\par\noindent\rule{\textwidth}{0.4pt}

Николай Фотиевич Логафетов, не обратив никакого внимания на угрозу почти
вытолкнутого им из своего дома Ильяшенко, на утро совершенно о нем позабыл. 
%page17
По обыкновению хорошо выспавшись, он встал рано не спеша стал одеваться, поверхностно 
умылся и предался спокойному чаепитию.

Часов около 10 утра он отправился в соседнюю лавку своего приятеля К...нова, где он бывал каждое утро
и где каждое утро приятели вели один и тот же разговор.
Во 1-х, они говорили, что жилось бы гораздо лучше, если
бы всегда можно было быть уверенным купить товарь дешево и продать дорого. Во
2-х, оба они печаловались друг другу о том, что в Мариуполе начинают селиться пришельцы из разных концов Империи, не греки;
за ними, говорили между собою приятели, и на базар не успеешь ничего купить;
при этом они вспоминали, как остроумно их общий друг, некий местный обыватель, выразил протест 
против такого вопиющего положения вещей. Этот их
друг, идя на базар купить курицу, увидел, что одна из
вновь поселившихся русских женщин возвращалась с базара и несла в корзине курицу. 

Такое предупреждение намерений греческого поселенца возмутило его дух; он остановил своего конкурента в деле покупки курицы
и, выхватив из корзины птицу, бросил женщине 20 копеек, причем был настолько галантен, что представил ей объяснение своего поступка:
``за вами чертями ничего не успеешь купить!''
``Еще 20 коп. уплатил'', добавлял Логафетов тем тоном, из которого 
безусловно явствовало, что он лично ограничился бы только отобранием
курицы без всяких дальнейших действий. В особенности
же Логафетов и К-в в своих собеседованиях возмущались тем обстоятельством, 
что пришельцы открывали в
городе лавки и являлись в торговле опасными конкурентами. В то сравнительно
недавнее время уеловия общественной жизни в Мариуполе не походили на сегодняшние.
%page18
Сегодня смешно говорить о Мариуполе, как исключительно греческом городе; в нем
нет серьезного различия по национальностям: все обыватели этого города
постененно смешиваются и во всяком случае положение всех обывателей равны перед
законом и властями. Тогда по свидетельству сларожилов, дело обстояло иначе.
Пришельцы не греки, составляли ничтожное меньшинство, которое побаивалось
греков, представлявших большинство, силу, с которой не могли бороться новые
поселенцы. Кроме того, греки имели свое самоуправление, свою обусловленную
привилегированную организацию, которая не распространялась на невыходцев из
Крыма. Последние подчинялись общим дореформенным учреждениям, под сенью которых
жилось не легко. Так продолжалось до начала семидесятых годов, когда введено
было городовое положение на общих основаниях. До этой же поры греки, выходцы из
Крыма, занимали господствующее положение, а остальные обитатели угнетенное.
Греки были недовольны пришельцами, а последние их побаивались и питали к ним
недобрые чувства.

И так Логафетов и К-ов затянули свой обычный разговор; на этот раз он был скоро
прерван. Из греческого суда прибежал старик сторож и на турецком наречии
(многие из греков поселенцев и теперь сохранили это наречи) передать
Логафетову, чтобы он поспешил придти в суд, что его требует председатель Попов.

Логафетов направился в суд совершенно спокойно.
Случалось и раньше, что его требовали иной раз в присутствие
для некоторых неважных разъяснений, которые он мог представить, как бывший член суда;
правда, что чаще эти объяснения переводились с места на собеседования по душе, которыми совершенно
затушевывалась главная цель вызова. Вот почему, даже не без некоторого удовольствия, Логафетов направился 
%page19
в присутствие, предвкушая хотя всегда один и тот же, но тем не менее всегда
одинаково для него интересный разговор о торговле, о ``пашенице'', о зловредных
новых людях, поселившихся недавно в Мариуполе.

В это время один из таких новых поселенцев,
почтенный купец -ов (ныне благополучно проживающий
в Мариуполе, развивший и многократно возвеличивший свое
предприятие) пришел в свою лавку и сел по обыкновению
на табурете, стоявшем у двери, выходившей на улицу. 

Моросил мелкий дождь, но ветер, свирепо дувший накануне с северовосточной
стороны, изменил направление, стал дуть с юго-востока и в воздухе начала
разливаться весенняя, оживляющая мир, мягкость; предшествующий холодный день
оказался на самом деле последним приступом зимы, утомившей непривыкших к холоду
южан своими длинными бездеятельными днями.

Купець -ов, смотря на улицу, среди мертвой тишины, 
громко передавал себе свои впечатления: 

\begin{quote}
\em\bfseries

И куда это
Иван Павлович (начальник инвалидной команды, капитан Лисенко) поехал в дрожках?
А вот за ним четыре солдата бегут рысцой, ружья держат под шинелью, чтобы дождик не замочил. А среди-то всех во как
ковыляет ногами старый солдат цирюльник: ему то сердечному, старому и хилому, не легко поспевать за строевыми!
\begin{quote}

Так в мыслях купца -ова отразилось виденное
им на улице и он осталея сидеть спокойно на своем
месте, праздно и разсеянно размышляя на тему: куда бы это
поехал Иван Павлович?

Эти размышления были по прошествии некоторого промежутка времени, прерваны появлением мастерового 
Анохина, имевшего ошалелый вид. В согласии с его необычным внешним состоянием были и его действия: войдя в лавку, 
		он начал, без всяких к тому побудительных причин, учащенно и усердно креститься.
%page20

— Чего крестишься? — вопросил удивленный купец -ов.

В ответ посыпались отдельные слова: \emph{Чудеса Капитан! Саблю наголо!..
Четыре солдата?.. Логафетов и т. п.}

— Да ты с ума сошел. — прервал безевязный поток слов купец —ов и увел расстроенного 
посетителя в следующую комнату, где шепотом Анохин продолжал столь странно начатое  
повествование.

Я уже заметил, что когла речь заходила о деяниях Логафетова, то мариупольские
обыватели говорили шепотом и со страхом, как будто они были
соучастниками Логафетова. Анохинъ же не только усвоить эту общую всем
привычку, но сверх того был приведен в трепет виденным и опасался как
бы не быть в ответе за то, что он рассказывает виденное... Впрочем,
восстановлять события по отрывочным словам и фразам Анохина я не берусь
и предпочитаю возвратиться к имеющемуся у меня точному, достоверному
материалу.

Ильяшенко, оставив писаря в канцелярии, направился
в греческий суд; там он застал одного столоначальника
уголовного отделения суда, Могулянского. Последнему, как
говорят, Ильяшенко сразу поставил вопросы ребром и
налег на него со стремительностью. ``Где председатель и
члены суда''? — ``Составь суда выезжает сегодня для разбора
дел в Ялту''. — ``Вы коронный''? (т. е. состоите ли на 
коронной службе или выборный). - ``Да, коронный''... Ильяшенко
показаль портрегь Государя, после чего столоначальник
превратился в воплощенный вопросительный и восклицательный знаки. ``Ни слова''! - грозно крикнул Ильяшенко. —
%page21
``Немедленно послать за председателем и членами''! Столоначальник пустился
бегом исполнять приказ. Состав суда (председатель и три члена) в это
время состоял из людей основательных, уже немолодых, умудренных, так
сказать, годами. Один секретарь был молод, но зато он был умудрен
знатемъ на память некоторых законов и наипаче циркуляров; это
признавали все и более всех уверенно сам секретарь.

Члены греческого суда, надо сознаться, при всей солидности, страдали полным
неумением составлять суждение о делах, подлежащих их рассмотрению, но
это не препятствовало функционированию названного учреждения, ибо в
этом случае выручал секретарь со своими циркулярчиками и законами. Он
писал определение, ссылаясь на проставленные им статьи, а председатель
и члены суда медленно и старательно их подписывали, не постигая их
содержания, что, однако, не задерживало течения дела. В остальном
секретарь ничем достопримечательным не выделялся.  Можно только
отметить, что уже прежде познания законов он постиг, что как законы,
так и циркуляры существуют для угождения начальству.

Не прошло и двадцати минут после ухода Могулянского, как запыхавшись, но уже в
настоящем одеянии и при надлежащем знаке, явились: председатель, три
члена суда и секретаръ. Ильяшенко продолжаль действовать с тою же
стремительностью, с какою он обратился к столоначальнику. Едва
поздоровавшись, он вручил председателю тот приказ, который он
продиктовал писарю и текст коего приведен выше.

Председатель навел глаза на врученную ему бумагу; соображал он туго, медленно, но все таки довольно
скоро понял, что случилась беда, и в душе у него похолодело;
%page22
он не знал, что сказать, что делать. Ильяшенко вывел
его из состояния нерешительности, потребовав дать ему
надежного человека, с которым он мог бы послать важную и секретную бумагу
начальнику команды, капитану Лисенко.

Председатель позвал одного из сторожей, которому Ильяшенко и вручил в
запечатанном конверте второй приказ, зараннее написанный под его диктовку,
писарем инвалидной команды. Председатель и по русски, и на турецком наречии
напутствовал посланного наставлениями. Ильяшенко затем объявил председателю,
что суд не поедет сегодня в Ялту, так как ему предстоит заняться рассмотрением
чрезвычайного государственного дела о Логафетове, которого Ильяшенко потребовал
немедленно призвать в присутствие суда. Остальные неопределенные, но
наставительного характера, общие указания весь состав суда выслушал, стоя и
молча, находясь в состоянии людей, которых пришибли сильным ударом по голове.

\par\noindent\rule{\textwidth}{0.4pt}

Капитан Иван Павлович Лисенко был старый служака: еще до севастопольской войны
он служил в кантонистах и за усердную службу был назначен фельдфебелем
селенгинского пехотного полка. Во время севастопольекой
войны он храбро сражался, был ранен и произведен в
подпоручики со старшинством. Когда в 1858 году была
сформирована команда из сорока рядовых, четырех унтер-офицеров, к которым впоследствии был прибавлен один
барабанщик и один пирульник, Лиеенко, в чине поручика, был назначен начальником этой команды. В том 
же 1858 году команда из Екатеринослава совершила переход в Мариуполь, 
%page23
где и оставалась под начальством
Лисенко, в скорости произведенного в капитаны. Таким
образом в мариупольском округе капитан Лисенко оказался единственным и 
властным представителем военной
власти. С виду Лисенко был настоящий богатырь: в плечах сажень, 
грудь как большой котел, а ростом он
был выше на голову каждаго из нижних чинов своей
команды и даже каждого из обывателей города Мариуполя;
голос его вызывал содрогание даже неодушевленных предметов. 
Лисепко мог пить, но всегда был трезв. К своей
службе относился строго; знал и исполнял дисциплину
в совершенстве. На парадах, при маршировке, при фронте
священнодействовал. К солдатам относился строго, справедливо и заботливо. Соблюдение дисциплины требовал
неумолимо и карал за всякое малейшее ее нарушение, в конец расстраивающее расположение его духа. Даже 
лучшего и образцового унтер-офицера он наказал на первый день Светлого Праздника за то, что он 
на параде, поставленный против ярко-весеннего солнца, ``натужился'' и чихнул, после чего Лисенко, разгневавшись, сейчас
же прекратил парад, считая дело испакощенным. 

Лисенко привык, умел и любил исполнять безпрекословно приказания и при этом
знал все, что от него требовалось. Одного он не знал и не умел за полным
отсутствием  практики в этом отношении: он не умел думать, и всей душой
ненавидел такое положение, когда ему приходилось отвечать на вопрос как
поступить, ибо он мог только поступать, но не обсуждать, как поступать. Такъ,
например, если бы он получил приказ перебить мариупольских обывателей, ему
легче было бы исполнить такое требование, чем обдумывать и размышлять над
вопросом, следует ли, возможно ли исполнить такой приказ.
%page24

Жена Лисенко, Дарья Кондратьевна, была основательная дама, сорока восьми лет 
(Арх. екатер. окружн. суда, указанное выше дело, л. д. 64). Грамоты она совсем не знала.
По крайней мере, под ее показанием, данным следственной власти,
 вместо ее подписи значится отметка следователя:
``не грамотная''. Зато она была несравненная хозяйка,
жизнь которой хлопотливо протекала среди неизменно каждодневно чередующихся 
забот о кухне, погребе, о соленьях,
о борще и т. д. Гигант капитан Лисенко по внешности
был под пару своей жене. Как почти все мужья, капитан слегка побаивался своей жены; 
последняя питала почтение, главным образом, к начальственному положению своего мужа и менее к его
личности. Несмотря на свое умственное неразвитие, капитанша вовсе не относилась к службе и деятельности
своего мужа с тем презрительным высокомерием, которым угощают нередко внешне приличные, но в душе полудикие
женщины, вышедшие замуж за писателей, художников, мыслителей и т. п. выдающихся людей.
Дарья Кондратьевна была неразвита, но не обладала дикой душой, стремящейся к разрушению.
Поэтому она мирилась, как с необходимостью, и с фронтом, и с маршировкой и с приносимыми из капцелярии
бумагами на квартиру в капитану. Большим необходимым
злом капитанша считала приносимые на дом бумаги, ибо
они всегда вызывали некоторое беспокойство и тревогу капитана, 
а через то и некоторый безпорядок в дом: капитан, поглощенный подписом бумаг, 
запаздывал к обеду, кричал, звал, отдавал распоряжения и т. д. Мирясь
со всем этим, Дарья Кондратьевна, хотя и не высказывала,
все таки в глубине души смотрела на все эти бумаги и безпокойства, 
как на дело пустое; она понимала, что из всего этого не
выйдет ни хорошей начинки для пирогов, ни солений, ни борща.

%page25
Вот почему, когда она первая встретила посланного
из греческого суда с приказом Ильяшенко, она пренебрежительно равнодушно приняла бумагу
двумя пальцами в
тех видах, чтобы на бумаге осталось поменьше бурякового
кваса, в который были смочены ее руки, и бросила эту
бумагу на пропитанный жиром кухонный стол. Об этом
моменте вот как свидетельствует сам капитан Лисепко.
Привожу доподлинное его показание, сохраняя его стиль и орфографию:

\begin{quote}
\em\bfseries
	
Приказ Ильяшенко прислан из суда неизвестным человеком, передан в руки моей
жены на крыльце во время бытности ее в кладовой... сказала мне, что
требуют в суд, сама же как говорит, понесла продукты в кухню, в то
время я быль в другой комнате переодевался из одежды (Л. д. 60
указанного выше дела).

\end{quote}

Текст этого приказа, написанного как мы знаем, писарем
команды, под диктовку Ильяшенко, был следующий: 

\begin{quote}
\em\bfseries

Государь Император Высочайше повелел соизволить, в Царском селе, 13 мая 1862
года, приказал вам исполнить мое требование, в отношении спокойствия и
прекращения злоупотребления в Новороссийском крае. Ныне поручаю вашему
благородию приготовить с вверенной вам команды четыре человека с одним
унтер-офицером и одного нестроевого цирюльника, при которых должны
находиться пара кандалов и арестантское платье. Все прописанные нижние
чины и вы сами лично обязаны явиться в тот час и минуту как получите
сей приказ, в мариупольский греческий суд, где, приняв преступника, с
первым этапом отправить к месту назначения за строгим караулом, а также
поручаю вам на будущее время, в случае важных происшествий в Мариуполе
и ближайших ему окрестностей, немедленно донести министру внутренних
дел, статс-секретарю Валуеву минуя свою прямую дистанцию.
	
\end{quote}
Следовала
%page26
затем подпись Ильяшенко, тождественная с подписью на
первом приказе, только после нее рукою Ильяшенко было
добавлено: \textbf{\em ``В десять часов одиннадцать минут, город
Марbуполь, шестого апреля, 1863 года''}.

В своих дальнейших показаниях капитан сознается, 
что он не обратил внимания на то обстоятельство,
что приказ написан хорошо ему известным почерком
писаря его команды. Произошло это вследствие того, что
слова приказа: \emph{``Государь Императоръ Высочайше повелел
соизволил''} сильно \emph{``ветревожили''} капитана.

По показанию старожилов, внутренная тревога Лисенко
проявилась наружу следующим порядком: голосом, способным на смерть ушибить 
не подготовленного слушателя,
капитан стал кричать: \emph{``одеваться, олеваться''}!... Многократным
повторением этого слова капитан хотел объяснить, 
чтобы ему помогли скорее облачиться в парадную форму.

Дарья Кондратьевна, хотя и была привычна к голосу своего супруга, однако,
подскочила несколько раз на месте, как токующий тетерев, испуганная неслыханным
еще столь сильным криком капитана: она даже обронила на земле те ``продукты'',
что держала в руках; тем не менее она скоро оправилась и устремилась к
капитану, который, спеша, никак не мог справиться с переодеванием в парадные
узкие брюки.  Ее помощь была как нельзя боле кстати задыхавшемуся от волнения
ее супругу.  Пока он окончательно подправлял себя, Дарья Кондратьевна успела
отдать приказание кучеру, чтобы он запряг лощадь и позвал четырех солдат,
цирюльника и барабанщика.

Хотя волнение и спех и помешали действовать с той
быстротой, какая требуется военными людьми, тем не менее
%page27
капитан достаточно скоро появился во двор в надлежащем виде и, не теряя ни
минуты, стал во главе четырех застывших пред ним солдат, цирюльника, и
барабанщика, с которыми и двинулся в путь в томъ порядке, в каком их видел
купец —ов, когда он из своей лавки смотрел на улицу. Капитан, очутившись на
улице и освеженный слегка моросившим дождем, оправился от волнения и чувствовал
в своей душе ту отвагу, ту решимость немедленно дфйствовать, какой он отличался
в геройские дни севастопольской войны, когда надо было сходиться грудь с грудью
с неприятелем.

\par\noindent\rule{\textwidth}{0.4pt}

После того как, по приказанию Ильяшенко, из греческого суда 
был послан один надежный человек с
извёстным нам приказом к капитану Лисенко, а другой - за Логафетовым, 
сам Ильяшенко не терялъ времени. Прежде всего он приказал председателю про себя
внимательно вчитаться в переданный ему приказ № 1-й,
коим, как нам известно, Ильзшенко определял сослать
Логафетова \textbf{\em ``на алтайские заводы в вечные работники''.} 
Затем он приказал смущенному и в глубине дубине души 
разъедаемому сомнениями секретарю суда, стряпчему Хартахаю,
составить от имени греческого суда определение об осуждении на каторгу Логафетова.

Секретарь пугался и от необычайного составляемого
им приговора, и от казавшегося ему несогласия определения суда с известными
ему законами и циркулярами, и, главное, от стремительности действий суда, упразднявшей всякую судебную
логику.

Но Хартахай в этот трудный момент не забыл
надежно сидевшей в его душе идеи, что все должно совершать для угождения начальству,
и он решил не противоречить Ильяшенко, тем более, что последний прибег
%page28
к довольно грозным покрикиваниям и понуканиям, когда
секретарь обнаружил замешательство. Вот почему определение быстро продвигалось вперед,
сопровождаемое иногда и одобрительными восклицаниями, в роде: ``так, верно,
пиши дальше''.... и было окончено к приезду капитана Лисенко.

Случилось так, что в присутствие греческого суда капитан Лисенко вошел почти
одновременно вслед за Логафетовым. Таким образом, все главные действующие лица
настоящей драмы оказались собранными вместе друг против друга. Такое их
стечение оказалось в высшей степени благоприятным для миссии Ильяшенко.
Это явствует из данных впоследствии объяснений действующих лиц по вопросу о возникновении \emph{``слепого верования''}
в \emph{``лжеуполномоченного Ильяшенко''}.

Капитавъ верил потому, что был требован в присутственное место, где все исполнялось 
по приказанию Ильяшенко, (Там же л. д. 17 об.). Председатель суда окончательно доуверовал, 
благодаря
\emph{``внезапному появлению начальника инвалидной команды капитана Лисенко, 
да еще с солдатами и в парадной форме''};
члены суда уверовали \emph{``смотря на председателя и секретаря
Хартахая, повиновавшихся привазаниям Ильяшенко''}. (Там
же л. п. 16 об.). Наконец председатель Попов, в последних своих объяснениях еще
указывал, 
что укреплению веры в ``лжеуполномоченного'' содействовал и Логафетов, который ``стоял с повинной головой, как настоящий
виновник''.
Правда, такое объяснение вызвало негодование
Логафетова, который подал жалобу на Понова и писал,
что эти слова Попова, он ``принимает за преднамеренную
обиду на письме принесенную; невольно влекусь к подозрению'' (там же л. д. 69—72), - продолжал он 
жаловалься, что Попов действоваль неспроста, что у него
были задние мысли... Но последние, по тщательному розыску,
%page29
найдены не были.

Едва капитан Лисенко вошел в присутствие суда,
вслед за Логафетовым, как Ильяшенко, указывая пальцем на 
Логафетова, твердым и решительным голосом
обратился к капитану: \emph{``именем закона приказываю вам
арестовать Логафетова''}. Решимость действовать проявилась
наружу: капитан молодецки выхватил саблю из ножен
и, держа в руке это оружие поднятым вверх, громовым
голосом скомандовал солдатам стать с ружьями у дверей
и никого не впускать и не выпускать из присутствия
суда. Наступила пауза, Ильяшенко один нарушил воцарившуюся 
гробовую тишину, объявил секретарю, что он
должен приготовиться к прочтенио определения суда. Хартахай сейчас 
поднялся и начал, не ожидая дальнейших
приказаний, без всякой торжественноети, монотонно читать
написанный им приговор. Ильяшенко с места его оборвал 
и грозно возопил: \emph{``Мальчишка, службы не понимаешь!
Не знаешь как, когда следует читать!''}... Мгновенно восстановилась 
мертвая тишина. Ильяшенко достал из-под
сюртука медальон с портретом Государя и, держа его
перед собою, скомандовал: ``На караул!'' Сабля капитана
Лиеенко сделала вольт в воздухе и он сам, равно как
и его солдаты, превратились в застывшие статуи. — \emph{``Теперь
можешь читать, только смотри толково читай''}, - строго обратился
Ильяшенко к секретарю; последний на этот раз
блатополучно прочитал приговор, который корреспонденту
``Сына Отечества'', в номере от 20 мая 1869 г., передает \emph{``на память потомству, как замечательный документ
в летописях суда''}. Привожу это определение суда дословно, без сокращений. Вот оно:
%page30
\begin{quote}
\em\bfseries
	
1863 года, апреля 6 дня. По Указу Его Ииператорского Величества, мариупольский греческий суд слушали
приказ уполномоченного Государя Императора Всероссийского,
Григория Павлова Ильяшенко, от 6 апреля, в коем объявлено 
по данной ему Государем Императором власти в
Царском Селе, 18 мая, 1862 года, лицом и именем
Его Величества повелевает: бывшего заседателя сего суда
Николая Логафетова, за грабеж и смертоубийство, и вообще
за все злоупотребления, лишить всех прав состояния с
ссылкою в алтайские заводы в вечные работники, а имение
его продать с публичного торга и удовлетворить всех
должников и претендателей, а остальное затем должно
поступить в казну. Приказали: решение это объявить бывшему заседателю сего суда 
Логафетову и передать его, согласно личному приказанию господина уполномоченного, в
распоряжение начальника мариупольской инвалидной команды,
штабс-капитана Лисенко, а исправляющему должность заседателя Ганжи 
предписать описать и оценить все имение
Логафетова и опись представить в суд. Председатель Николай Попов. 
За секретаря А. Ганжи, заседатель Оксюзов,
исправляющий должность секретаря и стряпчего Хартахай.
Не согласен на исполнение и буду телеграфировать г. начальнику губернии. 
Исправляющий должность стряпчего Хартахай. В присутствии мариупольского греческого суда постановление 
утверждаю. Полномоченный Государя Императора
верноподанный Григорий Власов Ильяшенко. 6 апреля 1863
года в $11 \sfrac{3}{4}$ часов слушали и подписали.

\end{quote}

Обстоятельства, которые будут указаны в дальнейшем
изложении обнаружат, что приписка Хархатая: ``несогласен
на исполнение и буду телеграфировать начальнику губернии'',
позднейшего происхожденя. Первоначальное определение суда
было написано без таких либеральных оговоров и, посль
%page31
скрепления этого приговора подписью Ильяшенко и его громким возгласом: 
\emph{``быть по сему''}, начался немедленно обряд приведения 
приговора в исполнение.

— \textbf{\emph{Цирульника, бритву, барабан!}} — быстро и громко скомандовал Ильяшенко.

Цирульник, тут же стоявший возле солдат, побледнел: бритва была им забыта
впопыхах на квартире команды.

Барабан только числился при команде, но в натуре его ни было.

Не давая объяснений, цирульник вылетел из присутствия суда и что есть мочи
устремился на квартиру за бритвой. В два конца ему пришлось побежать более
версты, но он себя не щадил и через каких нибудь 10 минут вернулся еще более
бледный, без дыхания, держа в дрожащих руках какую-то ободранную, изъявленную
зазубринами бритву.

Пока шли краткие приготовления к обряду исполнения приговора, Логафетов
переживал тяжелые минуты. В первое мгновение он не полностью постигал ожидавшую
его участь: он хотелъ верить и верил, что случилось какое-то неприятное
недоразумение, которое сейчас само собою разъяснится; но ход событий с его
стремительным натиском подрывал эту веру. Страх и отчаяние в конце концов
охватили бывшего заседателя суда; он не знал, что делать, что сказать в свою
пользу и в этом подавленном состоянии, на местном турецком наречии обратился,
в поисках за помощью к председателю суда, но тот не оказался добрым утешителем:
указывая на Ильяшенко, председатель только сказал:

— Власть его выше закона.

Взгляд Логафетова случайно упал на принесенную
бритву и это вызвало у него мысль заявить Ильяшенко,
\textbf{\emph{``что, кажется, теперь не велено брить''.}}
%page32

Стряпчий Хартахай, привычный указывать греческому
суду законы п давать указания, добавил от себя:

— Брить отменено.

Этого Ильяшенко не мог простить и он сердито разнес стряпчего, 
приведя его в состояние настоящего смятения.

— Мальчишка! — кричал Ильяшенко, — ничего не попимает; отменено брить содержащихся, 
а уголовныхь преступников, ссылаемых по конфирмации куда следует,
должно брить по прежнему.

— Подожди, — переводя дыхание добавил строгий ревизор, — доберусь и до тебя потом и тебе не миновать Сибири!

Последине слова как-то особенно глубоко запали в
душу стряпчего; он подумал: ``Кто не без греха... Неровен час... всяк человек 
в воле начальства''.

Шансов на сопротивление у Логафетова не оставалось
никаких, ибо капитан Лисенко был в той фазе решимости дейстровать, 
когда не оставалось таких препятствий,
которых бы он пе еокрушил ради исполнения приказа начальства. Он сам впоследствии передавал 
своему знакомому, нынешнему секретарю мариупольского съезда, г. Б — му,
что в тот торжественный момент он так уверовал в
Ильяшенко, что мурашки у него бегали по спине и волос
с затылка поднимался. \emph{Если бы Ильяшенко}, — передавал
всегда восторженно и с экстазом капитан, — \emph{в тот момент приказал ``коли'' — всех бы переколол, еели бы
скомапдовал ``пли'' — всех перестрелял бы.}

По прошествии многих годов капитан всегда приходил в экстаз при воспоминании 
о знаменательном моменте; предоставляю всякому судить, какой силы был подъем духа 
капитана в самый момент действия.

По кивку головы Ильяшенко капитан Лисенко и два
создата приблизились к Логафетову и, не встретив ни
%page33
малейшего с его стороны сопротивления, посадили его, за
отсутствием барабана на сундук; найденным в присутствии полотенцем
ему связали назад руки. После этого
ноги его одели в кандалы. Затем к нему подступил
дрожащий цирульник; в правой руке он держал наготове 
ту единственную, изъязвленную зазубринами бритву,
которая числилась в инвентаре, находящемся при команде.
Старожил Мариуполя, передававший мне порядок бритья
половины головы, пояснял, что цирульник, после беготни
и со страха, исполнил бритье неправильно. 
``Сами подумайте'', - говорил разсказчик, -  ``начал брить с затылка
против шерсти. Ну, конечно, задрал кожу; он хотя и
примазал, а Логафетову все таки было неприятно, и кровь
текла как из резаного''. Эта мелочь ни на минуту не
остановила обряда и бритье было закончено безпрепятственно;
после чего на голову осужденного солдата натянули арестантскую шапку; 
в заключение бывшему заседателю развязали руки и одели его в арестантский халат. Логафетов,
считавшийся опасным человеком, которого всяк еще так
недавно побаивался, в виду его крутого нрава, вдруг сделался
кроток, как ягненок и старательно исполнял все,
что от него требовали. В произведенном алекс—им исправником 
следствии этот учиненный над Логафетовым
обряд назван ``делом о невинном истязании мещанина
Николая Логафетова в присутствии мариупольского греческого
суда'' (л. д. 158, там же). Относится ли слово невинный
к истязанию или Логафетову остается не выясненным. Но
когда происходили указанные выше действия над Логафетовым, 
никто не думал об истязании; все знали только,
что приговор приводится в исполнение на законном основании. 
Вот почему капитан Лисенко не замедлиль поставить Логафетова между солдат, из коих два стояли
сбоку, один спереди и один сзади. 
%page34
Сам же капитан, получив от Ильяшенко приказ отвести ``преступника'' в
тюрьму, стал впереди и, держа саблю наголо, приказал
отряду следовать за собой. Вскоре собравшийся на улице
народ с любопытством смотрел на шествие отряда, в
котором среди солдат, путаясь в кандалах, шествовал
бывший начальник, еще так недавно наводивший страх
на местных обывателей. Все смотрели на удалявшегося Логафетова со страхом,
некоторые с грустью; более же всех
уныло смотрел на кортеж стряпчий Хартахай, из смущенной души которого 
под тяжелым впечатлением недавнего
разноса вырвался следующий меланхоличесвй возглас: \emph{``все
мы пропали, скоро все пойдем туда-же''}...

События этого дня в Мариуполе, как видит читатель, текли с поразительной
быстротой; около 10 часов утра Ильяшенко впервые открылся писарю команды в
канцелярии начальника команды; к 12-ти часам дня произнесен над Логафетовым
приговор и моментально утвержден, и приведен в исполнение. В 12 часов дня
капитан Лисенко ведет осужденного окруженного четырьмя солдатами в тюрьму.
Несмотря на такую стремительность событий, толпа успела осведомиться о
чрезвычайном событии и, собравшись в большом количестве, длинным хвостом
замыкала шествие небольшого отряда, предводимого капитаном Лисенко.  Коренные
обыватели Мариуполя, хотя и хорошо были осведомлены о служебных подвигах
Логафетова, тем не менее не только не говорили: \emph{``по делом вору и
мука''}, но видимо были смущены и сконфужены. Надо помнить, что с самого
момента своего переселения из Крыма в Мариуполь, греки находились в
исключительных условиях и пользовались такими правами, каких до 60-х годов не
имело население Российской Империи, за исключением дворянства.
%page35

Не говоря о специальных привилегиях, заметим только, 
что греки пользовались правом избирать из своей
среды состав греческого суда, являвшийся учреждением,
которое полностью совмещало судебную, 
полицейскую и хозяйственно-административную функцию во всех делах, касавшихся
греков. Это учреждение было излюбленным детищем
греческого общества и в свою очередъ являлось доброй
матерью для всякого грека и самой злой мачехой для всякого
постороннего. Греки любили, ценили и гордились своим
греческимъ судом и иначе не называли его как \emph{наш}
судъ, делая ударение на слово \emph{наш}. Понятно поэтому, что
быстрая расправа над бывшим судьей греческого суда,
являлась достаточно ощутимым афронтом для всего греческого населения.  За то
новые поселенцы Мариуполя, не греки, терпевшие притеснения от греков, ликовали
столь шумно и неудержимо, как это требуется при проезде начальства или при
установленных приказанием начальства праздневствах. Неизвестно почему кто-то в
этот момент встретившихся ликований и смущений пустил слух, что явившийся
ревизор ни кто иной, как сам Великий Князь Михаил Николаевич. Слух моментально
распространился и ему так поверили, что всякий, кто осмелился бы заявить
сомнение, наверное, если бы не быль избит, был обруган дураком или еще хуже
приписан к бунтовщикам, к неблагонамеренным, опасным людям, отрицающим власть.
Итак, одни были сконфужены, другие ликовали, а Логафетов, со скверно обритой
головой, в арестантском халате, улрученный, сидел в тюрьме.

Председатель Попов, согласно доброй, старой традиции
и поныне к счастью не умирающей, рассудил, что как
бы важны нн были государственные дела, тем не менее
они не умаляютъ существенной важпости обеда. Поэтому,
%page36

после того как Лисенко отрапортовал, что престулник
завлючен в тюрьму, Нонов обратился к высокому ревизору с почтительной 
просьбой сделать честь откушать хлеб-соль. Предложение было принято. 
Тогда Попов пригласил
также обедать весь состав суда, секретаря и капитана
Лисенко, которые весьма рады были и поесть, и побыть в
обществе высокопоставленного лица. Обед состоял из местных греческих явств, и, главным образом, отличался
обилием разного материала, подлежащего еде; желающие
могли, не обирая соседей, есть до достижения отвращения к
пище и все таки не одолеть всего обеда. Шампанское в
то время в Мариуполе не водилось; заздравных тостов
произносить не умели и поэтому ревизор не был осыпан
речами с исчислением его добродетелей. Правда, за водкой,
стуча рюмкой о рюмку, участники обеда говорили: \emph{``за ваше
здоровье! по чаще! по больше!''}... Но дальше этого красноречие не шло.
После нескольких рюмок, когда настроение людей становится более откровенным, Попов обратился к
высокому гостю с просьбой ``по душе'' сказать; верит ли
он полной виновности Логафетова, на что Ильяшенко ответил, что в виновности этого последнего
не может быть ни малейшего сомнения. Также не отказался Ильяшенко ответить некоторым из 
присутствующих, интересовавшихся узнать, каким образом он получил свои чрезвычайные
полномочия от Государя. Не стесняясь, Ильяшенко рассказал 
какую-то фантастическую историю об его участии в
охране священной особы Государя Императора, после чего
Государь его лично узнал и осчастливил особым вниманием и доверием.
Когда же затем в достаточно обнаженной форме высказывалось, что настоящее дело такого рода,
что 16.000 рублей ассигнациями такая сумма, которая не
велика, чтобы вычеркнуть все дело Логафетова и уничтожить
%page37

ве его слФды, то Ильяшенко прямо отвфтиль: „и радъ бы,
но не могу, ибо аично уполномочень Госуларемь Имнера-
торомъ!* Оббдь не затянулся; черезь какой нибудь чась
времени веф чувствовали себя въ такомъ васыщенномъ ©9-
стояши, что продолжать ду было бы для нихъ мучешемъ,
Ревизоръ выразилъ желан!е отправиться къ себ на квартиру,
при этомъ запротилъ кому либо его сопровождать. Хозяияъ
дома приказалъ заложить лошадь въ дроги — единственцый
въ то время извзетный обывателямь Мар!уполя экипажь.
Пока закладывали лошадь, Поповъ и его сослуживцы про-
сили Ильяшенко объяепить, гдф онъ остаповилея; они ечи-
тали себя обязанными явиться въ ревизору на домъ ш сотротс,
такъ сказать, еще разъ представиться и откланятьея. Но
Ильяшенко категорически отклонилъ всяк!е тавше проекты.
не указаль своей квартиры и освободиль марГупольсвя влаети
отъ явки для представлешя.

Разставаясь съ хозяиномъ и его гостями Ильяшенко
отоввалъь въ сторону капитана Лисенко и вручиль ему тремй
и послёдн приказъ, написавный также, какъ мы знаемъ,
рукою писаря. Этоть приказъ былъ болфе кратокъ; тексть
его сл8дующ!й: „Уголовный преступникь Николай Логафе-
зовъ по корфирмащи подлежитъ ссылкф въ Алтайсве заводы
въ вЪчные работники и долженъ сл®довать чрезъ Екатери-
нославскую губернию, куда немедленно его отправить за ввЪ-
реннымъ Вамъ Его Величества карауломъ до города Бер-
лянска, и оттуда черезъ г. Алексапдровевъ въ г. Екатери-
нославъ, гдф мфстное начальство приметъь свои м$фры для
лальнзйшаго отправлешя. По данной мнф власти 18 мая
1862 тола въ Царекомъ Сел. симъ повелфваю полпомочеп-
ный Государя моего и вфрноподданный Григорй Власовъ
Ильяшенко. Г. Маруполь, 6 апрфля 18683 года, зъ чаеъ
пополудни“. ,

Несмотря на выраженное ревизоромъ желан1, чтобы
никто пе утруждаль себя панесетемъ прощальнато визита,
Поповъ, тёмъ не менЪе, солидно разсуждая, что и излининее
усерме къ начальству только приносить’ пользу, тайно при-
казалъ своему кучеру замфтить домъ, у котораго остановился
начальникъ. Когда Ильяшенко разеталея съ Поповымъ, осталь-
пая компан, посовфтовавшись съ хозяиномт, рЬшила, что
общая явка къ внезапно прибывшему высокопоставленному
лицу безусловно необходима. ПорЪшили, что должны авиться
купно всЪ власти, имя во главф военную власть, капитана,
Лисенко; затЪмъ находили, что было бы весьма великолвоно,
если бы шестые властей замыгали главнЪйпие именитые
граждане г. Маруноля. Оповфетить послвднихъ обязались
члены суда, которые и не замедлили разбЪжаться по разнымъ
нонцамь города. Итакь было условлено, что часа черезъ два,
т. а. между тремя н четырьмя часами дня, ве$ соберутся
шь Попову в озтуда совмбстно ‘направятея во временную
квартиру г. ревизора. До тфхъ поръ каждому предоставля-
лось идти домой для приведеня себя въ достодолжный по-
рядокъ. Стряций Хартахай, пользуясь свободным временемъ,
занель въ своему знакомому, Д—чу, сообщить © чрезвы-
чайномь происшестьи дня. Д—чъ быль изубетенъь всему
Мархнолю, какъ человзкъ образованный, развитой; еверхъ
того совершенно справедливо его считали весьма свфдущамъ
въ законахь. На служб онъ не состояль, проживаль въ
Мар!уполь частнымъ человфкомъ, владфя невдалек® отъ этого
города землей. Люди съ злымъ язывомъ увБрили, что Д—чъ
быль единственный умный человзкъ во всемь городф. ВЪро-
ятно, къ тавому завфренцо нужно внесли поправку и сказать,
что Д-чь быль самый умный среди окружающихь его обы-
вателей г. Маруполя:

о

Угнетенный собычями дня, стряпый Хартахай, вновь
восприняль смущающЕ духъ ударъ, когда Д-_чь на его
слова замфтиль: се шло таке, що не Логафетова, а васъ
зсихъ за вашъ судъ погонять на алтайске заводы“. Д—зъ
любиль уснащать свою рЁчь малоросейскими словами и фра-
зами и его слова, какъ по опредфленности содержашя, такъ
п по е106обу вхъ выражения, всегда производили на слума-
теля сильное впечатлф ве. Перепуганный стряпчй заметался
отъ этой новой точки зрнтя на дфло. Что же лЪлать? — по-
просиль онъ”совфта. Сейчась телеграфировать губернатору, —
опредфленно рёшиль Д— 95.

Этотъ совВть прояениль мысли Хартахая; онъ сразу
сообразить, что телеграфировать по службЪ начальствомъ
разр®шастея и что телеграмма не вызоветь гифва ревизора,
разъ онъ составить телеграмму въ емыслв доношен!я его пре-
воеходительству о прибыти чрезвычайнато уполномоченнаго.
ь Вь такомъ вил онъ и послаль телеграмму. Изъ изло-
женнаго сейчась ясно, почему я выше замфтиль, что про-
тестующан приписка Хартахан на приговор$ суда была едф-
лапа не въ моментъ подиисаня приговора, а позже. Въ тоть
момептъ страпуй не посмфлъ бы противорфчить грозвому
ревизору. Такое мое заключене подтверждается еще слфду-
ющимъ мфстомъ изь позднёйней жалобы Логафетова, кото-
рый, по конедцъ своей жизни остался недоволенъ дЪйетвую-
щими лицами греческаго суда. Логафетовь писалъ: „Харта-
хай телеграфироваль въ ЕкатеринославЪ, не только по. с0-
вершени надо мною истязавя, но едва ли не посл откры-
1 шарлатанства Ильятенко, въ чемъ можно удостовфриться
справками па телеграфЪ“. Сверхъ того, Логафетовь убазы-
валъ. что Хартахай въ телеграмм только доводить ло свЪ-
дня начальства о появлеши уполномоченнаго“. (тамъ же
л. д. 41). | а

— 40 —

Воть почему, говоря словами изь жалобы Логафетова,
„я вевольно влекусь къ подозрн!ю“, что приписка Хартахая
на приговор греческаго суда, сейчаеъ поелФ подписи: „не-
согласенъ на исполнене и буду телеграфировать г. началь-
нику губернии“, — сдфлана впослЪдетви, посл совЪщаня съ
Д-—“чемъ или, вФрнфе, посл полученя отвтпой телеграммы
губернатора, которая была доставлена какъ разъ въ то время,
когда всё собрались у Попова, чтобы купно пиествовать от-
кланиваться въ отьёзжалощей персон%.

Необходимо пояснить, что вт то время телеграммы хо-
дили не такъ, кавъ теперь. Тогда люди меньше чмъ теперь
пользовались этимъ способомъ сношеня другъ съ пругомъ,
поэтому, при соотвтетви числа телеграфнымъ аппаратовъ съ
количествомъ посылаймыху, телеграмм, послфдн!я не зале-
живались на телеграфныхь станщяхъ и передаточныхъ пун-
ктахъ въ роли вандидатокъ, долго ожилающихъ своей очереди.
Словом+®, практика показываеть, что тенерь телеграммы хо-
дятъ у васъ, на югф Росйи, со скоростью оть 8 до 10
вереть въ част и не измфняють этой, практикой установив-
шейся скорости, даже въ экстренныхь случанхь: тажъ, на-
примЗръ, во время недавнихъ безпорядковъ на мар!упольскихъ
закодахъ, телеграфное сообщене торжественно сохраняло
принципь равнопразя и телеграммы начальства аккуратно
прибывали, послЪ прУВзда лицъ, извЪщавшихъ о своемъ при-
быи и дёлавтихъ свои предварительныя распораженя. Но
въ описываемое время такихъ прогрессивныхь новтествъ не
было; телеграммы начальнику губерниу ть него передава-
лись почтительно и безъ замедленй. Конечно, намъ съ на-
шимъ настоящимь каждодпевнымъ опытомъ трудно повфрить,
чтобы существовало такое благодатное время, когда на те-
леграмму, посланную изъ Машуполя. въ Екатеринославъ по-
лучалея бы отвфтъ черезъ часъ или два часа; теперь въ по-

добномъ случаВ, мы должны ожидать отвфта два или три
дня. Тмь не мене. каковы бы ни были наши представлен1я,
факть остается фактомъ, и въ описываемое время, получить
въ МарутолЪ отвфтную телеграмму черезь часъ или два,
считалось нормальным. Чоэтому пусть читатель не удивля-
ется тому обстоятельству, что предуорежденный Хартахаемъ
начальник телеграфной конторы, принесъь отвфтную теле-
грамм губернатора въ домъ Попова, между тремя и четырьмя
часами пополудни, когда веБ власти и именитые граждане
были въ сберф для шеетыя къ высокому начальнику съ

представлен емъ.

Собравшееезя общество оказалось въ превеликомъ за-
трулнени, велЗдетые того, что кучеръ Попова не узналъ, гдЪ
квартира начальника. ревизора. Ве по этому поводу охали,
торячились, спорили, и едипоглаено критиковали несообрази-
тельность кучера. Но послфдый не быль виновать; онъ пе
мотъ узнать, гф квартира начальнива, благодаря хитроет-
нымъ дфйствямъ` этого поелВдняго. ВыЪзхавъ за городъ, на
гору, ЕЪ тому м$ету, тд теперь тюрьма, Ильяшенко оста-
новилъ кучера, даль ему на воду и ириказаль повернуть
и убхаль назалъ въ городъ. Ослуматься кучеръ не поемлъ
и шагомъ поёхаль обратно, разечитывая оглянуться и зам-
тить, хотя направлен1е, куда направить свои стопы началь-
викъ. Но этотъь послЪдвй, вЪроятно, быетро прилегь въ
первой попавшейся рытвинЪ, потому, что кучеръ, оглянув-

шись никото не видЪлъ и только про себя замтиль: „канулъ,

_кавъ въ воду“. ВелЪдетые такого оборота вещей, все обще-

ство, собравшееся у Попова, не знало что дфлать и какими

путями разыскать начальство. Изъ затрудненя вефхъ вывела,
|.

телеграмма. векрытан Хартахаемъ; въ ней значилось: „Ири-

казан й уполномоченнахо не исполвять, немедленно его арес- -

товать“. СлФдовала подпись губернатора. „Д—чъ угадаль!“ —
перепуганнымъ голосомъ возопилъ стряпЙ Хартахай, — „мы
веЪ пойдемъ на каторгу“... У капитана Лисенко первый
разъ въ жизни подкосились ноги и онъ трузно опустилея
на стулъ. Остальные разомъ заохали, зачмокали, застонали.
Хартахай раньше другихъ посгигь произведенную телеграм-
мой губернатора повую шие еп зсепе и у него сейчасъ
же проснулось приеущее стряпчему, прокурорскому оку,
усерме найти автора преступлоены п повергнуть. онаго подъ
варощую длань неумолимаго закона. „Господа“, — первый
затовориль онъ,— „нельзя упустить преступника!“ — „Да ка-
вой онъ преступникъ, онъ невинно сидитъ въ тюрьмф“, —
послышалось въ отвЪтъ нЪеколько голоеовъ. Ве$ такъ освои-
лись съ мыслью осужденя Лотафетова, что почитали слова
стряпчаго относящимлея къ Логафетову. Хартахай взбфсился
оть этой неповоротливости сообразительности и на кончик»
языка виефлъ у него отьфть: „барашы“! Но лица, къ коимъ
должно было приложиться @1е пазваве, уже попали свое
Ча рго Чао, и стали совмфетно и громко говорить что-то
въ свое оправдане, причемъ веЪ усердно ув$ряли другъ друга,
чтб они съ перваго момента, какъ увидфли Ильяшенко, бы-
ли увфрены, что онъ шарлазанъ. Почему рапьше никто не
открылъь такой своей увзренности, объ этомъ никто не ска-
заль пи одпого слова. Веб тЪмъ ве мене говорили много
съ явной тенденшей превознесть собственную свою прони-
цательность. Въ общемъ поднялся неимовзрный гвалтъ: вов
представители греческаго общества жестикулировали почти
всБми органами своего тфла, хватали другъ друга руками
за ворогъ сюртуковь, тывали пальцами другъ другу въ лицо
и ве6 крачали, какъ будто соперничали между собою голо-

сами, подобно опернымъ пЪвцамъ. Бздный Хартахай надор-

валъ свои голосовыя связки, прежде чЪмъ ему удалось воз-
становить кой какую тищину. Ме теряя ни минуты, онъ
сейчасъ же выработаль плапъ поимки преступника. Рф шено
было собравиихея у Попова именитыхь торода Маруполя
гражданъ направить за, поисками лжеуполномоченнаго, (такъ
съ этого момента Хартахай называлъ Ильяшенко), въ пред-
мфетье города, на Марьинскую сторону, куда всего удобнфе
было укрыться съ того мЪфста, гдВ Ильяшенко исчезъ съ
глазъ кучера Попова. Засфдатели греческаго суда должны
были набракь сторожей разныхь присутетвенныхь мфетъ,
вликнуть волонтеровь и съ этими соединенными силами пус-
титься въ обходъ всего города. Наконець капитанъ Лисенко
вомандировалъь 3 создать и одного унтеръ-офипера, тоже па
Марьинекую сторону въ подкрфнлеше къ именитымъ граж-
дапамъ. Самъ же председатель Поповъ, стряцчй Хартахай
и гапитань Лисенко должны были оставатьея недвижимо въ
квартир Попова и ждать извфемй. Ихъ пазпачене состозло
въ томъ, чтобы по открыти м$етонахожденя лжеуцолномо-
ченнаго, немедленно отправиться къ этому послзднему и
объявить его арестованнымт. ]

Цоиски имепитыхь гражданъ и соединеннаго отрада,
руководимаго засфдателями были безуспВшны; на воз ихъ
разсироем имъ отвьфчали, что никакой назальвикь нигдф не
проявлялея. Они ошибочно искали чрезвычайныхь зваменй
чрезвычайнато уполномоченнаго въ видф развфвающатося флага
надь домомъ, занимаемомъ этимъ послфднимъ, или въ вид.
полосатой будки съ заключеннымъ въ ней полицейскимъ и
т. п. Но такихь знамен! не дано было имъ открыть. Име-
нитые граждане ходили толпой, горячо разговаривали, спо-
рили и тоже пичего знаменательнаго не открыли. Носланные
же капитаномъ солдатики замЪтили возлВ одной изъ хать
на Марьипской сторонЪ ту самую одноконную повозку, на
м фи рее

которой Ильяшенко и его товарищи наканунф въЪхали въ
г. Мар!уполь. Солдатики уклонились оть сложныхь разсу-
‘жденй и непосредственио рфшили, что эта повозка никого
иного, канъ только начальника, вЪротно въ пользу этого
быстраго заключен!я говорило то обетоятельетво, что мфетное
павелеше не пользовалось одноконными телфгами, а упот-
ребляло двухконныя, похозйя на современные фургоны. Одинъ
солдатикъ, подкравшись, заглянуть въ окно и увидёль че-
ловЪка, растянувшагося на лавьВ; кавъ заглянув въ окно,
такъ и его товарищи опять непосредственно уфшили, что
никому иному не растягиваться на завьВ, какъ только на-
чальнику и побЪжали отрапортовать о своей находеЁ капи-
тану Лисенко.

Впередь каюсь; на одну минуту я измёню своей роли
лВтониеда-протоколиста и не воздержусь отъ упрека, который
мн хочется бросить Ильяшенко. Мн хочется ему сказать:
„Ильяшенко, Ильяшенко, почто ты, ‘нагрузившись достаточно
за обфдомъ у Попова, напилея окончательно пьянъ, придя
домой. Зачфмъ ты, спровадивъ кучера Попова, не запрегъ
своей лошади и вмВетф съ твоими товарищами ве пере$халь
зерезъ рфку Кальзусъь на Донскую сторону. Для этого надо
было потерять менфе полчаса времени и черезь какихъ ни-
будь полчаса никто никогда тебя не отыскаль бы, и никто
никогда че узпалъ, откуда, еъ пеба или съ земли, появилея
этоть ошеломляюний, властный уполномоченный. Какое бы

осталось широкое поле любителямъ мистики излатать теорю’

о навожденши, любителямъ точнаго зная—6 границахъ са-
мообмана чувствъ, коллективнаго внушен!я; а людамъ бла-
товамфреняниъ, трепеть носящимъ въ душ, представился
бы случай поговорить на тему, что никто не знаетъ ни дня,

9

ни часа, ниже образа и вида, въ которомъ можеть поя-
витьея высшее начальство. О, Ильяшенко, зачфиъ ты напился
пьянь! ".

Прелетавители разнородныхъь властей: председатель Но-
повъ, стряшй Хартахай и капитанъ Лисевко стояли въ
нерзшительности передъ дверьми того дома, въ которомъ
солдалики выедёдили предполагаемаго начальника. Никто
не рЬшалея»войти. Предевдатель говориль въ томъ емыслЪ,
что безстрапе военныхь людей обязываеть вапитана пока-
зать прим®ръ доблести. Вапитанъ ничего не говорилъ, но
рёшительно и отрицательно качалъь головой. Старожилы ут-
верждають, что капитанъ, жестоко перепугавугись посл

получен1я телеграммы губернатора, впаль в% /йервное состо-

яне, что сталъ всего боятьен и даже „самЪ себя боялся“.
Хартахай намокнулъ, что Поповь первое лицо въ городЪ,
но на это, въ видф возражешя, получить отьфтъ, что стряп-
И — око закона. Наконецъ посл долтихъ колебави вез три
представителя власти рзшили войти вмветв. Набравшиеь духу,
они влстфли въ землянку, гдВ почиваль опьянзвиий Ильи-
тенко. Онъ проснулся, протеръ глаза и, услышавъ что онъ
арестованъ, тавимъ громовымъ голосомъ разнесъь вошедшихъ,
грозя наторгой и вефми ужасами Сибири, что вошедше не
выдержали отпора и б%жали изъ хаты, вновь пораженные
тревожною мыслью: & вдругъь передъ ними персона. Такъ
разносить. кричать, ругать можеть только начальство — въ
этомъ вефк трое были тлубоко убфждены. Но поглощенная
Ильяшенко водка опать сму повредила. Пока Ильяшенко
разносиль, явивнЧяся къ нему власти задыхались не только
оть волненшя, но и отъ отвратительнаго запаха сивухи, ко-
торая исходила оть Ильянтенко, какъ изъ разливитейся бочки,
О

Благодаря этому Хартахай и смогъ сообразить, что насто-
ящй вачальникь не будегь преисполнаться исключительно
столь омерзительнымв питьемъ, кавъ простая сивуха: но
разсужденио Хартахая, вастоящая персона можеть только
для начала отвфдать простой водки, — для отврытя тавъ
сказать, выпивки. а излишеству предается болфе блатород-
ными пимями. 0ъ отимъ справедливымь разсужденемь стряп-
чаго его спутпики вполнЪ сотласились. ВеЪ трое опять
вошли. Хартахай р$шительно заявилъ, что Ильяшенко име-
помъ закона арестовань и что, въ случав сопрстивленя, ва-
питанъ пустить въ дЪло стоявшихъ у дверей солдатъ. Серьез-
ность положен отрезвила Ильяшенка; онъ понядь, что
доведен1е дзла до рукопашной было-бы фатально для его
престижа-и онъ избралъ другой епособъ соиротивленя. „Хо-
рошо“, отвфтиль онъ ворвавшимся въ нему влаетямъ, „я
пойду за вами, только знайте. что я вамъ покажу, вто я;
весь вЪкъ будете плакатьея!“ Ильященко. дЪйствительно, но-
слфдоваль за Хартахаемъ, Поповымъ и Лисенко.

Было болфе 5 часовь пополудни; еклонявтееся къ за-
паду, яркое, весеннее солнце выглянуло изъ за разорвав-
шихея тучъ и мягкимъ, радующимь взоръ свфтомъь залило
невзрачные МарТупольсве домики. грязную немощеную улвцу,
Ильяшенко, Попова, Хартахая, капитана Лисенко и большую
сопровождавшую ихъ толну людей. Вс эти люди направ-
лялись въ присутетые греческаго суда, кула они скоро
и прибыли безъ всякихь инцидентовъ. Отсюда, цервымъ
дЪломъ, предефдатель суда Шоповь отправилъь въ тюрьму
приказъ, который я копирую съ сохраненемъ не только вы-
раженй, но и ороографя. Вотъ этотъ приказъ: „СОмотрителю
Тюремпаго замка. Сейчасъ освободить изъ тюрьмы и ири-
слать въ судь содержащагосе Николая Логафетова. АпрЪль
6 дня 1863 года. ПрелеБдатель К. Поповъ“.

-- АТ —

Согласно этого приказа Логафетовь былъ доставленъ
въ судъ въ томъ самомъ видЪ, въ какомъ утромъ его пре-
проводили въ тюрьму, т. е, въ кандалахъ и арестантекомъ
одфяни.

Логафетовъь продолжаль пребывать въ состоянти угнете-
ня, доведшаго его ло полной апатш; онь молчаль, не вы-
фажаль радости но поводу возвращенной ему свободы, тупо
й медленно озиралея, и только по временамь тяжело со
стономъ вздыхалъ.

Капиталь Лисенко былъ ни живъ, ни мертвъ, иресл$-
дуемый слЗдующими безъисходными лумами: „если Ильяшенко
шафрлатапьъ, капитану не избфжать суда за содфаянное надъ
„Лотафетовымъ; если же Ильятенко таинственный и чрезвы-
чайный начальникъ, какъ въ этомъ онъ продолжаль упорно
увфрать, постоянно твердя: „попомните меня, никото не
забуду“... то совефмъ не трудно угодить въ каторгу, еели
не на зисфлицу“... .

Хартахай, предефхалель, засвдалель и ваводнивиие при-
сутстые суда именитые граждане пылали зл0б0й противъ
Ильяшенко, обиженные посмфян1емъ надъ ихъ роднымъ учреж-
дентемъ и наль вовми ими.

Снявъ съ Лотафетова арестантекое одзян1е, эти разсер-
женные люди со злобой одЪли въ него лжеуполномоченнаго.

Но заключить Ильяшенко въ кандалы греки не поем$ли;
они все-таки по инерци продолжали чувствовать совершенно
ни па чемъ опредЗленномъ не основанный страхъ; вЗроятно,
проето на ихъ впечатлительность дйствовало то обстоятель-

ство, что Ильятенко, не падая духомъ. продолжаль ихъ

разносить и стращать всякими ужасами; а также вЪфроятно
имъ импонировала сохраненная арестованнымъ манера дер-
жать себя съ неукоризнениой величественностью настоящей
переоны.
" — 48 —

Но такъ какъ надо было что нибудь дЪлать еъ опас-
нымь узникомъ, то въ концф концовъ по наставленйо Хар-
тахая, судъ въ полномъ состав и присоедивившаяея толна
именитыхь гражданъ, всБ вм$етБ, повели Ильяшенко въ
тюрьму. Половину нурешестве совертили безпрепятственно.

Только со стороны русекихь поселенцевь, находизтихея
въ толп, было проявлено н®что, похожее на мапифестатю
въ пользу Ильяшенко. Эти руссше люди отерыто увфрали,
что греки влекутъ въ тюрьму Великаго Ёнязя. Конечно.

если-бы тавихъ протестантовъ ‘находилось н\аволько соть

челов къ, шествию не сдобровать; оно было-бы силою оста-
новлепо и узникъ получить бы свободу, но на несчастье
Ильятенко тавихъ русскихь людей не было и десятка; ихъ
протесть не могъь имЪть реальнаго значен!я; на нихъ никто
ке обращаль вниманая. Шестве же, когпа половина пути
была пройдена, простановилось благодаря р8шительному
ДЪИствию одного изъ именитых гражданъ, содёйстворавшаго
властямъ и греческому суду въ арестовавши лжеуполномо-
ченнаго. Подобно доброму коню, получившему шпоры, онъ
стремительно выскочилъ изъ толпы, сталь на ея дорогф и,
растопыривъ руки, во всю глотку заоралъ:

— Стой, стой, стой!...— произнося это слово то по русски,
10 по турецки. Задыхаясь, онъ сталь затВмъ говорить скоро,
крича, размахивая руками и часто отилевываясь. Омысль его
обильныхь сизшно сыпавшихея словъ сводился къ тому, что
всв греви тотовятъ себф Сибирь, каторгу, что веф оли „ду-
раки и бараны“, не могли понять, что они сажаютъ вь
тюрьму не только Ильяшенко, но и царсюй портретъ, ви-
сящий на шев арестованнаго.

Мноче изъ слушавшихь воскликнули, мноме ударяди
себя собственною дланью по лбу, всЪ остановиливь на ми-
нуту вавь вкопанные, и затВмъ все шесте повернуло об-

=
——

49

фалтно въ гречесый судъь Здфеь люди, погаалвь еъ четверть
часа, поспоривъ и въ пфоколькихь отлльныхь елучаяхь
обругавъ другь друга, еняли сь шей Ильшиевко медальонт
<ь портретомъ Государя и оаять повели узника въ тюрьму,
вуда на этотъь разь доставили его благополучно и безпре-
пятственио.:

Так» пончились вачальственвыя похожлены Ильншенко
въ г. Мар!уполв. Какф видить читатель, ови ограничились
однимъ днемь: въ 10 час. утра Ильяшенко объявилея въ
канцелар!н уачальника команды, вь 12 чае. дня свершихся
судъ п исполнене его рфщеня надъ Логафетовыме, между
часомь и двумя состозлея обфдь на манеръь банкета, а къ

5 часам» пополудни Ильяшенко уже былл, ареетовант,

Эпилогь моръ-бы составить новую исторю, открывалюо-
щую одву изъ обычныхь страничеюь дореформеннаго буди,
отживавшатго въ то время свои послвдые дни. Порадвя эти
веБмь хороно извзетны и мы будемъ кратки

Логафетовъ, обрВиь свободу и возвратившись домой,
долгое время оставалел въ раздумьи: обрить-ли ему вторую
половину головы или ждать, пока бритая половина отростеть.
Ньсколько м$фелцевъ ходилъ онъ съ повязаннымъ на голов
плалкомъ. Онъ випль злобой противъ всего состава гре-
ческаго суда — бывшихъ приятелей, и, пользуаеь услутами
какого-то трамотВя, писалъь безкопечныя жалобы на вс
мЪетныя власти и особенно на предс®дателя Нопова и стряи-
чаго Хартахая, усматривая въ дЪйстьяхь этихъ послфднихь
лиць злыя, корыстныя намфрена прохивъ своей личности:
Зе эти жалобы; несмотря на многослоне, заключали! толь-
ко одну улику’ противь обвиняемыхъ „Логафетовымь динъ,

состоявшую въ томъ, что Ильяшечйо быль въ. проотомт,
|

Е т

старомъ пальто и весь наружвый видь его ни мало не со-
отвЗтетвоваль настоящему начальнику

Такая шаткая улика оказалась, понятно, недостаточной
и жалобы Лотафетова по бездоказательности, оставлены безъ
посл детвий.

Но хуже всего для жалобщика было то обстоятельство, что,
поразившее его въ моменть приведеня надъ ним приговора
въ исполнене, угнетенное состояне духа ме только не 0с-
лаблялось, но съ каждымъ двемъ усиливалоеь. Овъ осунулея,
избфгаль людей, потерялъь апиетитъ, словом быстро шелъ
на убыль. Протявувъ въ такомъ видф года ©ъ полтора, Ло-
гафетовт' умеръ.

"Тазъ относительно него, еще въ земныхь условнхъ,

исполнилел заковъ вовдаян1я зломъ за зло.

Капитанъ Лисенко, Хартахай, Поповъ и члены гречес-
каго суда около двухь лВтЪ пребывали въ нензреченномъ
страх. Кром Лиеенко, вс остальныя лица, ища спасены
отъ отвфаственности за содфянное, обнаружили не особенно
высокую культуру душевныхъ свойствъ. Поповъ, Хартахай
и вообще весь составъ греческаго суда прилагази всё уси-
эн, чтобы вызвать полозрн!е, будто капитавъ Лисенко на-
ходился въ предварительномъ заговор съ Илзъятенко.

Ол$ды такихь злокозненныхь дзйстый сохранились въ
н%которыхь докумевтахъ.

Такъ, напр., въ журналв „входящим и исходящимъ
мартупольской команды па 1865 ‘тодъ“ есть ‘указав!е, что
уже 229 апрфля мазупольскй гречесый судь отнотешемъ
за № 1598 новарно просиль валитана Лисенко увдомить,
на какомь основани начальникь команды прибыль съ людь-
ми въ сей судъ 6 азирфаля для арестованя мЫщанина Ло-
тафетова”.
и

а

Капитанъ Лисенко не кривилъь дущою и отношенемъ
отъ 95 апр$ля, за № 207, напрямикь отвЗчаль, „что при-
быме начальника команды ©ъ конвойными людьми въ нри-
сутете сего суда 6 аирЗля состоялось по полученному
приказу, подписанному именующимь себя полномочнымъ
Государя Императора — Ильяшенко“.

Затьмъ, въ судебномъ дл есть особый слфдетвенный
акть отъ 14 сентября 1863 года, возникпий „по поволу
неодновратныхь есылокъ предс®дателя Попога и Хартахая
на 10 обстоятельство, 910 нев\роятно, чтобы лжеупономо-
ченный Ильяшенко не имзль свидамя съ капитаномь Ли-
сенко ло прибымтя въ гречесый судъ“.

Но всЪ эги хитрые маневры, ‘долженствовавийе напра-
вить слфделые въ желательному для Попова и Хартахая
направлени не увЪфнчались успфхомъ: слфдетые воочию до-
кавало, что вапитанъ Лисенво все время дЪИствоваль, какъ
бравый солдалъ, по совфети, Бопа Нае.

Добролвтель не пострадала и капитанъ Лисенко пе
подвергся никакому преслёлованю. Года черезь три онъ
совсфыт оевободилея отъ страха отвфтетвенности за содвян-
пос падъ Логафетовымъ. Когда ему приходилось разсказывать
о знаменательномъ днф 6 апрфля, онъ всегда и неизмВнно
внадалъ въ энтумазмъ. Н%сволько лфтъ тому назадь, въ
тлубокой старости, вацитанъ скончался. |

Немного сложн%е и длиннЪе должно быть повфетвова-
ве о дальнфйшей судьбв Ильяшенко,

Въ то время, кашъ вс участники настоящей эпоцен
иребывали въ страх% и трепет, боясь отв®тетвенности, одинъ
Ильяшенко, котораго судьба висфла на волоскВ въ течене
почти трехъ лёть, не падаль лухомъ. Заключенный вЪ ма-
рупольскую тюрьму, онъ сохранидж такой начальственный

видъ. держаль себя такъ свысока, то порицая, то одобряя,
4=
`

2 —

й
1
1

то лагражлая обружающихь будущими объщалями, что не
только низние чины тюремной `адмивиетразии. но и самъ
смотритель тюрьмы подвергся дъйетьню превеликаго сомиЪя
и смущеня. Кавъ скажеть бывало Ильяшенко съ величеет-
венной ‘ввушительностью: „Нопомнини, меня! увидишь скоро,
одобряю, награжу“, — то и иойдеть въ тюрьмв топот,
отчего и пронеходило смятене‘ тюремныхь начальственныхь
душь. Омущало везхьъ. что такого преступвика фацьше пе
бывало, и боэтому неудивительно, если вов ованчивали свой
яшопогь заключешемъ: „нфть, туть что-то не спроста“.

Этому сматенио душь содфйствовало и то обстоятельство,
что внЪ тюрьмы разросталея, кавъ вкругъ оть брошеннаго
въ воду камня, пущенный нфеволькими русскими аюльми
незёный слухъ, будто Ильяшенко ни ‘вто иной, какъ Великий
Цнязь. Авторы этого елуха основывали догадку па томъ,
палавтиемея мяогимь убфлительномъ сообращенш, что ни у
вого на шсЪ не могло быть портрета Государя, какъ только
у Великаго Князя — брата Государя.

Кань бы то ни было, черезъ пЪекольво дней, по заклю-
ченш Ильяшенко въ марупольскую тюрьму, не было въ
город тапого человфка, который бы ве слышалуь объяснения,
что греки посадили въ тюрьму Великаго Куязя. Этот слухъ
не замедлиль выйдти изъ предфловь города Маруполя, и,
окрфиши, сталь распростравяться по окрестнымь селамъ,
деревнамт, и даже проникъ въ сосфдые города.

ДостовВрно извЪфетно, что весною 1563 тода, въ сель
Макеимимановв, отстоящемъ оть города Малуноля боле
чфмъ на ето верст. сельсый сходъ, собравшись въ присут-
ств старосты, имль совзщан!е о своихЪ нуждахъ и; „между
прочимъ. имфль суждеше о томъ, чго господа помфщики
посадили въ тюрьму Врата Государя, а потому, поговоря
между собою, постановили“...

*

ыы

— 58 —

Тавъ записаваль малограмотный сельсый писарь сущ-
ность того, что составляло предметь разсуждетй на сходё.
По обыкновеню, онъ писалъ постановлеше схода, посляЪ того,
какъ этоть нослВднШ разотелея. Писарь мало думалъ о томъ,
что хотВль ностановить сельск!й парламентъ; по опыту отъ
хорошо зпалъ, чго вее имъ написанное сойдеть за постано-
влеше схода; для этого вфдь стоило ему только вписать
фамими веграмотныхь крестьянъ села и зат6мъ призвать
одного грамотнаго крестьянина и предложить вму роснисаться
за себя и 3% неграмотных; на практик викогла не было
примфра отказа въ учинени такого пустлка, какъ росин-
саться, хотя нн одинъ изъ руконрикладчиковь пикогда не
интересовалея звать, въ чемь ©06т01л0 ностановленте схода
и вообще, что онъ подпиеывалъ.

(исаря затрудняло другое обетоятельство: онъ никакъ
не могь сообразить, что писать поелз слова: „постановили“.
Во всвхь приговорахь. написанныхь имъ раньше, нослб
этого слова писали: раепредфлить землю, нанять пастуха,
обложить платежем и т. д. Никайя тавут слова въ лавномъ
елучаВ не подходили. Писарь тщетно морщиль чело, вер-
тВлея па табуреть, погружалу сусиное перовъ чернильницу,
Ничего не выходило

Въ концф копцовъ, писарь скомкаль Въ рукЪ пачатое
имъ постановлене сельекато схода.

Косноеть деревни препятствовала таинственнымь сау-
хамъ, которымъ вов безусловно вЪрили, вызвать какой нибуль
значительный эффекть. Люди удивлялись коварству помВщи-
ков (которыхъ. въ слову сказать, въ мартупольскомь уфздё
теперь совебмъ нЪттъ, а въ то время было не бол8е 5 или
6 человВкъ), и затбыъ, пошум$ въ иногда на сеходЪф, входили
в спокойную колею ‘деревенской жизни, не предпринимая
вичего дъйствительнаго и цвлесообразнаго относительно в9з-
мутившихь ихЪъ покой собыий.

ПБ: Ш ад

+

3 — 954 —

Но зато ма) польсмя дамы, почуявь таанствениое,
восиламенились къ Ильяшенко и старались поддержаль его
репутацию. Он потянули въ тюрьму кавъ на ‘богомолье.
Благо въ то время никакихъ. тюремныхь стротостей и фор-
мализма пе практиковалось: веяый кто хотфлъ могъ, безъ
веякато разрёщешя, посфщать тюрьму, котда ему было угодно
и сколько разъ ему было угодно. -

Боюсь, что неловфрчивый читатель сейчасъ спросить:
кажъ-же, при такомъ отсутстым надзора и строгостей, вс
арестанты не убЪфгали изъ тюрьмы. Замзчу на ато попро-
шелощему, что и я ставиль старожиламь такой же вопрост;
они отвьзчали мн, что не полагалось бЪгать и что въ то
время эрестантовь бфжало не больше, чВмыъ теперь. Всякий,
молъ, зпалъ, что начальство не дозволяеть бЪжаль.

Благодаря такимъ удобнымь услойямъ для посфщеня
тюрьмы, дамы стали самымъь щедрымъ образомъ наносить
визиты узниву. Ильяшенко ихъ принималъ, ни па минуту

. не измфняя своей пачальственной величественности, & награ-

диль многихь изь посзтительниць обфщан!емъ, что онъ ихъ
въ надежное время ВСПОМНИТЬ.

Одинъ изъ достов®риЪйшихь старожиловь объясниль
мнЪ, что дамы въ общемь дфалф ухаживанья за Ильяшенко
не ссорились между собою, ибо въ это ухаживаие ромьни-
чесвя чувства не входили. Даже, напротивъ, проявляли са-
мую усердную, солидарную заботливость объ Ильяшеако,
онф съ общаго вовхъв сотлаея, ввели принципь раздфленя
труда: барыни изъ у6зда узлами привозили для питант уз-
ника фрукты, овощи, маело, молоко и всякую живность,
городев1я-же дамы” ежедневно приносили въ тюрьму горяще
пирожки, обфды, сладости и т. п. Но особенно трогательно
выразилась заботливоеь дань въ доставленыи узнику всякаго
потребнато бфлья и костюмовъ. Въ этомъ отноптеня щедрость
Е

й
5
я

ихъ была такъ велика, что по свилфтелеству старожиловъ,
Ильяшенко, каждый разъ, вызываемый къ слфдователю нА
допроеь, являлся въ новомъ вкостюмЪ.

Паломничество дамъ въ тюрьму не особенно импони-
ровало тюремной администрати и начальнику тюрьмы. Но-
слъдшй смотржль на дамъ, вакь на народъ несерьевный,
большей частью не понимают того, что дВлаеть. Но зало
посъщен1е разныхь неизвфетныхь гоеподъ, прЁзжавшихь
издалека и усердно стремившихся и добивающихся азумении
у заключеннаго, продолжавшаго себя держать ©ъ величест-
венностью персоны наполняло ядомъ тревогь и сомнфый
бЪдную душу начальника тюрьмы.

_ Окончательно его добиль ныЫй полковникь и нёкй

`тенераль, посБтивиие Ильяшенко въ оцинъ и тотъ же день.

НолковниьЪ, послБ аущенщи, быстро вышель, вотрахнувъ
везиь тБломъ и быетро ушелъ, бросивъ одно только слово:
„удивительно“. Гонералъ-же на обращенный къ нему во-
просъ: „вакъ полагаете ваше превосходительство?“, отвфтилъь
только „поразительно... не сироста“... и, усФвшись въ свою
коласпу, запряженную тройкой добрыхъ донскихъ коней, ука-
тиль въ свое имзыюе на донскую сторопу.

Поелв этого начальникъ тюрьмы вналь въ такое безио-
койство, вакъ будто ожидаль, что ежесекундно налъь его
тюрьмой разразится небесный огонь. Мимо камеры ИльязшенЕо
онъ не ходиль иначе, какъ дрожа веБмь т$ломъ и на пы-
почках. Въ его голов зарождалась даже мыель, что не
лучше ли будетъ, если онъ явитея въ камеру узника и по-
вергнегь, къ его ногамъ во ключи тюрьмы. Во веякомъ
случаЪ съ каядымь днемъ пазальникъ тюрьмы все боле
поднадаль подъ воздфйстые укрБплявшихся слуховъ, полу-
чавшихь силу абсолютной достовфрности и создавшнхь Иль-
яшенко ореоль таинетвениаго зеличя. (Юращене за совЪ-
томъ кф овружеющимт, вабъ всстда въ такихь случаяхъ,
было безполезно. Дюли говорили много, умно, а въ итогв
дарили совсВмъ безплодпый совфть: будьте осторожны, емо-
трите въ оба, поступайте, какъ лучше... Оть столь рЪши-
тельнато шага, капт полнесене ключей тюрьмы, начальникъ
этой посяФдней былъ удержанъ только благодаря тому обето-
ятельству, что изъ Екалтеринослава прибыль произвести елвд-
стые особый чиновникъ губернатора, командированный своймъ
начальствоимъ съ этой спешальной пфлью.

По мЪрф тото, накь разросталось сядет, тюремная
алминистратя успокаивалась,

Идльятенко-же не издаль духомъ и на допроеЪ нпро-

полжалъ держать себя по прежнему, кавъ начальникъ. На
первомъ допроеВ онъ даже смутиль слблователя, когда на
предложене объяснить. чёмъ онъ занимался, отьфтиль: „р
нахожусь по службф, някому объяснить ве желаю, вром®
Госуларю Императору“. Но слЁдователь быль изъ опытныхь
и ве поддался первому впечатльню. Онъ -осторожно соби-
ралъ сВдетвенный матераль, который оказался убШетвеннымъ
для разыгрываемаго Ильяшенко фарса. Тфмъ не менфе это
слЗдстые не дфйствовало угнетающе па пашего героя. Бла-
тодаря заботлизости дамъ, опъ, за времз пребывав въ ма-
ртупольской тюрьм%, откормился, поздоровЪль и, прилично
одфтый, сталъ болфе походить на персону, ч5мъ въ тотъ
моменту, когда онъ въ длинныхь емазныхь сапотахъ и ста-
ренькомъ коричневомь потертомь пальто, впервые явился въ
канцелярю вачальника марупольской команды. Тавимъ 00-
разомъ, въ благополучномъ еостояни Ильяшенко оставалея
въ Маруполь до осепи 1863 года; затЪмъ его перевели въ
Алекеандровекую тюрьму, гдВ онъ сразу попалъ въ тяжелую
обстановку. Въ этой тюрьмф на него не обрацали внимания,
в онъ очутился в положени обыкновеннаго рядового аре-

—=

станта. Улруленный холенмъ пребывамемь въ оТюрьыт, Иль-
Ященко ухнатилея за послвВднее средетво; онъ началъ пиезль
длинныя прошеня ва Высочайшее вмя и въ Юкатеринослав-
скую палату уголовнато суда. Въ эхихь прошевях» онъ
жаловался на свое’ положене въ тюрьмЪ, на свое „неогра-
ниченное жертвоприношен!“ въ служеши государственнимь
цфлямъ, но ничего не помогало; судебрал воловита тянулась
евоимъ обычнымъ ходомь и только 96 мая 1964 гола
‚алевсандровеюмй уБзлный судъ обще съ тородевой ратушей“,
въ качеству сула первой инетанши, приступил въ раземо-
трёню лфла объ Ильяшенко и ето сообщнякахь: купиЪ
Поддубнв, МазинЪ, Волосеовскомъ и Пичахчи,

Кажь мы знаемъ, первые три изъ названных лиць
вотрьтилась еъ Ильященко въ №. Бердянск в два изъ нихъ
сопровождали ето въ Мартноть. Что касается Пичахчи, то
слЬлетыемь было установаено, что Ильяшенко его постав
наканунЪ своихь знамевательныхь дйстй въ г. Мартулоав.
По отеутетвно добстаточныхь уликъ, на основан которыхь
можно было-бы установить евязь въ дфйслыяхь Ильяшенко
0 вефми названными лицами, уЪздный сулъ въ евоемь вер-
диктз постановить означенныхъ лиць, .прихосновенаыми къ
сему двлу пе читать“. (СЛ. д. 42 тамъ-же) Что же васа-
етея Ильяшенко, то судъ, признавь его яВйствовазшимт въ
добромъ .здрайи и виновным, полвель его лВйетия подъ
большое количество твхъ статей изъ разнахь попцовъ бозь-
того ХУ т. св. завоновъ, по перечислении конхъ, слЗховало
роковое ваключене: & посему опредЪлиль; по лишении вофхь
правъ состоян!я, гослать Ильяшеняо въ каторжныя работы
безь срока. |

По счастью Ильяшенко рфшеше сула не было оконча-
тельно, и онъ воспользовайея своимъ правоуъ перенести
ДЪло. во 2-ю инстанщю, въ Икатеринославскую палату уго-

== 58. Не

зовнаго суда. Это судебное учреждев!е отнеслось къ участи
Ильяшенко ‘болфе гуманно: ‘оно обратило внимане на его
„неограниченное пертвоприношене“, на отсутегне корыстной
пли въ его дфйстыяхь, быть можеть также на его роль
Пемезиды, воздающей каждому по двламъ его, и подняло
вопроеъ о нормальности его душевной дфятельности въ мо-
менть совершеня преступленя. Правда, что ототъ воир®сь
не особенно уб®дительно согласовалея со везми обстоятель-
ствами лЪла, что и заставило Уфздный судъ отвергнуть пред-
положене о ненормальности подсудимаго; тфмъ (не менЪе,
при старомъ формальномь суд, таковъ быль единственный
путь для спаеевя Ильяшенко. Заря новыхь вфавй, нред-
шествовавшая ввеленшо суда присяжныхь и подготовлявшая
почву для правосумя, основывающагося на принцио сво-
бодной совфети, внутренвяго уб®еденшя, правосущя, судящаго
не только дВяшя, но и самаго преступвива — эта заря вЪ-
роятно бросила свой зарождающийся съЪтЪ и въ отживавиия
послёдше дни старыя судебныя учрежденя. Формализиъ въ
въ дВИствительности слабЪль, и это принесло спасене Иль-
яшенко. Еказеринославская палата, уголовнаго суда отнеслась
въ преступленлямь Ильяшенко не только съ точки зрашя
удовлетвореня капцелярсвихь, формальных» условй; она
тщательно разобрала внутреннюю сторону дфянЙ подеудимаго
и нашла, что Ильяшенко дЪйствоваль не въ здравомь ум,
а потому ршенемъ, состоявшимся 2 августа 1865 года
опреявлила: преступленя совершенныя отетавнымъ чертеж“
викомЪ Ильяшенко не вифнять ему въ виву.
Освобождеше Ильяшенко отъ всякаго уголовнаго нака-
зан!а вызвало приливъ бЪфшенства у стряпчаго Хартахая.
Его раздражала мысль, что судебное учреждене косвенно
признало, что не только всё Мартупольсыя власти, но и
онъ, Хартахай, стряпч, стражь закона, принимать сума-
50

сшедшаго за высокопоставленную персону. Воть почему,
воспользовавшись своими нрокурорекими правами, какъ стряц-
Шй, Хартахай разразился жалобой въ сенать на рёшеще
Ккатеринославской палаты уголовнаго суда. На многихь
листахъ дохазываль онъ правительствующему сенату, что у
Ильяшенко умь самый здравый, что всВ его поступки въ
МарулолВ верхь тфлесообразности и разумности, что едва
оправданный Ильяшенко тотчасъь же получилъ мЪ$ето съ го-
довымъ оБладомъ въ 500 руб. вь Екатеринославской гоерод-
ской дум ит. д. По веф эти доводы оказались тщетнями:
сенатъ оставиль протесть Хартахая безь уваженя, удержавъ
съ него въ пользу казны 3 руб. 60 кон. пошлинъ за не-
правильную жалобу. (Л. д. 156—158 тамъ же). До рёшешя
сенала Хартахай очень горячо выражаль свое негодоваше
противь рЬшешя палаты и всакаго встр$чнаго обдавалъ
фразой: „помилуйте, какой же онъ сумасшедиий, развЗ бы
я могь исполнять приказан!я сумаешедшего“ |... Увфдомден-
ный о рёшен1и сената, Хартахай примолкт, и только отиле-
вывалея, когда заходила рзчь объ Ильяшенко.

Но стряпчаго и весь составь суда ожидало еще боль-
шее огорчеве. Въ ннварЪ 1869 года утверждено опредЗлеше
палаты тавого содержаня: „Бывшимь членамъ мартупольскаго
греческаго суда: Попову, Газавджи, Ганжи, Охсюзову н
секрегарю Хартахаю виновнымъ въ бездВйстви власти сдЪ-
лать на основан 343 ст. улож. о нак. замВчане“.

„Воть такъ правда“, говорили осужденные. „ВмЪсто то-
го, чтобы пожалфть людей, надъ которыми потлумилея, наемял-
ел шарлалант, яхъ обвиняють въ престунленш но должности“!

Вфролтно, нвкоторыя лица спросять мена, зачЪмъ я
возстановиль всю эту маловфролтную, фантастическую исто-
— 60 —

ро, въ тБйствитольность которой викте бы не новфрилъ,
если бы Л въ своемъ изложети не опирался па безспорный
доказательства. ОтвЪчу кратко. Мвв хотЪзось этой истомей
показате прежде веего, наскозько городъ Мамуполь. пазуз
чивиий въ семидесятихь годахъ городегос самоунравлене
на началахь не исключительности, по равенства вебхв ио-
’редь закономъ, подвинулея вперещь въ своемь развийи. Че
только въ настояние дци, но уже въ конц® семидесятыхт
толовт, исторя Ильяшенко казалась ма\упольцамт гакимт-то
чудовииинамь сномъ; до тазой степени яваялосью бы совер-
нение невозможнымт повгорене чего либо полобнаго тому,
что продфлаль Ильяшенио: Но отмфчая такого рода прор-
реесъ, д въ тоже время хочу напомнить, что и по лынЪ въ
мартупольсвом» увздф, какъ и во’ многихь другихь уфзлахь,
можно вотрутить большия и малыя села, гдВ свободно и се-
годвя можно повторить оныть Ильяшенко, продзланвый им
31 зЪтъ тому назадт. Мн лично извФетенъ случай, когда
одинъ тосподинъ, получивь въ дарь отъ казны амфие за
свою службу и цоселивиись, но зыходв въ отставку, на
подаренной землЪ, заблагоразеудилт присвоить себ неогра-
ниченную власть въ смысл коптроля и управлемя дВлами
ближайших, сельскато и волостного правлений. Это случилось
за нфеколько ЛЁтЪ до введешя института земекихъ началь-
никовтЪ. Вновь поязвивнийся землевладлець, опиразеь на
газетные слухи © будущей роли земскихъ начальников и
ъесьма опгибочно объявивъ, что онъ именпо получить наз-
ванную должность, фактически присвоиль себф тажя права,
вавихъ не получили и будуше представители судебно-адий-
нистративной власти. Тавъ управлён1е этого начальника-во-
лонтера блатополучно продолжалось два года, до тВхь поръ,
лова не явился настояпий земскй начизьникъ, упразднивиий

его фактическую, въ предБлахь волости, государственную

28

ке

ео

=
а

..

р

дфательность. Нродфлка Ильяшенко и веб подобрыя невф-
роятныя шутки вытекают изъ одного общаго источника:
изъ полнаго отсутотыя въ нашемь обществ элемевтарны хи
прелставленай о дЪйствующихь учрежденяхь, законахъ пра-
вахь и обязанностаяхъ,

Но этоть факть я равьше подробно указывал в» с6о-

ихъ относащихел &ь этому предмету реботахт: равным

образомъ, я ущазываль на удачную полытку вфкоторыхь за-
падно-евролейскихь тосударетвь ввести. Элементарное озпако-
ылеше съ существующимь государствевнымь и юридическим
строемъ страны въ начальныхь пародныхь училищахь. Тогда
ще л объясниль, что настала пора осуществить ту же пр-
пытку и среди нашего крестьянскаго населеня. (П. В. На-
менсюй. Преподаваюе граждапекой морали вь народныхъ
школах. 1896 г. Харьковъ).

Наконецт. возстановляя историю Ильяшенко я считалъ,
что этоть маленьюмй опизодь интересенъ тВыъ, что онъ яв-
ляется прямымъ результатомь тёхъ вфян!Й, благодаря кото-
рымъ держалось убЪждеше, что для блага общежимя пужпы
правители, знающ!е тольхо исполнительность, дисциплину и
проникоцеся сознаюемъ, что опий „не могутъ смЪть свое
суждене ныть“,

Мысль, что оргавы Власта должны ‘быть, хотя бы въ
скромпой мЪрБ, проевьщенными людьми еъ устойчивыми мо-
рально-правовыми представлемями считалась не только заслу-
живающей внимания, но иросто вольнодумствомъ. Помнауйте,
говорили в® то время, (аи тенерь подъ часъ повторяютРь
таъ называемые ревните\н порядка). вочему эти громым
слова, и 60635 нихъ люди жили, — главное, чтобы чиновники
дфао дфаали и прежде него въ точности исполняли прика-
зая начальства безь хитростныхь размышиленй; тогда ната

общественная. жизнь быстро освободится оть тяготбющихь

Ето ==

надъ ней недостатковъ. Сердцу сторонниковъ такого взгляда,
долженъ быть очень любезенъ капитанъ Лисенко. Онъ вЪль
такъ былъ выдрессированъ своей эпохой, что только и жаж-
даль исполнять, не разеуждая. А вВль члохо могло придтиеь
мартупольевимъ обызателямъ, если бы Ильяшенко, ошал®вити
отъ удачнаго исполневя своей роли, приказалу, писциплини-
ровапному капитану колоть или стр®лять греческй судьи
именитыхь граждан. Бапитанъ Лисенво не замедлиль бы
показать столь желанную исполнительность; не даромь ка-
питанъ въ своихъ позднфйшихъ признаняхь соржествевно
и съ паеосомъ объявиль: „прикажи Ильяшенко стр®лятЕ,
веЪхъ-бы перестрфлялъ, приважи колоть, вофхь переко-
лолъ-бы“. Хорошо, что дфло ограничилесь лишь бритьему
головы.

и №

„Ликар - стоматолог

208
Ярнтия Эт

^я
