
%title
%page1

П. В. Каменский

История Одного Дня

Достоверное Сказание

Ликар-Стоматолог ВДОВ

Валентин Дмитрович

ЕКАТЕРИНОСЛАВ.

Типо-литография М. С. Копылова. Проспект, собств. дом

1900.

%page2

Дозволено цензурою. Харьков, 25 сентября 1900 г.

%page3

ИСТОРИЯ ОДНОГО ДНЯ

(ДОСТОВЕРНОЕ СКАЗАНИЕ)

Около 20-х чисел марта 1868 года в г. Бердянске,
в гостиннице ``Белого Лебедя'' ореховский 3-й гильдии купец Поддубня,
торговавший чаем, дегтем и салом, рассказывал двум своим знакомым, Мазину и Колосовскому,
о приключившихся ему злоключениях в г. Мариуполе. В числе слушателей
находился еще третий слушатель, неизвестный Поддубне, но также внимавший рассказчику;
назывался он \textbf{\em Григорий Власов Ильяшенко}. Это был молодой человек 33 лет, блондин, внешней 
фигурой ничего из себя особенного не представлявший, никаких, как говорится в паспортах, особых примет не имевший.
Он был уроженец гор. Николаева: в описываемое время проживал с женой в г. Бердянске, где занимался частно
чертежными работами у местного архитектора. До прибытия в Бердянск Ильяшенко
находился на службе в севастопольской инженерной команде морской строительной части
чертежником и был награжден, как это удостоверяется
официальными документами, бронзовой медалью на Андреевской ленте.

Разеказ Поддубни не был связный. Он начал с объяснения, в чем состояло в г. Мариуполе
особое греческое управление, но его объяснения были мало уяснительны,
%page4
ибо они сводились на повторение одних и тех же слов: каторжные греческие порядки,
проклятое греческое царство и т. п. Понятно, что эти повторные фразы,
переплетаемыя с ругательствами, никак не знакомили 
с учреждениями, которым подчинялось в то время греческое население.
На самом же деле, ознакомление с этими учреждения не представляется делом сложным.

Переселившиеся в конце прошлого столетия из Крымского полуострова греки
основали город Мариуполь и 353 греческих села.  По позднейшим законодательным
актам они составили особый греческий округ, и населявшие его греки подчинялись
исключительно созданному для них учреждению, называвшемуся греческим судом. Это
было учреждение одновременно судебное, административное и полицейское.  Его
компетенция распространялась только на греков, другие национальности ведению
этого суда не подлежали. Состав суда состоял из председателя, и трех членовъ,
которые назывались заседателями, секретаря и подчиненных ему столоначальников.
Веё дела решались коллегиально, и только по обычаю, но не по закону;
практиковалось, что на одного заседателя возлагались полицейские обязанности,
на другого обязанности следователя, а третий заведывал хозяйственной частью.
Исключая секретаря и подчиненных ему столоначальников, остальной состав
избирался греческими поселенцами на трехлетний период.

Для этого один раз в трехлётний период каждое
из 28-х греческих поселений посылало в гор. Мариуполь
двух уполномоченных, которые, вместе с представителями города, избирали часть состава
греческого суда, т. е. председателя и трех заседателей. По обычаю, принявшему вследствие своей неизменной
повторности силу закона, избирался всегда тот кандидат, который ставил избирателям больше вина.
%page5
Так как в состав суда попадали большей частью люди, не только не блиставшие административным и судебным опытом,
но даже недостаточно грамотные, понятия которых не шли дальше вопросов о ``паленице и льне'', то,
для наставления их на путь законности и достодолжного уразумения сих путей, правительством назначался
и присылался из губернского города Екатеринослава нарочито к сему приспособленный секретарь, именовавшийся
еще стряпчим и обладавший прокурорскими правами.

Из дальнейшего разговора, происходившего между посетителями гостиницы ``Белого
Лебедя'', можно было узнать, что в конце 50-х и начале 60-х годов членом
греческого суда состоял некий обыватель гор. Мариуполя, Логафетов; он ведаль
полицейскую часть, был, таким образом, администратором, имёвшим своей задачей
споспешествовать благоустроению мариупольского греческого округа и его
обывателей. В этакой своей роли он ничего памятного потомству не оставил. Но за
то, как полицейский чин, руководивший обходами во время ярмарок в Мариуполе,
Логафетов приобрел широкую известность; молва соединяла его имя с целым рядом
мелких и крупных преступлений. Знаменитые логафетовские обходы, надолго
сохранившиеся в памяти потомства, обычно состояли в следующем: под своим
начальством Логафетов составлял отряд из сторожей греческого суда и тех молодых
людей из города, которые во время ярмарок поступали в его отряд в качестве
волонтеров. С таким отрядом, заседатель греческого суда Логафетов отправлялся
на ярмарку ``смотреть за порядком''. На самом же деле вся эта банда предавалась
разврату, кутежам и обжорству, благо, в ярмарочных кабаках и трактирах этих
охранителей порядка безвозмездно кормили и поили, в виду
%page6
их начальственного значения. Расправы же и суда искать
было негде, по дальности расстояния от высшего начальства, губернатора и по безплодности
результатов расследований, производимых присылавшимися чиновниками-ревизорами.

Однажды, во время такого начальствования на ярмарке со своим обходом, Логафетов повстречал казака
Капсуленко, который резко ответил на требование не курить.
Произошла свалка; Капсуленко пал, получив смертельную рану в бок, нанесенную ему дротиком,
который Логафетов носил вместо палки. Хотя смертельный удар и не был нанесен лично
Логафетовым, а одним из его споспешников, тем не менее, в гор. Мариуполе
все передавали за достоверное, что заседатель - убийца казака
давали за достовфрное, что заеЪдатель — убмйца казака, что
разследование по делу об этом преступлении не дало никаких результатов, вследствие того, что
Логафетов успел склонить людей своего обхода к даче ложных показаний, которыми его обелили.
Молва указывала на изворотливость Логафетова перед старой следственной властью и добавляла, что Бог
покарал некоторых нечестивцев, лжесвидетельствовавших после присяги: уверяли, будто
лжесвидетели были поражены внезапной смертью сейчас,
после дачи ложных показаний.

Обо всех этих изложенных обстоятельствах разсказывал Подлубня своим слушателям. 
Одновременно передал он и им и то, что лично над ним проделал Логафетов в 1861 году.
``Приехал я, значит, с товаром на ярмарку. - 
говорит Поддубня, — расторговался на первый сорт,
спустил всякого товару. Выручил чистоганом на деньги
5,015 рублей, запрятал деньги под жилетку и, конечно,
напилея пьянъ, потребовал музыку''...
%page7
Когда в загрязненном сором ярмарочном трактире три оборванныхе музыканта,
вооруженных двумя иструментами, похожими на скрипки и бубном, услаждали пьяный
слух пьяного Поддубни, в другом углу того же трактира, Логафетов во своим
отрядом предавался даровому обжорству и пьянству. Отуманенный чрезмёрной едой и
излишне выпитым, Логафетов сталъ буйствовать и потребовал, чтобы музыка,
услаждавшая Поддубню, немедленно стала бы играть перед ним. Как ни был бы пьян
Поддубня, но он выступил защитником своих прав, говоря, что он музыке заплатил
вперед, и что он не отпустит, ``пока она не отыграет своего''.

Произошло недоразумение, в результате которого Поддубня оказался выброшенным с
завязанными назад руками в какой-то полутемный чулан, названный арестантской
камерой.  Здесь Поддубня заснул крепким пьяным сном, от которого очнулся часов
чрез пять, когда какой-то неизвестный человек стал выталкивать его на свободу.
Ужас охватил узника, когда он, ощупав у себя под жилетом, замтиль отсутствие
денег. Он сразу отрезвелъ и понял, что в конец разорен, так как там под жилетом
находилась выручка за весь полностью проданный товар, он понял, что теперь не
было ни товара, ни денег. Осталась нищета. Поддубня обратился за помощью к
местным властям, но это оказалось бесполезным. ``Там проклятое греческое
царство'' - объяснил он своим слушателям, - ``Логафетов богат, у него большая
родня, знакомства, наш брат ничего не поделает''...  Писал Поддубня жалобу и
губернатору, что вызвало разследование особо командированного чиновника,
который, исписав много бумаги, ни к какому результату не пришел. В общем же
жалобы на Логафетова Поддубни и других лиц, вызвали
%page8
устранение его от исполнения обязанностей члена греческого суда. От этого,
конечно, не было легче Поддубне: деньги его пропали. ``По миру пустил,
проклятый... грабитель'' ... - заключил свой разсказ Поддубня, уснастив свое
заключительное слово теми богатыми по своим оттенкам ругательствами, которые,
(хотя еще и теперь нередко слышатся на улицах в больших даже городах) нетерпимы
в печати.  Когда Ноддубня кончил, Ильяшенко вмешался в разговор. Он предложил
Поддубне ``взяться'' за его дело и пояснил, что раньше с успехом вел пред
начальством дела молокан.  Действительно, безспорными официальными данными
устанавливается, что Ильяшенко посещал колонии молокан, осматривал их на манер
ревизора-чиновника, выражал им иногда свое благоволение, похвалы и обещания
предстательства пред высшим начальством в Петербурге, какового, конечно, он
никогда не исполнял и исполнять не мог.  иеполняль и исполнять не могъ.  При
этом достойно замечания, что Ильяшенко в этом случае действовал ``без всякой
корыстной цели и даже без особой побудительной причины''. (См. в архиве
екатеринославского окружного суда: журнал решений за 1965 г. л. 29 л. д. 14
об).

На предложене Ильяшенко взяться за дфло, Иоддубия
отвтилъ отказомь, замфтивъ... „съ чВмъ же вести до,
вогла я остался голь какъ соколь| „Ну тавъь знайте“, воз-
разиль Ильяшенко, „я вамъ этого самаго Лотафетова въ
вандалахь чрезъ Бердянскъ отправлю въ екатеринославев ю
тюрму!“ До сихъ поръ разговоръ происходижь сравнительно
спокойно и на вы, но, посл еловъ, сказанныхь Ильяшенко;
Подлубня впалъ въ патетическй тонъ и перешель на ты:
„буль благодВтель, запри его въ тюрьму, послЗляюю сороч-
ку сниму, отдамъ!“... завопиль Поддубня. „Ничего не нало,
только поставишь могарычъ®, веливодущничаль Ильяшенко,

—ы=

————)

———————и——цЬ—

ры

ОТ ЕЕ

посл% чего внезапно познакомивипеся заключили другъ дру-
та въ объят1я, и стали пить водву изъ вновь принесенной
бутылки. ВозлБ этого сосуда объединилась вся компан1я изъ
четырехъ челов%къ, и дружно вела громкимъ шопотомЪ ка-
кой-то длинный разговоръ.

День 5 апрВля 1863 гола являлея однимъ изъ тх® ве-
сеннихь дней, когда зима дЪлаетъ послВднюю вылазку про-
тивъ наступающаго лЪта. Было „сиверко“, какь говорить
напгь простой народъ; небо аволокло сЪрыми тучами, кото-
рыхъ не могь прогнать пронизывающий, ни на одну мину-
ту не ослаблявпий евоей порывистоети еЪверо-воеточный в%-
теръ; степь не усп®ла еще покрыться ‘зеленью, и вме
съ сзрымъ небомт, представляла кавъ бы сплошную  е$ро-
вато-бурую массу. Если-бы не длинные дни, да не 060б0е
весеннее обвфщен1е, которое, не см.тря на заврывия сола-
це тучи, таки ощущалось, можно было бы подумать, что
времена Года повернули вепить, что настачь ноябрь мФеяцъ.
Едва образовавиййся, накагъь по дорогамъ испортилея. бла-
голаря безирерывио сыпазшемуся на землю мелкому дождю.
Среди такой обстановки, въ указанный дель 5 апрФая, по
дорог$ изъ Бердяцева въ Мамуполь, верелахь въ двадцати
отъ послБхаяго города, тащилась одноконная подвода, на ко-
торой сидфли уже извфстныя намъ лица: купець Мазинъ,
ившанинъ Волосовеяй и отставной чертежникь Григорий
Власовь Ильяшенко. (См. въ архив екатеринославеваго ок-
ружнаго’ суда дЪ$ло, вступившее въ палату изь александров-
скаго уфзднаго суда обще съ тородовой ратушей объ от-
ставномъ чертежникв изъ оберъ-офицерсвихь дЗтей Гр. Вл.
Ильяшенко я. д. 12).

Верстъ 6046 50 по невеселой дорог® ‘Ъхали они; имъ
оставалось ло Мамуцоля еще вереть 25 такого же невеее-
лаго пути. Предь путниками простиралаеь ” таже степь

== 10 ==

олнородная, сФрая, бевъ деревца, безъ какого то ни было
нейзажа, на которомъ можно остановить глазъ. Но паши

путешественники относились къ этом но видф-.
У С ос ому равподушно и сид

ли на своей невзрачной повозкВ, проникнутые той апалтей,
воторая по необходимости овладфваеть путникомъ, хорошо
напередь знающимъ, что никакими усилями, хитростями,
мольбами, Фдущему пе сократить, пи измфпить долгихъ то-
мительныхе чабовъ пути ереди бозмолвныхъ степей.

Около четырехь часовъ пополудни Ильяшенко и его то-
варищи подъфхали къ Мар!уполю и остановились въ предъ-
метьи города, на Марьинекой сторонф, въ одной изъ нен-
зрачныхь избъ; гостинниць хогда вь МаруцолЗ не было и
пр#зжаюние останавливались въ частныхь квартирахъ. То-
го же дня вечеромъ, когда уже стемнЪло, Ильяшенко розы-
скаль домъ Логафетова и явился къ этому

‘иВднему.
{журнал рЪ$шен]я еБатеринославекой палалы уголонаго ву-
да за 1365 г., дл. 29). Пакъь мы уже знаемъ, Логафетовь въ
те время не быль властью: онъ быль устраненъ отъ дол-
жноеги члена греческаго суда. ТЪиъ не менфе онъ жилъь
припфваючи: быль холоетъ, богаль, не овобенпо старъ, ему
было около 50 ЛЬтъ, недугами не страдалъ, въ тород$ имфлъ
богатыхь и сильныхь родственниковъ, среди которыхь ов
былъ свой и дорогой ихъ сердцу челов ьъ. Словомъ жилось
ему хорошо, беззаботно. спокойно; его служебные подвиги
ето не тревожили, ибо веЪ разелВдован1я о его проплахь
дъяшихь кончилоеь совезыъ благополучно: нио какихъ су-
дебныхь преслёдованаяхъ не было и рфчи. Вотъ почему Ло-
тафетовь свысока отнесся въ Ильяшенко, когда тоть сталь
ему объяснять, что ему грозать пеприятноети по жалобамъ
Полдубни, и вогда Ильяшенко предложиль ему свое  еводВй-
стые для улаживан!я этого ‘дла. Въ конц концовъ Лога-

фетовь грубо выпроводилъ Ильяшенко п. закрывая за вимъ

2

дверь, пропустиль мимо ушей обращенное къ нему воскли-
цаше: „ты меня попомнишь!“... Попазно. что этимъ словамт,
Логафетовъ не придалъ ровно никакого значеня.

*

М: поп раде Ёег: пезетее, 010й пезсвий и миф
не стыдно сознаться, что у меня нЪть умфшя писаль твор-
чески, создавая художественные типы, частично разоросан-
ные въ жизни. Если бы ян обладаль такимъ умфыенъ, я
предпочель бы изобразить описываемыя собымя въ форм
комеди или драмы, и въ этомъ мфег$ изображеня дЪйстви-
тельности опустиль-бы зннавзеъ, закончивъ [-й автъ. Ва-
тВмъ я началь бы 2-й и посабдьй антъ, Такъ внесено бы-
10 бы боле интереса и жизни въ настоящую работу; но.
къ сожальнИю, это не по силамь мнЪ; я только смиренный
афтошисецъ-протоколиеть, и возстановляю забытую и, на мой
взглядъ поучительную исторшо по тому снособу и приему.
какъ составилютъ обыкновенный судебный или полицейсый
протоколъ. При этомъ я пользовалея безспорными довумен-
тами, хранящимися въ архивВ окатеринославскаго окружна-
го сука, проемотръ коихъ любезно быль мнВ предоставлень
почтеннымь предсвдателемъь назвапнато суда И. Ц. Пато-
номъ; затьмыъь я имбю въ своемъ распоряжени розысканную
мМОИМЪ уважлемымъ товарищемъ уфзднымъ членомъ татанрог-
вваго окружнаго суда В. 0. Станцевичемь корревпонделцио
„Сына Отечества“ оть 20 мая 1869 тода и переданпую
ими мнЪ; накоцеинь я широко воспользовался евилтельски-
ми показашями изкоторыхъ старожиловъ г. Маруполя, ко-
торые были пе только непосредствениыми евидьтелями, но
даже отчасти участниками описываемыхт, мною собычй. Поль-

О >

= Эр

зуюсь случаемь принееть мою глубокую благодарность этямъ
старожиламъ, изъ которыхъ нфкоторые не пожелали, чтобы я
упомянуль ихъ имена; я ихъ благодарю 38 ихъ длинныя, под-
‚робныя показашя и главнымъ образомъ за поразительную вЪр-
ноеть’и точность ихъ свидфтельствъ, въ чемъ я убЪдилея со-
поставляя ихъ показаня съ документальными данными, хра-
вящимися въ архив скатеринославекаго овкружнаго‘ суда.
Посл этого объяснев1я продолжаю прерванный протоко.гь.

6 апрбля 1863 года писарь, зав дывавитй кандзлярей
„Начальника Мар1упольской команды внутренней стражи“,
по заведенному порядку, въ 8 чае. утра, явилея въ пеболь::
шую комнату, на дверяхъ которой быль прибить полу-листъ
бумаги съ криво выведенною болыпими буквами ладписью:
„Канцелярия“. Писарь быль изъ молодыхъ; эту должность
онъ занималъ только еъ 1862 тода. Раньше онъ входилт
въ составъ команды, сформированной въ г. Екатериноелавв
изъ 40 радовыхъ; команда была сформирована въ 1858 тоду,
и въ тоть же тодь была прислана въ горедь Маруполь
подь начальствомь оя постояннаго начальника капитана Ли-
сенко. Въ команду, о которой илеть рфчь, входили соллалы
лучшей репутащи, имфвое нативки. По евидфтельству еза-
рожиловь въ то время господствовала строжайшая военная
диецаплина: „лосталочно было чихнуть во фронтВ, чтобы
получить въ морду оть начальства“, привожу подлинныя
слова одного бывшаго солдата команды. Замфчательно при
этомъ, что, давая тамя показанля, старожилы изъ бывших
соллатъ тВмь не менфе лобавляютъ: „жить было презрасно“.
Лалье они объясняють что веб знали лпругъ друга, что „во-
ровезва, грабежа. мошепничества, извозчиковъ и гостиницщь
не было“. Очевилно, значить, что продфаки и преступленя
Логафетова въ счетъ не пыи: вЪрозтно, ихъ игнорировали

изъ уважешя кь его начальственному состоянию.
=

Писарь войсковой команды отличался какъ образцовый
„службисть“ и за образцовое исполнеше диециплицы, въ
1860 году, быль произведенъ въ уптеръ-офицеры. Обязан-
ности писаря онт исполняль съ тВмъ-же неизмзннымь усер-
демь, съ тВмъ-же строгимъ исполненемъ дисциилины, какъ
и обязанности рядового.

Пияйля 6 апрфля въ свою канцеляр!ю, писарь подошель
къ большому, неуклюжему, грубо сдфланному, косому шка-
фу, досталь изъ него н№еколько толетыхь тетрадей въ по-
тертомъ переплетВ и положиль-ихъ на тутъ-же стоявпий
большой столь, тацой-же пеуклюлый, вакъ и шнафъ.

Заз ть, обмакнувъ въ пузырекъ съ черниломъ гусиное
перо, отъ сталь этимъ скрипучимь орудемъ выводить буквы
ва сВрой, немного мохнатой и похожей на войлокъ бумагв,
иЗЪ которой состояли книги для входящихь и исходящихь
бумагъ.

„Въ то время“ (такъ показываеть бывиий писарь, ны-
нЪ благополучно проживаюний въ гор. Мартупол6), твердымъ
шагомь, не спёша, во и не медая, входить приличный, 60-
аидвый, съ бравымъ видомъ господинъ лётъ тридцати. Роста
онъ быль выше средняго, блондинъ, одфть вь плохенькое
пальто евЪтло коричневаго цвфта, замфтно потертое; въ та-
41а пальто обыкновенно наряжены приказчики, стояние за
рилавкомь Нензвьетный, войдя въ канцелярию, поздоровал-
ся въ писаремъ; послЗдшЯ всталъь, вытянулся, повлонилея,
молча сЪль и сталь опять усердно вызволить буквы, но душа
его была уже охвачена какой-то неясной тревогой.

Писарь почувствовалъ, что необычайное появлене ненз-
вфестпаго господина не спроста,

Между неизвфотнымь и писаремъ въ это время завя-
зался слВлующ далогъ.

Неизвьствый:— Кло начальниет?
Де, |

Писарь. — Штабеъ-капитамъ Лисенко ..

Неизвестный. — Хорошо-ли обралщается съ солдатами?

Иисарь.— 'Такъ точно, все обетоить благополучно, поз-
вольте доложиться начальн...

Неизвьстный (рЪзко обрывая).—Не падо останься...
Есть нисчая бумага?

Уже послф второго вопроса писарь понять, что пред-
чувстве его пе обмануло; ему показалось очевиднымт, что
предь нимъ какой-то большой пачальникь, „ибо никто иной,
кавъ тольшо начальникъ #2 станеть спрашивать, какъ обра-
щаются съ солдатами“; отъ этой мысли онъ завотновался,
„весь затряеся“ и дрожащею рукою положиль на’‘столь 6
листковъ бЪлой бумаги.

Между тм, „большой начальнику“ уже отдаль сухо
приказъ:

— Напиши на этой бумаг, что я «кажу...

Оторопь охватила писаря, въ головё мелькнула мысль:
удрать, убъжать, но это была только мыель безъ р5шимости;
на самомъ дЪлЪ писарь боялся тронузься съ мфета, Онъ только
смогь произнести тихем®. умоляющимь толосомтъ:

-— Позвольте доложить начальнику команды.

Но па эту мольбу пеизяъетный еще бол$е сухо, повы-
сивъ голосъ, отввтилт:

— Не надо... Пиши’

Писарь покорно сфлъ передъь зистомь бфлой бумаги.
Неизвбоехный, поглядывая въ пямятную книжку, вынутую имъ
изъ внутреннаго кармана своего верхняго пальто, сталь дик-
товать, а писарь записывать сяБдующее:

„По данной мн власти Государя Императора Всерос-
сШскаго Александра Николаевича, въ Царекомъ Сел, 18
мая 1862 года, лицомь и Именемъ Которато повелфваю:

—4

—

ЕН

бывшаго засвдателя сего суда Николая Логафетова, за гра-
бежь и смертоубйство“... —

Но силы писаря ему измфоили. Уже когла онъ выво-
дилъ слово: „Государя Императора“, ему сперло духъ ий ояъ
чувствоваль. какъ земля ухолитъ изъ подъ его ногъ, Котла
же были упомянуты преступлешя Логафетова, о которыхъ
шопотом» и почему-то со страхомъ говорилось въ город%, какъ
будто-бы всф были соучастниками его преступлен, писарь
везмъ своимъ существомъ постигъ. что передъь нимъ—вели-
кую власть имущая персопа, и такъ заволновалея, что его
руки съ гусипымъ повят ерели’ етиснувшихся пальценф
затиеенене па бумагЪ. Начатый листъ быль иепорчель по-
явившейея на немъ чернильной кляксой и до неузлаваемостя
безобразно выведевными буквами послБдвихь словъ, отра-
жавигими Аляску руки на бумаг.

Неизвветный прекратиль диктапть и повелительно ека-
залъ писарю:

— Ветань, оправься, пройдись по комнатф, не надо
дрефить!

Пока писарь оправлялея, неизвзетный еталъ около яве-
ри, отрфзавъ путь къ ОЪгетву.

— Ну, теперь нишги,-—оцять прехазаль ноизвфетный,
и. поедВ вписана на чветый листь уже написяннаго, иро-
должало дивтовать слбдующее: „и вообще за всЪ злоупо-
требленя лишить веБут правъ состояня еъ сеылкою на Алтай-
све заводы въ вфчпые работники. име продать еъ пуб-
личнато торга и удовлетворить вебуь должниковь и претен-
дентовъ; а остальное зат$мъ должно поступить вт казну“,
Проемотрвь написанное неизвегный собетвенпоручно „под-
писуетъ“ его такъ: „Полномоченный Государя моего, врно-
поданный Григорий Власовъ Ильяшенко. Марлуполь, апрЪля
6 дня, 1865 года“. Подпись эта не произвела внечаза я

— 16 —

на писаря, который, посяБ перенесенныхь тревогъ, пере-
сталь временно реагировать и только пассивно продолжалъ
Нисаль подъ ДиктовЕХ, еше два приказа, тексть коихь я
приведу въ точноети несколько далБе.

Веф 3 приказа, написанные на 3-хъ отдфльныхъ листахъ,
Ильяшенко спряталь во внутреный варманъ своего пальто:
затБыъ, ставь влотпую ипредъ писаремъ и свазавъ „емотри“
Бириотнухь верхы1я пуговицы жилета и вынулъ изъ’ подъ
хилеза сплалной медальонъ изъ красной м$ди, зиоБвиий на
Апдреевской левтф. Въ этомъ медальон быль портретъ
Государя и подъ крошкой кусоченъ бБлой бумаги на коей
ниВлаеь надииеь: „быть по сему, Алексанаръ 2-й, Царское
село, 17 мая №302 года“. Эти слова были написаны обык-
новеннымь почеркомъ Пльяшепко, кавъ это удостовбрено
судебной экспертизой п гакъ это было очевидно впослёдетвия
лля воякаго обозрфвавшато вастоящую подпись. (Я урналь рз-
шений Юкалер. Налаты угол. суда за 1565 года, л. 39).

„Смотри“, товориль Ильяшенко; настуная на писаря;
„ты знаешь, кто я таковъ, видить, чфмъ я вагражденъ отъ
Государя. Зпай, что сели ты изофетили, начальника комавдхы

ревийв мною суда, сегодня же будешь повфшенъ“! ©ъ
этими словами, произнесенными съ зловфщимъ грознымъ пто-
потомъ, Ильяшенко быстро вышель изъ канцеляри. У пи-
саря отнялея духъ, опять потемифло въ глазахь, застучало
въ толовВ: не донести начальнику— Ода, донестя еще хуже,
мелькнуло въ его сознан!и; недвижныый, словно окаменфлый,
остадея онъ въ своей канцеляр!и, вакъ будто тысяча желфз-
нахъ цфней приковали его къ мЪету.

Николай Оомевичь Логафетовъ, не обративъ никакого
вниманя на угрозу почти вытолкнутага имъ изъ своего дома
Изьншенко, па утро совершенно о немъ позабыль. Шо обык-
новеню хоропго выспавшись, онъ веталь рано не спёша сталь
одфваться, поверхностно умылея и преданея спокойному чае-
пит ю. Часовъ около 10 утра онъ отправился въ сосЪлнюю
лавку своего приятеля В....нова, гдф онъ бываль каждое утро
и тдф каждое утро пруятели вели одинъ и тоть же разговоръ.
Во 1-хъ, они говорили, что жилось бы гораздо лучше, если
бы всегда можно было быть увфренныхъь купить товарь де-
шево и продать дорого. Во 2-хъ, оба они печаловались друг
другу 0 томъ, что вь МаруполВ начинають селитьея при-
шельцы изъ разныхъь концовь Импери, не греки; за ними,
говорили между 060ю приятели, и на базар не успфешь
пичего купить; при этомъ они вспоминали, какъ остроумно
ихъ общй другъ, иышй мфетный обыватель, выразиль про-
тестъ противъ тапого вошющаго положеня вещей. Этоть ихъ
пругъ, идя на базаръ купить курицу, увпдфле, что одна изъ
вновь поселивигихея русевихъь женщинъ возвращалась съ ба-
зара и несла въ корзин$ вурнцу. Такое предунрежлене на-
иБрен греческаго поселенца возмутило его духъ; онъ оста-
повить своего копкурента въ дВлВ покупки курицы п, вы-
уваливъ изъ корзины птицу, бросиль женщин® 20 копЪекъ,
при чемъ быль настолько галантенъ, что представиль ей
объяснеше своего пострика: „за вами чертями пичего нс
успфешь купить!“ „Юще 20 коп. унлатиль“, добавляль Ло-
тафетовъ тВмъ-яае тономъ. изъ котораго безусловно явество-
вало, что онъ лично ограничилея бы только отобрашемъ
курицы безъ вслкихь дальн Яшихт дЪНетвй. Въ особенности
же Логафотовь и К—вь въ своихь собъездаваняхь возмуща-
лись ТЬмъ обстоятельствомь, что пришельцы открывали въ
городб лавки и явлинись въ торговиВ опасными конкурен-
тами. Вь то сравнительно недавнее время уеловя обществен-

ной жизни въ Мар1упол® не походили на сегоднашея. (е-
2

= а

тодня смфшно говорить о МарунолВ, какъ исключительно
греческомъ городЪ; въ немъь нЪфть серьезнаго различя по
нацюнальностямь: вс обыватели этого торода постененно
емфитиваютея и во всякомь случаЪ положене вовхъ обывате-
лей равны передъ закономъ и властяни. Тогда по свихътельству
сларожиловь, дЪло обстояло иначе. Пришельцы не греки,
составляли ничтожное меньшинство. которое побанзалось гре-
ков, предетавлявшихь большинство, силу, въ нотороф не
могли бороться новые поселенцы, Кром того, греки имфли
свое самоуправяеше, свою обусловленную привилаегированную
организацию, которая ле распростравялась на невыходцевъ
изъ Крыма. Послфдье подчинялись общинь дореформеннымь
учрежденямь, поль еЪнью козорыхь жилось не легко. Таку
продолжалось до начала семидесятых головъ, когда введено
было городовое положене на общихь основашяхь До этой
же поры греки, выходцы изъ Крыма, занимали господетву-
ющее положене, & остальные обытатели угнетенное. Греки
были недовольны пришельцами, а посл6щн!е ихъ побаивались
и нитали къ нчмъ недобрын чувства.

И тавь Логафетовь и КЪ-овь затянули свой обычный
разтоворъ; на этотъ разь онъ былъ скоро прерванъ. Изь
гречесваго суда прибфжаль старикъ сторожъ и на турецкому
нар чи (многче изъ грековЪ поселенцевь п теперь сохранили
это нар е) передать Логафетову, чтобы онф посизшиль
придта въ судъ, что ‘его требуеть предефдатель Поповъ.

Лотафетовъ паправилея ВЪ будь совершенно спокойно.
Случалось и раньше, что его требовали иной разъ въ при-
сутетые дая нфкоторыхь неважныхь разъясненй, которыя
енъ могь представить. какъ бывийй членъ суда; правда, что
чаще эти объяененя переводились съ мфета на собесфдова-
ня по душф, которыми совершенно затушевывалась главная
ЦЗль вызова. Воть почему, даже не безь нЪкотораго удо-

= РА -ВЕ

вольетыя, Логафетовъ направился въ приеутетые, предвку-
шая хотя воегла одинъ и тоть же. но тфмъ не менфе веег-
да одинаково для него интервеный разговоръ о торговл%, о
„пашениц$“, о вловредныхь новыхъ людяхъ, поселившихся
недавно въ Маруполв.

Въ это время одинъ изъ такихъ новыхъ поселенцев,
почтенный вупець —овъ (нынЪ благополучно проживающий
зъ МаруполВ, развивпий и многократно возвеличивий свое
предимяе) пришелъ въ свою лавку и еёль по обыкновение
на табуретВ, стоявшемъ у двери, выходившей на улицу. Моро-
силъ мелый дождь, но вфлеръ, свирфно дувиий накапунв съ
еВверовосточной сторопы, измфниль направлеше, сталь луть
съ юговостова я въ возлухЪ начала разливатьея весенняя,
оживляющая мгръ, мягкость; предшествующий холодный день
оказалея на самомъ дБлб послфднимъ приступомъ зимы, уто-
мившей непривыкшихь къ холоду южанъ своими длинными
бездЗаятельными днями.

Купець —овъ, смотря на улицу, среди мертвой тиши-
ны, громко передаваль себБ свои виечатленя' „И вуда это
Иванъ Мавловичь (начальникъ нивалилной команды, вапи-
танъ Лисепко} пофхарь въ дрожкахь А вотъ за ипимъ че-
тыре солдаты бфруть рыецой, ружья держать полъ шинелью,
чтобы дождикъ не замочилъ. А среди-то всфхь во какъ ко-
выляеть ногами старый соллать цирюльник: ему то сердеч-
ному, старому и хилому, не легко пософвать за строевыми!“,

Такъь въ мысляхь купца —ова отразилось видФнное
имъ на улипф и онъ осталея сидфтзь спокойно на своемъ
эиБетф, праздно и разефянно размышляя на тему: куда бы это
позхаль Иванъ Шавловичь? |

Эти размышленя были по прошествя нфкотораго про-
межутка времени, прерваны позвлешемь мастерового, Ано-
хина, инфвшаго ошал$лый видъ. Въ согласии съ его\/ббыч-

о
2

а:

нымъ вившнимь состоянемь были и его дЪфйстьыя: войдя въ
лавку, онъ началъ, безъ всякихъ къ ‘чому побудительных
иричинъ. учащенно и усердно креститься.

— Чего врестишьея? —вопросилъ удивленный купець
-—— 9ВЪ.

Зъ отвЗтъ посынались отдфльвыя елова: „Чудеса Ка-
питанъ! Саблю па голо!.. Четяре создата?.. Логафетовь ит. п.

— Да ты въ ума сошелъ. — прервалъ безевязный по-
токъ словъ купецт —овъ и увелъ разстроеннаго пос’бтитезя
въ слЁдующую комнату, гхБ июпотомь Авохинъ продолжалъь
столь странно начатое повфетвоваше,

Я уме замфтиль, что когла рфчь заходила о дБявяхъь
Логафетова, то марТупольске обыватели говорили шопотомъь
и 0 страхомъ, какъ будто они были соучастниками Лога-
фетова. Анохинъ же не только усвоить эту общую вофмъ
привычку, но сверхь того быль криведенъ въ трепеть ви-
дфннямъ и опасался какъ бы не быть въ отвЪгБ за то, чте
онъ разсказываетъ видЪфнное... Впрочемъ. возсотановлять со-
быя по отрывочнымъ словамъ и фразамъ Анохина я не
берусь и предночитаю возвратиться къ имфющемуся у меня
точному, ЛоетовВрному матералу.

Ильяшенко, оставивь писаря въ канцелярти, направился
въ греческ судъ; тамъ онъ засталъ одного столоначальника
уголовваго отлЪлешя суда, Могулянскаго. ПослВднему, какъ
говорятт, Ильяшенко сразу поставиль вопросы ребромъ и
налегь на него со стремительностью. „Гл предефдатель и
члены суда“? — „Составь суда выфзжаеть сегодпя для разбора
дфлъ въ Яалту“.— „Вы коронный“? (т. е. состоите-ли ва ко-
ронной службЪ или выборный).-- „Да, коронный"... Ильяшенко
показаль портрегь Государя, послБ чего столоначальникъ
превратился въ воплошепный вопросительный и восклица-
тельный знаки. „Ни слова `!--грозно крикнулъ Ильяшенко. —

=

„Немедленно послать за предсфдателемь и членами“! Столо-
начальникъ пустился бфгомъ исполнять приказъ. Ооставъ
суда (предсфдатель и три члена) въ это время состояль изъ
людей основательныхъ, уже немолодыхъ, умудренныхъ, такъ
сказать, годами. Одинъ секретарь быль молодъ, но зато онъ
быль умудренъ знатемъ на намять нзкоторыхъ законовъ и
наипаче циркуляровъ; это признавали во и боле вефхь
увфренно самъ секретарь.

Члены греческаго суда, надо сознаться, при зеей со-
лидности, страдали полнымъ неум$н!емъ составлять суждене
о дБлахъ, подлежащихь ихъ раземотрфнию, но это не пре-
пятетвовало ‘функшюнированрю названнаго учрежденя, ибо
въ этомь слузав выручаль секретарь в0 своими циркуляр-
чндами п законами. ОнЪ лисалъ опредфлене, ссылаясь па
проставленныя им статьи, а председатель и члены суда мед-
ленно и старательно ихъ подписывали. не постигая ихъ со-
держа, что, однако, не залерживало течешя дЪль. Въ
остальномъ секретарь низфуъ достопримфчательнымь не вы-
дзазлея. Можно толъво отмфтить, что уже прежде познашя
завОНовЪ ОПЪ постиеь, что канъ законы, такъ и циркулявы
существують лля угождевя начальству.

Не щинило и двадцаен минуть моел6 ухода Могулян-
скаго, кАКЪ затыхавшись, но уже въ настоящемь одфяшин
при падлежащемь знакф, явились: предефдазель, три члена
суда и секрегаръ. Ильяшенко продолжаль дФйствовать съ тою-
же стремительнаетью, съ какою ошъ обратился къ столона-
чальнику. Едва поздоровавшись, онъ вручить председателю
тотъ приказъ, который онъ продиктовалъ. писарю и текеть
поего приведенъ выше. т

Проедсфдатель навелъ глаза на врученную ему бумагу;
вообразваль онъ туго, мециенно, но все таки довольно скоро
поняль, 910 случилась офда, и вь душБ у нес» похололвло;

|
5
ь>
|

онъ не зналъ, что сказать, что дфлать. Ильяшенко вызвелъ
его изь состояня нерфшительности, потребовавь дать ему
надежнаго человзка, съ которымъ онъ могъ-бы послать важ-
ную и секретную бумагу начальнику комачды, капитану
Лисенко.

Предевдатель позваль одного изъ сторожей, которому
Ильяшенко и вручиль въ запечатанномь конверт второй при-
казъ, заранфе написанный подъ его диктовку, нписаремъ ин-
валидной команды. ПредсВдатель и по русски, и на тубецком
нарЪч1и напутствоваль посланнаго наставленями. Ильяшенко
затВмъ объявиль предефдателю, что судъ не пофлеть сегодня
въ Ялту, такъ вакъ ему предстоитъ заняться разсмотренемъ

чрезвычайнаго государственнаго дфла о Логафетов, которато:

Ильяшенко потребоваль немедленно призвать въ присутстве
суда. Остальныя неопредбленныя, но наставительнаго харак-
тера, общия указан1я весь соетавь суда выслушалъ, етоя и
молча, находясь въ состолЫи людей, которыхъ пришибли
сильнымъ ударомъь по голов.

Капитань Иванъ Павловичь Лисенко былъ старый слу-
жака: еще до севастопольской войны онъ служилъ въ канто-
нистахъ и за усердную службу былъ назначенъ фельдфебелемъ
селенгинекато пфхотнаго полка. Во время” севастопольекой
войны онъ храбро сражался, былъ раненъ и произведенъ въ
подпоручики со старшинствомъ. Когда въ 1858 году была
сформирована команда изъ сорока рядовыхъ, четырехъ унтеръ-
офицеровъ, шъ которымъ впослдетви былъ прибавленъ одинъ
барабанщикь и одинъ пирульникъ, Лиеенко, въ чинф пору-
чика, былъ назначенъ начальникомь этой команды. Въ томъ-
же 1858 году команда изъ Екатеринослава совершила пере-
22

=

ходъ вь Маруполь, гдВ и оставалаеь подъ начальством
Лисенко, въ скорости произведеннаго въ капитаны. Такимъ
образомъ въ марупольскомъ округ капитанъ Лисенко ока-
залел единственнымъ и властнымъ представителемъ военной
власти. Съ виду Лисенко былъ настозиий богатырь: въ пле-
чахь сажень, грудь какъ большой котелъ, а ростомъ онъ
быль выше на голову каждаго изъ нижнихь чиновъ своей
команды и даже каждаго изъ обывателей города Мартуполя;
голосъ его вызывалъ содратаве даже неодушевленныхъ пред-
метовъ. Лисепко могъь пить, но всегда быль трезвъ. Въ евоей
служб относилея строго; знаяъ и исполняль дисциплину
въ совершенетвЪ. На парадауъ, при маршировкЪ, при фронт%
священнодЯствоваль. Въ сохдатамъ отноенлея строго, спра-
ведливо и заботливо. Соблюден1е дисциплины требоваль не-
умолимо и караль за всякое мал5йшее ея парушен!е, въ
ковець м > расположене его духа. Даже лучшаго
и образцовато унтеръ-офицера онъ наказаль на первый день
Сътлаго праздника за го, что онъ ва парадф, поставленный
противъ ярко-весенняго солнца, „натужилея“ и чихнулъ,
поелЪ чего Лисенко, разтн®вавшись, сейчаеь же прекратиль
парадъ, считая дВло испакощенныхь, Тисенко привыкъ, умфлъ
и любиль иеполнаять безпрекословно приказатя и ирн этомъ
зналь вее, что оть него требовалось. Одного онъ не зналь
и не умблъ за полнымь отсутстыемь практики въ этомъ от-
ношени: опъ не умфль думать, и всей душой невавидль
таное положене, когда ‘ему приходилось отвЗчаль на вопросъ
кавь поступить, ибо онъ могф только поступать, но не об-
суждать, кавъ поступаль. Такъ, напримфру, еслибы онъ но-
лучиль приказь перебить марупольекихь обывателей, ему
легче было бы исполнить такое требоване, чёмъ обдумывать
и размышлять надъ вопровомъ, слФдуеть ли, возможно” ли
исполнить такой приказъ.

А в

Жена Лисенко, Дарья Кондратьевна, была, основательная.

дама, сорока восьми лёть (Арх. екалер. окружн. суда, ука-
занное выше дЪло, л. д. 64). Грамоты она совезмъ не знала.
По крайпей м$Зр3, поль ея показанемь, даннымъ слБдетвен-
ной власти, выфето ея подциси значится отм тка сл$дователя:
„не грамотвая“. За то она была несравненная хозяйка,
жизнь которой хлопотливо протеказа среди неизмнно каж-
додневно чередующихся заботъ о вухнЪ, погребЪ, о солепьях»,
о боршБ и т. д. Гыгантъ капитанъ Лисепео по внфшности
быль подъ пару своей женЪ. ВКакъ почти всВ мужья, ваци-
танъ слегка побанвалея своей жены; послбдняя же е-зен-
пиатеняи. м р

заеь ©5 почтешежа, главнымь образемь, къ начальственному
положению своего мужа и менфе къ его личности. Несмотря
на свое уметвеннае неразвише, напитанта вовсе не отноеи-
лась къ службЪ и дрительности своего мужа съ тфмъ пре-
зрительнымь высоком уемъ, которымъ угошалютъ нерЪлко
внфшне приличныя, но въ душ% полудивя женщины, вытиед-
пя замужъ за писателей, художников, мыслителей и т. п,
выдающихся людей. Дарья Кондратьевна была неразвита,
по не обладала дикой душой, стремящейся въ разрушелю.
Поэтому она мирилась, кажъ съ необходимостью. и съ фрон-
томъ, и еъ маршировкой и съ приносимыми пзъ капцеляри
бумагами на квартиру въ капитану. Больщимъ необходимымь
зломь гашитанша считала приносимыя ия домь бумаги, ибо
онЪ всегда вызывали нфкоторое безповойство и тревогу ка-
питана, а черезь то и нфкоторый безпорядоюь вь дом: ка-
питань, поглощенный подписомъ бумагь, запаздываль къ
обЪду, кричаль, звалъ, отдаваль распоряженя ит, д. Мирясь
с0 всфмъ этимъ, Дарья Кондратьевна, хотя и не высказывала,
все закн въ глубинЪ души смотр$аа на вс эти бумаги и безпо-
войства, кавъ на дЪъло пустое; она понимала, что изъ веого этого не
выйдеть ни хорошей начинки для пироговъ, ни солен!й, ни борща.

4

$

— 95 —

Воть почему, когда она первая ветрФтила посланнаго
изъ греческаго суда еъ приказомъ Ильяшенко, опа пренебре-
жительно равнодушно приняла бумагу

двумя пальцами въ
тфхт видахъ, чтобы на бумагВ осталось поменьше бураковато
кваса, въ который были смочепы ея руки, и бросила эту
бумагу на пропитанный жиром кухонный столь. Объ этомъ
моментВ вотъ вакъ свидфтольствуеть самъ капитан Лисепко.
Привожу доподлинное его показаве, сохраняя его стиль и.
ороографию: „Прикаязъ Ильяшенко присланъ изъ суда неиз-
вфстнымь челов6комъ, переданъ въ руки моей жены на
врыльцв во время бытности ея въ кладовой... сказала миф,
что требуютъ въ судь, сама же какъ говорить, понесла про-
дукты въ вухпю, въ то время я быль въ другой вомнат%
передфвалея изъ одежды“ (Л. д. 60 указанпаго выше дЪла).
Тевсть этого приказа, написаннаго какъ мы знаемъ, писаремь
воманды, полъ диктовку Ильяшенко, быль слёлующй: „[-
сударь Импералоръ Высочайше повезбть соизволилъ, въ Цар-
скомъ сел, 13 мая 1862 тода, привазать вамъ исполнить
мое требоване, въ отношени спокойстыя и прекращеня
злоупотребленя въ Новоросейскомъ краб. НынВ поручаю
вашему благородио приготовить съ ввфреппой вамъ команды
четыре челов$ка съ однимъ унтеръ-офицеромъ и одного ие-
строевого цирульнива, при которыхъ должны находиться нара
кандаловь и арестантекое платье. Веф пронисанные нижне
чины и вы сами лично обязаны явиться въ тотъ часъ и ми-
нуту какъ получите сей приказъ, въ марупольсвй греческай
судъ, гдЪ, принявъ преступника, съ первымъ этапомъ отира-
вить къ могу назначен я за строгимь карауломъ, а также
поручаю вамъ на будущее время, въ случа важныхь про-
исшестый въ. Мар!уполВ и ближайшихь ему окрестноетей,
немедленно донести миниетру внутреннихь дфлъ, сталеъ-се-
кретарю Валу минуя свою прямую дистанцио“. Сл давала

ПР

затБыЪъ подинсь Ильяшенко, тождественная съ подписью на
первомъ приказ, только лослё нея рукою Ильяшенко было
добавлено: „Въ десять часовъ одиннадцать минуть, городъ
Мар!уполь, шестого апрЪла, 1863 года“.

Въ своихъ дальнЪйшихь показашяхь капитанъ сознает-
ся, что онъ не обраталь вниман на то обстоятельство,
что приказъ написанъ хорошо ему извфетнымЪ позчеркомъ
писаря его команды. Произошло это вслфдстве того, что
слова приказа: „Государь Императоръ Высочайше повелфль
соизволилъ“ сильно „ветревожили“ капитана.

По показаню старожиловь, внутренная тревога Лисенко
проявилась наружу слфлующимь порядком»: солосомъ, впо-
собнымъ на смерть ушибить не подготовлениато елушателя,
капитанъ сталь кричать: „одфваться, олфваться“!... Много-
хратнымъ повторенемъ этого слова капитанъ хотВаъ 06ъ-
яенить, чтобъ ему помогли секорзе облачитьея въ парадную
форму.

Дарья Кондратьевна, хотя и была привычна къ голосу
своего сунруга, однако, подскочила нЪевольво разь на м-
ств, какъ тоБующИЙ тетерев, испуганная неслыханнымъ еще
столь сильнымь крикомъ капитана: она даже обронила на
земь т „продукты“, что держала въ рукахъ; тфиъ не ме-
ифе она скоро оправилась и устремилась къ капитану, ко-
торый, спфша, никакъ не могь справитьея съ переодБва-
темт въ парадные узЕые брюки. Ея помощь была какъ
нельзя боле кстали задыхавшемуея оть волнешя ея супру-
ту. Пока онъ окончательно подправлялъ себя, Дарья Бон-
дратьевна усиЪла отдать приказан1е кучеру, чтобы онъ за-
прегъ лощадь и позвал ифехь солдатъ, цирульника и ба-
рабанщика.

Хотя волнене и спзхъ и помфшали дЪйствовать съ той
быстротой, капая требуется военными людьми, тЪыь пе ме-
Я ОП

нфе капитанъ достаточно скоро появилея во двор въ вад-
лежащемъ видЪ и, не теряя ни минуты, сталь во главЪ че-
тырехъ застывшихъ предъ нимъ солдатъ, цирюльника, и ба-
рабанщика, съ которыми и двинулся въ путь въ томъ по-
рялкЪ, въ какомъ ихь видфль купецъ—овЪъ, когда онъ изь
своей лавки смотрёлъ на улицу. Каоитанъ. озутившиеь на
улицё и освБженный слегка моросивтимъ дождемъ, опра-
вилея отъ волненя и чувствовать въ своей душ ту отвагу,
ту рёшимость немедленно дфйствоваль, какой онъ отличал-
ся въ геройсые дни севастопольской войны, когда надо бы-

ло сходиться грудь съ грудью съ непр]ятелемъ.

ПоелЪ того какъ, по приказатю Ильяшенко, изъ гре-
ческато суда быль поеланъ одивъ надежный человфкь съ
извёстнымь намъ приказомъ нъ капитану Лисенко, а дру-
той — за Логафетовымъ, самъ Ильяшенко не терялъ време-
ни. Прежде всего онъ приказаль предефдателю про себя
знимательно вчитаться въ переданный ему приказъ № 1-й,
конмъ, какъ намь извфетно, Ильзшенко опредфляль ©0-
слать Логафетова „на алтайсв!е заводы въ вЪчные работни-
ки“. ЗатВмъ онъ приказаль смущенному и въ глубин ду-
бин души разъфдаемому сомнфиями секретарю суда, сгряш-
чему Хартахаю, составить отъ имени греческаго суда опре-
дЪлеше объ осужденш на каторгу Логафетова.

Секретарь пугался и отъ необычайнаго составляемато
имъ приговора, и оть вазавшагося ему нееоглаея опредф-
левя суда съ извфетными ему законами и циркулярами, и,
главное, отъ стремительности хьйствй суда, упразднявтей
всякую судебную логику.

Но Хартахай въ этоть зрудвый моментъь не забылъ
надежно сидЪвшей въ его душф идеи, что все должно с0-
вершать для угожден!я начальству, и онъ рёшиль не про-
тиворвчить Ильяшенко. тфыь болфе, что послёдый прибъгь

вь довольно грозным покривизашямъ и понуванимъ, когда
секретарь обнаружиль зам талельетво. Вотъ почему опред$-
лене быстро подвигалось впередъ, сопровождаемое иногда
и одобрительными восклицанями, въ родЪ: „такъ, взрно,
пиши дальше“.... и было окончено къ прТЪзду капитана
Лисенко.

Случилось тазь, что въ присутстые греческаго суда ка-
питань Лисенко вошелъ почти олновременно велЪдъ за Ло-
гафетовымъ. ЧТацимъ образомъ, вез тлавпыя дЬйствуюцщия
лица настоящей драмы оказались собраввыми вызетВ хругъ
противъь друга. Такое ихъь стечене оказалось въ вызчией
степени благопуятлымь для мисаи Идьяшенко Это яветву“
егь изъ данныхь впослбдетын объяснемй  лЪйствующихь
лиць по вопросу о возникновели ‚.слФпого вЪфровавя“ въ
„лжеунолвомоченнато Ильяшенко“. Капитавъ вБриль потому.
что Эшль требовань въ присутственное место, тдб ве
аеполнялось по приказанию Ильяшенко, (Тамъь же 1. д. ПИ
об.). Предевдатель суда окончательно доуввровалъ, благодаря
„внезапному появлению начальника инвалидной команды ва-
питана Лисенко, да еще съ солдатами и въ парадной ферм;
члены суда увЪфровали „смотря на предсБлателя и секретаря
Хартахая, повиновавиихся привазаямъ Ильященко"”. (Тамъ
же п. п. 16 06.). Наконець предефлатель Оеновъ. вь но-
слфлвихь своихъ обтъяснешяхь еще увазыраль, что уврБиле-
вю вёры въ „лжеуполномоченнато“” солфЯствоваль и Лога-
фетон», который „стояль съ повинной головой, какъ настоящий
виновникь“. Правда, такое объяснене вызвало негодоване
Логафетова, который подалъ жалобу па Понова и писалт,
что эти слова Попова, онъ „принимаеть за преднамВренную
обиду на письмЪ принесенную; невольпо влекусь въ подо-
зрёнио“ (тамт. же л. д. 69—72), --продолжалъ онъ жало-
вальел, что Поповь дфйствоваль не сироста, что у него
о

были заднйя мысли... Но нослБдня, по тщательному розыску,

найдены не были.

Едва капитанъ Лисенко вошелъ въ присутстве суда,
велфдъ за Логафетовымъ. какь Ильяшенко, указывая паль-
цемъ на Логафетова, твердымъ и рЪшительнымъ голосомъ
обратился къ капитану: „именемъ закона приказываю вамъ
арестовать Логафетова“. Рёшимость иЪйствовать проявилась
наружу: капитанъ молодецки выхватилъь саблю изь ножень
и, держа въ рук это оруяые поднятымь вверхъ, громовымъ
голосом скомандоваль солдатамъ стать съ ружьями у двз-
рей и никого не впускать и не выпускать изъ приеутствя
суда. Наступила чауза, Ильяшенко одинъ нарушиль воца-
ривштюел гробовую тишину, объявиеъ секретарю, что онъ
долженъ приготовиться къ прочтенио опредвленя суда. Хар-
тахай сейчась поднялся и назалъ, ве ожидая дальифйшихъь
приказан1й, безь всявой торжественноети, монотонно читать
написанпый имъ приговоръ. Ильяшенко съ м%ета его обор-
валъ и грозно возониль: „Мальчишка, службы не ионимаешь!
Не знаешь кавъ, когда сафдуеть читаль!“... Мгновенно воз-
становилась мертвая тишина. Ильяшенко достать изЪъ-подъ
стортука медальонъ съ портретомъ Государя п, держа его

ь

передъ собою, скомандовалъ; „На караулъ:“ Сабля капитана
Лиеенко едБлала вольть въ воздухЪ и опъ самъ, равно вакъ
и его солдаты, превратились въ застывпия статуи. — „Теперь
можешь читать, только смотри толково читай“, --строго 0б-
ратилея Ильяшенко въ секретарю; послфдшй па этоть разъ
блатополучно прочиталъь приговоръ, который корреспонденту
„Сына Отечества“, въ номерЪ оть 20 мая 1869 г., пере-
даеть „на память потомству, какъ замфчалельный документь
въ афтописахь суда“. Привожу это опредфлене суда 20-

словно, безъ сокращенй. Вотъ оно:

— 30 —

„1363 года, апр%ля 6 дня. По Указу Его Ииператор-
скато Величества, мар!упольсвй гречесвй судъ слушали при-
казъ уполномоченнато Госухаря Императора Всеросейскаго,
Григоря Павлова Ильяшенко, отъь 6 апр®ля, въ воемъ объ-
явленс по данной ему Государемъь Императоромъ власти въ
Царекомъ Сел8, 18 мая, 1862 года, лицомь и именемъ
Его Величества повелФваетъ: бывшаго засфяателя сего суда
Николая Логафетова, за грабежь и сыертоубШетво, и вообще
за веЪ злоупотребленя, лишить вевхъ правъ состоя съ
ссылкою въ алтайстие заводы. въ иЪчные работники, а, имвне
его продать съ публизнаго торга и удовлетворить вефхъ
должников и претендателей. ь остальное затёмъ лолжно
поступить въ казну. Приказали: рёшене это объявить быв-
шему засЪдателю сего суда Логафетову и передать его, со-
рласно личному приказанию тгоспедяна уполномочепнаго, въ
распоряжене начальника марупольской инвалидной команды,
штабсъ-капитана Лисенко, а исправляющему должность за-
сВлателя Ганжи предписать описать и оцфнить вее име
Логафетова и опиеь представить въ судъ. Предевдатель Ни-
колай Поповь. За секретаря А. Ганжи, засфлатель Овстозовт,
исправляющЕй должность сепретаря и страпчаго Хартахай.
Не согласенъ на исполнене и буду телеграфировать г. на-
чальнику губернш. Испразляюций должность стряичаго Хар-
тахай. Въ присутетыи мартупольекаго греческаго суда по-
становлене утверждаю. Полномоченный Государя Императора
вБрибподанный Григорий Власовъ Ильяшенко. 6 апрфля 1863
года въ 113/4 часовь слушали и подписали.

Обстоятельства, которыя будуть указаны въ дальнфйшемъ
изложени обнаружатъ, что приписка Хархатая: „несогласенъ
на исполнене и буду телеграфировать начальнику губерни“,
поздийшаго происхожденя. Первоначальное опредфлеве суда
было нацисано безъ такихъ либеральныхъ оговоровъ и, посль

екрБилев!я этого приговора подписью Ильяшенко и его гром-
кимъ возгласомъ: „быть по сему“, началея немедленно об-
рялъ приведенля приговора въ исполневе.

— Цирульника, бритву, барабанъ! —быстро и громко ско-
хандовалъь Ильяшенко.

Цирульникъ, туть же етоявний возлВ солдатъ, поблд-
пЪлъ: бритва была имъ забыта впопыхахъ на квартир команды.

Зарабань только числилея при командь, но въ натурЪ
его ни было.

Не давая объясненй, цпирульникъ вылетзль изъ при-
сутетыя суда и что есть мочи убетремилея на квартиру за
бритвой. Въ два конца ему пришлось побъжаль болфе версты,
по онъ себя не щадиль и черезъ какихъ нибудь 10 минутъ вер-
нулея еще боле блфдный, безь дыханя, держа въ лрожащихъь
рукахъ какую-то ободранную, изъявленную зазубринами братву.

Пока шли кратыя приготовленя къ обряду исполненя
приговора, Логафетовь переживаль тяжелыя минуты. Въ
первое мгновеше онъ не полностью постигалъь ожидавщую
его участь: онъ хотБлъ вфрить и в$рилъ, что случилось ка-
ное-то непр]ятное недоразум не, которое сейчасъ само собою
разъяенитея; но ходъ собъй съ его стремительнымъ натис-
комъ подрываль эту вфру. Страхъ и отчаян!е въ вонцЪ кон-
цовъ охватили бывшаго засфлателя суда; онъ не зналъ, что
дЪлаль, что сказать въ свою пользу и въ этомъ подавленномъ
состояи, па мфстномъ турецкомъ нарзчи обратился, въ
поискахь за, помошью къ предебдателю суда, но тотъ не
оказалея добрымъ утВшителемъ: указывая па Ильяшенко,
предсфдатель только сказалъ:

— Блаеть его выше закона.

Взглядъ Логафетова случайно упаль на принесенную
бритву и это вызвало у него мысль заявить Ильяшенко,
„910, кажется, теперь не велВно брить“.

5

Стрип Хартахай, привычный указывать греческому

суду законы п давать указан1я, добавилъ отъ себя:
—- Брать отм цено.

Этого Ильяшенио не могь простить и онъ сердито раз-

несъ стряпчаго, приведя его въ состояше настоящего емятевя.
— Мальчишка! — кричаль Ильященко, — ничего не попи-
маетъ; отм5нено брить содержащихся, а уголовныхь пре-
ступниковь, ссылаемыхь по конфириащи куда слВдуетъ,
должно брить по прежнему.

— Подожди, — переводя дыхаше добавиль строги реви-
заръ, доберусь и до тебя потомъ и тебф не миновать Сибири!

Поелфде!я слова какъ-то особеппо глубоко запали въ
душу стрличаго; онъ подумаль: „Ито не безъ грБхь... Неро-
венъ част... всяюъ человвкЪ вт волф начальетва“.

Шанеовь на сопротивлеше у Логафетова не оставалось
никанихь, ибо капитель Лисонко быль въ той фаз рЬши-
мости дЪйстровать, когда не оставалось такихъ препятствай.
когорыхъ бы онъ пе еокрушияь ради исполневя праказа
начальстра. Опъ самъ впослдетвт передавалъ своему зна-
комому, пыниинему секретарю ма]лупольскаго стБзда, г. В — му,
что въ тотъ торжественный момент» онъ тавъ увФровать въ
Ильяптенко, что мурашки у него бЪгали по спинВ и волось
еъ залылка поднимался. „Если бы Ильятенко“, — иередаваль
всегда восторженно и съ экстазомъ капитанъ, —въ тоть мо-
менть приказаль „коли“ —всбхь бы перекололь, еели бы
скомапдоваль „пли“ — воВхъ перестрляль-бы“,

По прошествии многих годовъ капитань всегда при-
ходизь въ экстазъ при воспоминая1и о знаменательномъ мо-
мент; предоставляю всякому судить, какой силы быль подъ-
емгъ духа капитана въ самый моментъ лфйствя.

По кивву головы Ильяшенко капитанъ Лисенко и два
создата приблизились къ Логафетову и, не встр$тивъ ни

петь

мальйшаго съ его стороны сопротивленя, посадили ето, за
отсутстыемъ барабана на сундукъ; найденнымъ въ присут-
сти полотенцемъь ему связали назадъь руки. Поблф этого
ноги его одзли въ кандалы. ЗатФиъ къ нему подетупиль
дрожавиий пирульникъ; въ правой рук онъ держаль на
готовЪ ту единствеяную, изъязвленную зазубринами бритву,
которая числилась въ инвентарЪ, находящемся при команде.
Старожиль Маруполя, передазавийй мнЪ порядокъ бритья
половины головы, пояенялъ, что ‘цирульникъ, иослё бФголни
й со страха, нсполниль бритье неправильно. „Сами поду-
майте“, говориль разсказчикъ, „началь брить съ затылка
противъ шерсти. Ну, конечно, задраль кожу; опъ хотя и
примазаль, & Логафетову все таки было непрятно, и врозь
текла каюъ изъ рзанаго“. Эта мелочь ни на минуту не
остановила обряда и бритье было закончено безпренятственно;
посл чего на голову осужденнаго солдать натянуль арес-
тантекую шапву; въ завлючене бывшему засфдателю развя-
зали руки и одфли его въ арестантсый халатъ. Логафетовъ,
считавшийся оцаспымъ человфкомъ, котораго всяк еще такъ
недавно побаивалея, въ виду его крутого нрава, вдругъ сд-
лалея кротокъ, кавъ агненокъ и старательно исполнялъ все,
что отъ него требовали. Въ произведенномъ алеке— имъ ис-
правникомь стёдетыи этоть учиненный падъ Логафетовымъ
обрядъь названъ „дфломъ о невинномъ истязати мЬщанина
Николая Логафетова въ присутстни марупольскато греческато
суда“ (1. д. 155, тамь-же). Относится ли слово невинный
въ истязанно или Логафетову остается не выясненнымъь. Но
когда происходили указанныя выше дфйстйя надъ Логафе-
товымъ, никто не думалъ объ истязани; всф знали только,
что приговор приводится въ исполнеше на законномъ ос-
новани. Воть почему капитанъ Лисенко не замедлиль цо-

ставить Логафетова между солдатъ, изъ коихъ два стояли

3
|...

т

сбоку, одинъ спереди и одипф сзади. Самъ-же капитанъ,
получивъ оть Ильяшенко привазъ отвести „преступника“ въ
тюрьму, сталь впереди и, держа саблю наголо, приказаль
отряду слФдовать за собой. ВекорЪ собравшийся на улиц
вародъ съ любопытетвомъ смотрфлъь на шестые отрада, въ
воторомъ среди солдалть, путаясь въ кандалахъ, шествоваль
бывиИЙ начальникъ, еще такъ недавно наводивиий страхъ
н& м$етныхь обывателей. ВсБ смотр®ли на удалявшагося Ло-
тафетова со страхомъ, нЪкоторые съ грустью; болбе-же всВхь
уныло емотрЪль на кортежъ стряц И Хартахай, изъ смущен-
ной души котораго подъ тяжелымъ. впечатдн1емъ недавняго
разноса вырвалея сл$дующий меланхоличесвй возтласъ: „ве
мы пропали, скоро ве пойдемъ туда-же“... |
Собычя этого дня въ МаруполЪ, какъ видить читатель,
текли съ поразительной быстротой; около 10 часовь утра
Ильяттенко впервые открылся ‘писарю команды въ канцелярти
начальника команды; вЪ 12-ти часамъ дня произнесевъ надь
Логафетовымъ приговор и моментально утвержденъ, и при-
веденъ въ исполнене. Въ 12 часовъ дня капитанъ Лисенко
окруженнаго четырьмя солдатами въ

2

ведеть осужденнаго
тюрьму. Не смотря на такую стремительноеть событй, толна
уепла освфдомиться о чрезвычайномъ событи и, собравшись
въ большомь количеств, длиннымь хвоетомь замыкала ше-
сотые небольшого отряда, предводимаго капитаномъ Лисенко.
Коренные обыватели Маруполя, хотя и хорошо были оевф-
домлены о служебныхь подвигахъь Логафетова, т$мъ не ме-
нЪе не только не говорили: „по дёломъ вору и мува“, но
видимо были смущены и сконфужены. Надо помнить, что

съ самаго момента своего переселев1я изъ Крыма въ Мар]-.

уполь, греки находились въ исключительныхь усломяхь и
пользовались такими правами, какихъ до 60-хъ годовъ не
имо наеелеше Россшевой Импери, за исключешемъ дво-

=

\,

фянства. Не говоря о спещальныхъ призилаемяхъ, замтимъ
только, что греки пользовались правомъ избирать изъ своей
среды составъ треческаго суда, являвцИйся учрежденемт,
которое полностью сови$щало сулебную, полицейскую и хо-
зяйственно-ахминистративную функц во веёхъ дфлахъ, ка-
авшихся грековъ. Это учреждеше было излюбленнымъ д®ти-
зцемъ греческаго общества и вЪ свою вчередъ являлось лоброй
матерью для всякаго грека и самой злой мачихой для вея-
ъаго посторонняго. Греки любили, цзнили и гордились своимъ
треческимъ судомъ и иначе не называли его квакь 0/4
судъ, дВлая ударене на слово наи. Понятно поэтому; что
быстрая расправа надъ бывшимъ судьей греческаго суда,
являлась достаточно ощутимымъ афронтомъ для всего гре-
ческаго населена. За то новые поселенцы Маруцоля, не
треви, терпЪвийе притфененя отъ трековъ, ликовале столь
шумно и неудержимо, какъ это требуется при про№: 1: на-
чальства или при установленныхь приказанемъ начальства
празднеетвахъ. НеизвЪетно почему кто-то въ этоть моменть
вотрфтивнихея ликованШ и смущешй пустиль слухъ, что
явивпийся ревизоръ ни вто иной, какъ вамъ Великй Внязь
Михаиль Николаевичъ. Слухъь моментально распространился
и ему такъ повфрили, что веяый, кто осмфлилея бы заявить
сомнф не, навЪрное, если бы не быль избить, быль обругант,
дуракомъ или еще хуже приписанъ въ бунтовщикамъ, Еф
неблагонау$реннымъ, опаснымъ люлямъ, отрицающимъ власть.
Итак, одни были сконфужены, друге ликовали. а Логафе-
товъ, со скверно обритой головой, въ арестантскомъь халат,
улрученный, сидфль въ тюрьм%.

Предездатель Поцовъ, соглаено доброй, старой традищт
и по нын$8 къ счастью че умирающей, разсудилъ, что кавъ
бы важны нн были государственныя дЪла, твмъ не мене

они не умаляютъ существенной важпости обфда. Поэтому,
3*

послф того какъ Лисенко отрапортовалъ, что престулникъ
завлюченъ въ тюрьму, Ноновъ обратилея къ высокому реви-
вору СЪ почтительной пробьбой едЪлаль честь откушать хлЪбъ-
соль. Предложене было принято. Тогла Иоповъ пригласилъ
также обфдать весь составъ суда, секретаря и капитана
Лисенко, которые весьма рады были и пофеть, и побыть въ
обществ высокопобтавленнаго лица. Обфдъ состоять изъ мфет-
ныхЪ греческихъ явотвъ, и, главнымь образомь, отличался
обимемт разнаго матерала, подлежащаго $48; желающие
могли, не обирая сосБдей, Зеть до достижен!я отвращевя къ
ниш и все таки пе одолбть всего обфда, Памнанекое въ
то время въ МаруполВ не водилось; заздравныхь тостов
произносить не ум8ли и поэтому ревизор не быль осыпанъ
рёчами съ исчисленемъ его добролтелей. Правда, за водкой,
стуча рюмкой о рюмку, участники обФда говорили: „за ватие
здоровье! по чаще! по больше!“... Но дальше этого краено-
р8\е не шло. Поелф нфеколькихь рюмокъ, когда настроеше
людей становится боле откровеннымъ, Цоповъ обратился къ
‘высокому гоетю съ просьбой „по душ“ сказать; кВритъ-ли
онъ полной виповности Логафетова, на что Ильяшенко отв$-
тилъ, что въ виновности этого поелёдняго не можеть быть
ни уалЪйшаго сомнЪн!я. Также не отказался Ильяшенко от-
вЪтить нзкоторымъ изъ присутетвующихь, интересовавтихся
узнать, какимъ образомъ онъ получилъ свой чрезвычайных
полномочя отъ Гостдаря. Не ственяясь, Ильяшенко разека-
залъ какую-то фантастическую исторпо объ его учасйи въ
охран священной особы Государя Императора, цослВ чего
Государь его лично узналъ и осчастливиль особымъ внима-
шемъ и довфрмемъ. Когда же зат$мъ вЪ достаточно обнажен-
ной форыВ высказывалось, что настоящее дфло такого рода,
что 16.000 рублей ассигнащами такая сумма, которая не
зелика, чтобы вычеркнуть все дфло Логафетова и уничтожить

"ей
т

|
©
=
|

ве его слФды, то Ильяшенко прямо отвфтиль: „и радъ бы,
но не могу, ибо аично уполномочень Госуларемь Имнера-
торомъ!* Оббдь не затянулся; черезь какой нибудь чась
времени веф чувствовали себя въ такомъ васыщенномъ ©9-
стояши, что продолжать ду было бы для нихъ мучешемъ,
Ревизоръ выразилъ желан!е отправиться къ себ на квартиру,
при этомъ запротилъ кому либо его сопровождать. Хозяияъ
дома приказалъ заложить лошадь въ дроги — единственцый
въ то время извзетный обывателямь Мар!уполя экипажь.
Пока закладывали лошадь, Поповъ и его сослуживцы про-
сили Ильяшенко объяепить, гдф онъ остаповилея; они ечи-
тали себя обязанными явиться въ ревизору на домъ ш сотротс,
такъ сказать, еще разъ представиться и откланятьея. Но
Ильяшенко категорически отклонилъ всяк!е тавше проекты.
не указаль своей квартиры и освободиль марГупольсвя влаети
отъ явки для представлешя.

Разставаясь съ хозяиномъ и его гостями Ильяшенко
отоввалъь въ сторону капитана Лисенко и вручиль ему тремй
и послёдн приказъ, написавный также, какъ мы знаемъ,
рукою писаря. Этоть приказъ былъ болфе кратокъ; тексть
его сл8дующ!й: „Уголовный преступникь Николай Логафе-
зовъ по корфирмащи подлежитъ ссылкф въ Алтайсве заводы
въ вЪчные работники и долженъ сл®довать чрезъ Екатери-
нославскую губернию, куда немедленно его отправить за ввЪ-
реннымъ Вамъ Его Величества карауломъ до города Бер-
лянска, и оттуда черезъ г. Алексапдровевъ въ г. Екатери-
нославъ, гдф мфстное начальство приметъь свои м$фры для
лальнзйшаго отправлешя. По данной мнф власти 18 мая
1862 тола въ Царекомъ Сел. симъ повелфваю полпомочеп-
ный Государя моего и вфрноподданный Григорй Власовъ
Ильяшенко. Г. Маруполь, 6 апрфля 18683 года, зъ чаеъ
пополудни“. ,

Несмотря на выраженное ревизоромъ желан1, чтобы
никто пе утруждаль себя панесетемъ прощальнато визита,
Поповъ, тёмъ не менЪе, солидно разсуждая, что и излининее
усерме къ начальству только приносить’ пользу, тайно при-
казалъ своему кучеру замфтить домъ, у котораго остановился
начальникъ. Когда Ильяшенко разеталея съ Поповымъ, осталь-
пая компан, посовфтовавшись съ хозяиномт, рЬшила, что
общая явка къ внезапно прибывшему высокопоставленному
лицу безусловно необходима. ПорЪшили, что должны авиться
купно всЪ власти, имя во главф военную власть, капитана,
Лисенко; затЪмъ находили, что было бы весьма великолвоно,
если бы шестые властей замыгали главнЪйпие именитые
граждане г. Маруноля. Оповфетить послвднихъ обязались
члены суда, которые и не замедлили разбЪжаться по разнымъ
нонцамь города. Итакь было условлено, что часа черезъ два,
т. а. между тремя н четырьмя часами дня, ве$ соберутся
шь Попову в озтуда совмбстно ‘направятея во временную
квартиру г. ревизора. До тфхъ поръ каждому предоставля-
лось идти домой для приведеня себя въ достодолжный по-
рядокъ. Стряций Хартахай, пользуясь свободным временемъ,
занель въ своему знакомому, Д—чу, сообщить © чрезвы-
чайномь происшестьи дня. Д—чъ быль изубетенъь всему
Мархнолю, какъ человзкъ образованный, развитой; еверхъ
того совершенно справедливо его считали весьма свфдущамъ
въ законахь. На служб онъ не состояль, проживаль въ
Мар!уполь частнымъ человфкомъ, владфя невдалек® отъ этого
города землей. Люди съ злымъ язывомъ увБрили, что Д—чъ
быль единственный умный человзкъ во всемь городф. ВЪро-
ятно, къ тавому завфренцо нужно внесли поправку и сказать,
что Д-чь быль самый умный среди окружающихь его обы-
вателей г. Маруполя:

о

Угнетенный собычями дня, стряпый Хартахай, вновь
восприняль смущающЕ духъ ударъ, когда Д-_чь на его
слова замфтиль: се шло таке, що не Логафетова, а васъ
зсихъ за вашъ судъ погонять на алтайске заводы“. Д—зъ
любиль уснащать свою рЁчь малоросейскими словами и фра-
зами и его слова, какъ по опредфленности содержашя, такъ
п по е106обу вхъ выражения, всегда производили на слума-
теля сильное впечатлф ве. Перепуганный стряпчй заметался
отъ этой новой точки зрнтя на дфло. Что же лЪлать? — по-
просиль онъ”совфта. Сейчась телеграфировать губернатору, —
опредфленно рёшиль Д— 95.

Этотъ совВть прояениль мысли Хартахая; онъ сразу
сообразить, что телеграфировать по службЪ начальствомъ
разр®шастея и что телеграмма не вызоветь гифва ревизора,
разъ онъ составить телеграмму въ емыслв доношен!я его пре-
воеходительству о прибыти чрезвычайнато уполномоченнаго.
ь Вь такомъ вил онъ и послаль телеграмму. Изъ изло-
женнаго сейчась ясно, почему я выше замфтиль, что про-
тестующан приписка Хартахан на приговор$ суда была едф-
лапа не въ моментъ подиисаня приговора, а позже. Въ тоть
момептъ страпуй не посмфлъ бы противорфчить грозвому
ревизору. Такое мое заключене подтверждается еще слфду-
ющимъ мфстомъ изь позднёйней жалобы Логафетова, кото-
рый, по конедцъ своей жизни остался недоволенъ дЪйетвую-
щими лицами греческаго суда. Логафетовь писалъ: „Харта-
хай телеграфироваль въ ЕкатеринославЪ, не только по. с0-
вершени надо мною истязавя, но едва ли не посл откры-
1 шарлатанства Ильятенко, въ чемъ можно удостовфриться
справками па телеграфЪ“. Сверхъ того, Логафетовь убазы-
валъ. что Хартахай въ телеграмм только доводить ло свЪ-
дня начальства о появлеши уполномоченнаго“. (тамъ же
л. д. 41). | а

— 40 —

Воть почему, говоря словами изь жалобы Логафетова,
„я вевольно влекусь къ подозрн!ю“, что приписка Хартахая
на приговор греческаго суда, сейчаеъ поелФ подписи: „не-
согласенъ на исполнене и буду телеграфировать г. началь-
нику губернии“, — сдфлана впослЪдетви, посл совЪщаня съ
Д-—“чемъ или, вФрнфе, посл полученя отвтпой телеграммы
губернатора, которая была доставлена какъ разъ въ то время,
когда всё собрались у Попова, чтобы купно пиествовать от-
кланиваться въ отьёзжалощей персон%.

Необходимо пояснить, что вт то время телеграммы хо-
дили не такъ, кавъ теперь. Тогда люди меньше чмъ теперь
пользовались этимъ способомъ сношеня другъ съ пругомъ,
поэтому, при соотвтетви числа телеграфнымъ аппаратовъ съ
количествомъ посылаймыху, телеграмм, послфдн!я не зале-
живались на телеграфныхь станщяхъ и передаточныхъ пун-
ктахъ въ роли вандидатокъ, долго ожилающихъ своей очереди.
Словом+®, практика показываеть, что тенерь телеграммы хо-
дятъ у васъ, на югф Росйи, со скоростью оть 8 до 10
вереть въ част и не измфняють этой, практикой установив-
шейся скорости, даже въ экстренныхь случанхь: тажъ, на-
примЗръ, во время недавнихъ безпорядковъ на мар!упольскихъ
закодахъ, телеграфное сообщене торжественно сохраняло
принципь равнопразя и телеграммы начальства аккуратно
прибывали, послЪ прУВзда лицъ, извЪщавшихъ о своемъ при-
быи и дёлавтихъ свои предварительныя распораженя. Но
въ описываемое время такихъ прогрессивныхь новтествъ не
было; телеграммы начальнику губерниу ть него передава-
лись почтительно и безъ замедленй. Конечно, намъ съ на-
шимъ настоящимь каждодпевнымъ опытомъ трудно повфрить,
чтобы существовало такое благодатное время, когда на те-
леграмму, посланную изъ Машуполя. въ Екатеринославъ по-
лучалея бы отвфтъ черезъ часъ или два часа; теперь въ по-

добномъ случаВ, мы должны ожидать отвфта два или три
дня. Тмь не мене. каковы бы ни были наши представлен1я,
факть остается фактомъ, и въ описываемое время, получить
въ МарутолЪ отвфтную телеграмму черезь часъ или два,
считалось нормальным. Чоэтому пусть читатель не удивля-
ется тому обстоятельству, что предуорежденный Хартахаемъ
начальник телеграфной конторы, принесъь отвфтную теле-
грамм губернатора въ домъ Попова, между тремя и четырьмя
часами пополудни, когда веБ власти и именитые граждане
были въ сберф для шеетыя къ высокому начальнику съ

представлен емъ.

Собравшееезя общество оказалось въ превеликомъ за-
трулнени, велЗдетые того, что кучеръ Попова не узналъ, гдЪ
квартира начальника. ревизора. Ве по этому поводу охали,
торячились, спорили, и едипоглаено критиковали несообрази-
тельность кучера. Но послфдый не быль виновать; онъ пе
мотъ узнать, гф квартира начальнива, благодаря хитроет-
нымъ дфйствямъ` этого поелВдняго. ВыЪзхавъ за городъ, на
гору, ЕЪ тому м$ету, тд теперь тюрьма, Ильяшенко оста-
новилъ кучера, даль ему на воду и ириказаль повернуть
и убхаль назалъ въ городъ. Ослуматься кучеръ не поемлъ
и шагомъ поёхаль обратно, разечитывая оглянуться и зам-
тить, хотя направлен1е, куда направить свои стопы началь-
викъ. Но этотъь послЪдвй, вЪроятно, быетро прилегь въ
первой попавшейся рытвинЪ, потому, что кучеръ, оглянув-

шись никото не видЪлъ и только про себя замтиль: „канулъ,

_кавъ въ воду“. ВелЪдетые такого оборота вещей, все обще-

ство, собравшееся у Попова, не знало что дфлать и какими

путями разыскать начальство. Изъ затрудненя вефхъ вывела,
|.

телеграмма. векрытан Хартахаемъ; въ ней значилось: „Ири-

казан й уполномоченнахо не исполвять, немедленно его арес- -

товать“. СлФдовала подпись губернатора. „Д—чъ угадаль!“ —
перепуганнымъ голосомъ возопилъ стряпЙ Хартахай, — „мы
веЪ пойдемъ на каторгу“... У капитана Лисенко первый
разъ въ жизни подкосились ноги и онъ трузно опустилея
на стулъ. Остальные разомъ заохали, зачмокали, застонали.
Хартахай раньше другихъ посгигь произведенную телеграм-
мой губернатора повую шие еп зсепе и у него сейчасъ
же проснулось приеущее стряпчему, прокурорскому оку,
усерме найти автора преступлоены п повергнуть. онаго подъ
варощую длань неумолимаго закона. „Господа“, — первый
затовориль онъ,— „нельзя упустить преступника!“ — „Да ка-
вой онъ преступникъ, онъ невинно сидитъ въ тюрьмф“, —
послышалось въ отвЪтъ нЪеколько голоеовъ. Ве$ такъ освои-
лись съ мыслью осужденя Лотафетова, что почитали слова
стряпчаго относящимлея къ Логафетову. Хартахай взбфсился
оть этой неповоротливости сообразительности и на кончик»
языка виефлъ у него отьфть: „барашы“! Но лица, къ коимъ
должно было приложиться @1е пазваве, уже попали свое
Ча рго Чао, и стали совмфетно и громко говорить что-то
въ свое оправдане, причемъ веЪ усердно ув$ряли другъ друга,
чтб они съ перваго момента, какъ увидфли Ильяшенко, бы-
ли увфрены, что онъ шарлазанъ. Почему рапьше никто не
открылъь такой своей увзренности, объ этомъ никто не ска-
заль пи одпого слова. Веб тЪмъ ве мене говорили много
съ явной тенденшей превознесть собственную свою прони-
цательность. Въ общемъ поднялся неимовзрный гвалтъ: вов
представители греческаго общества жестикулировали почти
всБми органами своего тфла, хватали другъ друга руками
за ворогъ сюртуковь, тывали пальцами другъ другу въ лицо
и ве6 крачали, какъ будто соперничали между собою голо-

сами, подобно опернымъ пЪвцамъ. Бздный Хартахай надор-

валъ свои голосовыя связки, прежде чЪмъ ему удалось воз-
становить кой какую тищину. Ме теряя ни минуты, онъ
сейчасъ же выработаль плапъ поимки преступника. Рф шено
было собравиихея у Попова именитыхь торода Маруполя
гражданъ направить за, поисками лжеуполномоченнаго, (такъ
съ этого момента Хартахай называлъ Ильяшенко), въ пред-
мфетье города, на Марьинскую сторону, куда всего удобнфе
было укрыться съ того мЪфста, гдВ Ильяшенко исчезъ съ
глазъ кучера Попова. Засфдатели греческаго суда должны
были набракь сторожей разныхь присутетвенныхь мфетъ,
вликнуть волонтеровь и съ этими соединенными силами пус-
титься въ обходъ всего города. Наконець капитанъ Лисенко
вомандировалъь 3 создать и одного унтеръ-офипера, тоже па
Марьинекую сторону въ подкрфнлеше къ именитымъ граж-
дапамъ. Самъ же председатель Поповъ, стряцчй Хартахай
и гапитань Лисенко должны были оставатьея недвижимо въ
квартир Попова и ждать извфемй. Ихъ пазпачене состозло
въ томъ, чтобы по открыти м$етонахожденя лжеуцолномо-
ченнаго, немедленно отправиться къ этому послзднему и
объявить его арестованнымт. ]

Цоиски имепитыхь гражданъ и соединеннаго отрада,
руководимаго засфдателями были безуспВшны; на воз ихъ
разсироем имъ отвьфчали, что никакой назальвикь нигдф не
проявлялея. Они ошибочно искали чрезвычайныхь зваменй
чрезвычайнато уполномоченнаго въ видф развфвающатося флага
надь домомъ, занимаемомъ этимъ послфднимъ, или въ вид.
полосатой будки съ заключеннымъ въ ней полицейскимъ и
т. п. Но такихь знамен! не дано было имъ открыть. Име-
нитые граждане ходили толпой, горячо разговаривали, спо-
рили и тоже пичего знаменательнаго не открыли. Носланные
же капитаномъ солдатики замЪтили возлВ одной изъ хать
на Марьипской сторонЪ ту самую одноконную повозку, на
м фи рее

которой Ильяшенко и его товарищи наканунф въЪхали въ
г. Мар!уполь. Солдатики уклонились оть сложныхь разсу-
‘жденй и непосредственио рфшили, что эта повозка никого
иного, канъ только начальника, вЪротно въ пользу этого
быстраго заключен!я говорило то обетоятельетво, что мфетное
павелеше не пользовалось одноконными телфгами, а упот-
ребляло двухконныя, похозйя на современные фургоны. Одинъ
солдатикъ, подкравшись, заглянуть въ окно и увидёль че-
ловЪка, растянувшагося на лавьВ; кавъ заглянув въ окно,
такъ и его товарищи опять непосредственно уфшили, что
никому иному не растягиваться на завьВ, какъ только на-
чальнику и побЪжали отрапортовать о своей находеЁ капи-
тану Лисенко.

Впередь каюсь; на одну минуту я измёню своей роли
лВтониеда-протоколиста и не воздержусь отъ упрека, который
мн хочется бросить Ильяшенко. Мн хочется ему сказать:
„Ильяшенко, Ильяшенко, почто ты, ‘нагрузившись достаточно
за обфдомъ у Попова, напилея окончательно пьянъ, придя
домой. Зачфмъ ты, спровадивъ кучера Попова, не запрегъ
своей лошади и вмВетф съ твоими товарищами ве пере$халь
зерезъ рфку Кальзусъь на Донскую сторону. Для этого надо
было потерять менфе полчаса времени и черезь какихъ ни-
будь полчаса никто никогда тебя не отыскаль бы, и никто
никогда че узпалъ, откуда, еъ пеба или съ земли, появилея
этоть ошеломляюний, властный уполномоченный. Какое бы

осталось широкое поле любителямъ мистики излатать теорю’

о навожденши, любителямъ точнаго зная—6 границахъ са-
мообмана чувствъ, коллективнаго внушен!я; а людамъ бла-
товамфреняниъ, трепеть носящимъ въ душ, представился
бы случай поговорить на тему, что никто не знаетъ ни дня,

9

ни часа, ниже образа и вида, въ которомъ можеть поя-
витьея высшее начальство. О, Ильяшенко, зачфиъ ты напился
пьянь! ".

Прелетавители разнородныхъь властей: председатель Но-
повъ, стряшй Хартахай и капитанъ Лисевко стояли въ
нерзшительности передъ дверьми того дома, въ которомъ
солдалики выедёдили предполагаемаго начальника. Никто
не рЬшалея»войти. Предевдатель говориль въ томъ емыслЪ,
что безстрапе военныхь людей обязываеть вапитана пока-
зать прим®ръ доблести. Вапитанъ ничего не говорилъ, но
рёшительно и отрицательно качалъь головой. Старожилы ут-
верждають, что капитанъ, жестоко перепугавугись посл

получен1я телеграммы губернатора, впаль в% /йервное состо-

яне, что сталъ всего боятьен и даже „самЪ себя боялся“.
Хартахай намокнулъ, что Поповь первое лицо въ городЪ,
но на это, въ видф возражешя, получить отьфтъ, что стряп-
И — око закона. Наконецъ посл долтихъ колебави вез три
представителя власти рзшили войти вмветв. Набравшиеь духу,
они влстфли въ землянку, гдВ почиваль опьянзвиий Ильи-
тенко. Онъ проснулся, протеръ глаза и, услышавъ что онъ
арестованъ, тавимъ громовымъ голосомъ разнесъь вошедшихъ,
грозя наторгой и вефми ужасами Сибири, что вошедше не
выдержали отпора и б%жали изъ хаты, вновь пораженные
тревожною мыслью: & вдругъь передъ ними персона. Такъ
разносить. кричать, ругать можеть только начальство — въ
этомъ вефк трое были тлубоко убфждены. Но поглощенная
Ильяшенко водка опать сму повредила. Пока Ильяшенко
разносиль, явивнЧяся къ нему власти задыхались не только
оть волненшя, но и отъ отвратительнаго запаха сивухи, ко-
торая исходила оть Ильянтенко, какъ изъ разливитейся бочки,
О

Благодаря этому Хартахай и смогъ сообразить, что насто-
ящй вачальникь не будегь преисполнаться исключительно
столь омерзительнымв питьемъ, кавъ простая сивуха: но
разсужденио Хартахая, вастоящая персона можеть только
для начала отвфдать простой водки, — для отврытя тавъ
сказать, выпивки. а излишеству предается болфе блатород-
ными пимями. 0ъ отимъ справедливымь разсужденемь стряп-
чаго его спутпики вполнЪ сотласились. ВеЪ трое опять
вошли. Хартахай р$шительно заявилъ, что Ильяшенко име-
помъ закона арестовань и что, въ случав сопрстивленя, ва-
питанъ пустить въ дЪло стоявшихъ у дверей солдатъ. Серьез-
ность положен отрезвила Ильяшенка; онъ понядь, что
доведен1е дзла до рукопашной было-бы фатально для его
престижа-и онъ избралъ другой епособъ соиротивленя. „Хо-
рошо“, отвфтиль онъ ворвавшимся въ нему влаетямъ, „я
пойду за вами, только знайте. что я вамъ покажу, вто я;
весь вЪкъ будете плакатьея!“ Ильященко. дЪйствительно, но-
слфдоваль за Хартахаемъ, Поповымъ и Лисенко.

Было болфе 5 часовь пополудни; еклонявтееся къ за-
паду, яркое, весеннее солнце выглянуло изъ за разорвав-
шихея тучъ и мягкимъ, радующимь взоръ свфтомъь залило
невзрачные МарТупольсве домики. грязную немощеную улвцу,
Ильяшенко, Попова, Хартахая, капитана Лисенко и большую
сопровождавшую ихъ толну людей. Вс эти люди направ-
лялись въ присутетые греческаго суда, кула они скоро
и прибыли безъ всякихь инцидентовъ. Отсюда, цервымъ
дЪломъ, предефдатель суда Шоповь отправилъь въ тюрьму
приказъ, который я копирую съ сохраненемъ не только вы-
раженй, но и ороографя. Вотъ этотъ приказъ: „СОмотрителю
Тюремпаго замка. Сейчасъ освободить изъ тюрьмы и ири-
слать въ судь содержащагосе Николая Логафетова. АпрЪль
6 дня 1863 года. ПрелеБдатель К. Поповъ“.

-- АТ —

Согласно этого приказа Логафетовь былъ доставленъ
въ судъ въ томъ самомъ видЪ, въ какомъ утромъ его пре-
проводили въ тюрьму, т. е, въ кандалахъ и арестантекомъ
одфяни.

Логафетовъь продолжаль пребывать въ состоянти угнете-
ня, доведшаго его ло полной апатш; онь молчаль, не вы-
фажаль радости но поводу возвращенной ему свободы, тупо
й медленно озиралея, и только по временамь тяжело со
стономъ вздыхалъ.

Капиталь Лисенко былъ ни живъ, ни мертвъ, иресл$-
дуемый слЗдующими безъисходными лумами: „если Ильяшенко
шафрлатапьъ, капитану не избфжать суда за содфаянное надъ
„Лотафетовымъ; если же Ильятенко таинственный и чрезвы-
чайный начальникъ, какъ въ этомъ онъ продолжаль упорно
увфрать, постоянно твердя: „попомните меня, никото не
забуду“... то совефмъ не трудно угодить въ каторгу, еели
не на зисфлицу“... .

Хартахай, предефхалель, засвдалель и ваводнивиие при-
сутстые суда именитые граждане пылали зл0б0й противъ
Ильяшенко, обиженные посмфян1емъ надъ ихъ роднымъ учреж-
дентемъ и наль вовми ими.

Снявъ съ Лотафетова арестантекое одзян1е, эти разсер-
женные люди со злобой одЪли въ него лжеуполномоченнаго.

Но заключить Ильяшенко въ кандалы греки не поем$ли;
они все-таки по инерци продолжали чувствовать совершенно
ни па чемъ опредЗленномъ не основанный страхъ; вЗроятно,
проето на ихъ впечатлительность дйствовало то обстоятель-

ство, что Ильятенко, не падая духомъ. продолжаль ихъ

разносить и стращать всякими ужасами; а также вЪфроятно
имъ импонировала сохраненная арестованнымъ манера дер-
жать себя съ неукоризнениой величественностью настоящей
переоны.
" — 48 —

Но такъ какъ надо было что нибудь дЪлать еъ опас-
нымь узникомъ, то въ концф концовъ по наставленйо Хар-
тахая, судъ въ полномъ состав и присоедивившаяея толна
именитыхь гражданъ, всБ вм$етБ, повели Ильяшенко въ
тюрьму. Половину нурешестве совертили безпрепятственно.

Только со стороны русекихь поселенцевь, находизтихея
въ толп, было проявлено н®что, похожее на мапифестатю
въ пользу Ильяшенко. Эти руссше люди отерыто увфрали,
что греки влекутъ въ тюрьму Великаго Ёнязя. Конечно.

если-бы тавихъ протестантовъ ‘находилось н\аволько соть

челов къ, шествию не сдобровать; оно было-бы силою оста-
новлепо и узникъ получить бы свободу, но на несчастье
Ильятенко тавихъ русскихь людей не было и десятка; ихъ
протесть не могъь имЪть реальнаго значен!я; на нихъ никто
ке обращаль вниманая. Шестве же, когпа половина пути
была пройдена, простановилось благодаря р8шительному
ДЪИствию одного изъ именитых гражданъ, содёйстворавшаго
властямъ и греческому суду въ арестовавши лжеуполномо-
ченнаго. Подобно доброму коню, получившему шпоры, онъ
стремительно выскочилъ изъ толпы, сталь на ея дорогф и,
растопыривъ руки, во всю глотку заоралъ:

— Стой, стой, стой!...— произнося это слово то по русски,
10 по турецки. Задыхаясь, онъ сталь затВмъ говорить скоро,
крича, размахивая руками и часто отилевываясь. Омысль его
обильныхь сизшно сыпавшихея словъ сводился къ тому, что
всв греви тотовятъ себф Сибирь, каторгу, что веф оли „ду-
раки и бараны“, не могли понять, что они сажаютъ вь
тюрьму не только Ильяшенко, но и царсюй портретъ, ви-
сящий на шев арестованнаго.

Мноче изъ слушавшихь воскликнули, мноме ударяди
себя собственною дланью по лбу, всЪ остановиливь на ми-
нуту вавь вкопанные, и затВмъ все шесте повернуло об-

=
——

49

фалтно въ гречесый судъь Здфеь люди, погаалвь еъ четверть
часа, поспоривъ и въ пфоколькихь отлльныхь елучаяхь
обругавъ другь друга, еняли сь шей Ильшиевко медальонт
<ь портретомъ Государя и оаять повели узника въ тюрьму,
вуда на этотъь разь доставили его благополучно и безпре-
пятственио.:

Так» пончились вачальственвыя похожлены Ильншенко
въ г. Мар!уполв. Какф видить читатель, ови ограничились
однимъ днемь: въ 10 час. утра Ильяшенко объявилея въ
канцелар!н уачальника команды, вь 12 чае. дня свершихся
судъ п исполнене его рфщеня надъ Логафетовыме, между
часомь и двумя состозлея обфдь на манеръь банкета, а къ

5 часам» пополудни Ильяшенко уже былл, ареетовант,

Эпилогь моръ-бы составить новую исторю, открывалюо-
щую одву изъ обычныхь страничеюь дореформеннаго буди,
отживавшатго въ то время свои послвдые дни. Порадвя эти
веБмь хороно извзетны и мы будемъ кратки

Логафетовъ, обрВиь свободу и возвратившись домой,
долгое время оставалел въ раздумьи: обрить-ли ему вторую
половину головы или ждать, пока бритая половина отростеть.
Ньсколько м$фелцевъ ходилъ онъ съ повязаннымъ на голов
плалкомъ. Онъ випль злобой противъ всего состава гре-
ческаго суда — бывшихъ приятелей, и, пользуаеь услутами
какого-то трамотВя, писалъь безкопечныя жалобы на вс
мЪетныя власти и особенно на предс®дателя Нопова и стряи-
чаго Хартахая, усматривая въ дЪйстьяхь этихъ послфднихь
лиць злыя, корыстныя намфрена прохивъ своей личности:
Зе эти жалобы; несмотря на многослоне, заключали! толь-
ко одну улику’ противь обвиняемыхъ „Логафетовымь динъ,

состоявшую въ томъ, что Ильяшечйо быль въ. проотомт,
|

Е т

старомъ пальто и весь наружвый видь его ни мало не со-
отвЗтетвоваль настоящему начальнику

Такая шаткая улика оказалась, понятно, недостаточной
и жалобы Лотафетова по бездоказательности, оставлены безъ
посл детвий.

Но хуже всего для жалобщика было то обстоятельство, что,
поразившее его въ моменть приведеня надъ ним приговора
въ исполнене, угнетенное состояне духа ме только не 0с-
лаблялось, но съ каждымъ двемъ усиливалоеь. Овъ осунулея,
избфгаль людей, потерялъь апиетитъ, словом быстро шелъ
на убыль. Протявувъ въ такомъ видф года ©ъ полтора, Ло-
гафетовт' умеръ.

"Тазъ относительно него, еще въ земныхь условнхъ,

исполнилел заковъ вовдаян1я зломъ за зло.

Капитанъ Лисенко, Хартахай, Поповъ и члены гречес-
каго суда около двухь лВтЪ пребывали въ нензреченномъ
страх. Кром Лиеенко, вс остальныя лица, ища спасены
отъ отвфаственности за содфянное, обнаружили не особенно
высокую культуру душевныхъ свойствъ. Поповъ, Хартахай
и вообще весь составъ греческаго суда прилагази всё уси-
эн, чтобы вызвать полозрн!е, будто капитавъ Лисенко на-
ходился въ предварительномъ заговор съ Илзъятенко.

Ол$ды такихь злокозненныхь дзйстый сохранились въ
н%которыхь докумевтахъ.

Такъ, напр., въ журналв „входящим и исходящимъ
мартупольской команды па 1865 ‘тодъ“ есть ‘указав!е, что
уже 229 апрфля мазупольскй гречесый судь отнотешемъ
за № 1598 новарно просиль валитана Лисенко увдомить,
на какомь основани начальникь команды прибыль съ людь-
ми въ сей судъ 6 азирфаля для арестованя мЫщанина Ло-
тафетова”.
и

а

Капитанъ Лисенко не кривилъь дущою и отношенемъ
отъ 95 апр$ля, за № 207, напрямикь отвЗчаль, „что при-
быме начальника команды ©ъ конвойными людьми въ нри-
сутете сего суда 6 аирЗля состоялось по полученному
приказу, подписанному именующимь себя полномочнымъ
Государя Императора — Ильяшенко“.

Затьмъ, въ судебномъ дл есть особый слфдетвенный
акть отъ 14 сентября 1863 года, возникпий „по поволу
неодновратныхь есылокъ предс®дателя Попога и Хартахая
на 10 обстоятельство, 910 нев\роятно, чтобы лжеупономо-
ченный Ильяшенко не имзль свидамя съ капитаномь Ли-
сенко ло прибымтя въ гречесый судъ“.

Но всЪ эги хитрые маневры, ‘долженствовавийе напра-
вить слфделые въ желательному для Попова и Хартахая
направлени не увЪфнчались успфхомъ: слфдетые воочию до-
кавало, что вапитанъ Лисенво все время дЪИствоваль, какъ
бравый солдалъ, по совфети, Бопа Нае.

Добролвтель не пострадала и капитанъ Лисенко пе
подвергся никакому преслёлованю. Года черезь три онъ
совсфыт оевободилея отъ страха отвфтетвенности за содвян-
пос падъ Логафетовымъ. Когда ему приходилось разсказывать
о знаменательномъ днф 6 апрфля, онъ всегда и неизмВнно
внадалъ въ энтумазмъ. Н%сволько лфтъ тому назадь, въ
тлубокой старости, вацитанъ скончался. |

Немного сложн%е и длиннЪе должно быть повфетвова-
ве о дальнфйшей судьбв Ильяшенко,

Въ то время, кашъ вс участники настоящей эпоцен
иребывали въ страх% и трепет, боясь отв®тетвенности, одинъ
Ильяшенко, котораго судьба висфла на волоскВ въ течене
почти трехъ лёть, не падаль лухомъ. Заключенный вЪ ма-
рупольскую тюрьму, онъ сохранидж такой начальственный

видъ. держаль себя такъ свысока, то порицая, то одобряя,
4=
`

2 —

й
1
1

то лагражлая обружающихь будущими объщалями, что не
только низние чины тюремной `адмивиетразии. но и самъ
смотритель тюрьмы подвергся дъйетьню превеликаго сомиЪя
и смущеня. Кавъ скажеть бывало Ильяшенко съ величеет-
венной ‘ввушительностью: „Нопомнини, меня! увидишь скоро,
одобряю, награжу“, — то и иойдеть въ тюрьмв топот,
отчего и пронеходило смятене‘ тюремныхь начальственныхь
душь. Омущало везхьъ. что такого преступвика фацьше пе
бывало, и боэтому неудивительно, если вов ованчивали свой
яшопогь заключешемъ: „нфть, туть что-то не спроста“.

Этому сматенио душь содфйствовало и то обстоятельство,
что внЪ тюрьмы разросталея, кавъ вкругъ оть брошеннаго
въ воду камня, пущенный нфеволькими русскими аюльми
незёный слухъ, будто Ильяшенко ни ‘вто иной, какъ Великий
Цнязь. Авторы этого елуха основывали догадку па томъ,
палавтиемея мяогимь убфлительномъ сообращенш, что ни у
вого на шсЪ не могло быть портрета Государя, какъ только
у Великаго Князя — брата Государя.

Кань бы то ни было, черезъ пЪекольво дней, по заклю-
ченш Ильяшенко въ марупольскую тюрьму, не было въ
город тапого человфка, который бы ве слышалуь объяснения,
что греки посадили въ тюрьму Великаго Куязя. Этот слухъ
не замедлиль выйдти изъ предфловь города Маруполя, и,
окрфиши, сталь распростравяться по окрестнымь селамъ,
деревнамт, и даже проникъ въ сосфдые города.

ДостовВрно извЪфетно, что весною 1563 тода, въ сель
Макеимимановв, отстоящемъ оть города Малуноля боле
чфмъ на ето верст. сельсый сходъ, собравшись въ присут-
ств старосты, имль совзщан!е о своихЪ нуждахъ и; „между
прочимъ. имфль суждеше о томъ, чго господа помфщики
посадили въ тюрьму Врата Государя, а потому, поговоря
между собою, постановили“...

*

ыы

— 58 —

Тавъ записаваль малограмотный сельсый писарь сущ-
ность того, что составляло предметь разсуждетй на сходё.
По обыкновеню, онъ писалъ постановлеше схода, посляЪ того,
какъ этоть нослВднШ разотелея. Писарь мало думалъ о томъ,
что хотВль ностановить сельск!й парламентъ; по опыту отъ
хорошо зпалъ, чго вее имъ написанное сойдеть за постано-
влеше схода; для этого вфдь стоило ему только вписать
фамими веграмотныхь крестьянъ села и зат6мъ призвать
одного грамотнаго крестьянина и предложить вму роснисаться
за себя и 3% неграмотных; на практик викогла не было
примфра отказа въ учинени такого пустлка, какъ росин-
саться, хотя нн одинъ изъ руконрикладчиковь пикогда не
интересовалея звать, въ чемь ©06т01л0 ностановленте схода
и вообще, что онъ подпиеывалъ.

(исаря затрудняло другое обетоятельство: онъ никакъ
не могь сообразить, что писать поелз слова: „постановили“.
Во всвхь приговорахь. написанныхь имъ раньше, нослб
этого слова писали: раепредфлить землю, нанять пастуха,
обложить платежем и т. д. Никайя тавут слова въ лавномъ
елучаВ не подходили. Писарь тщетно морщиль чело, вер-
тВлея па табуреть, погружалу сусиное перовъ чернильницу,
Ничего не выходило

Въ концф копцовъ, писарь скомкаль Въ рукЪ пачатое
имъ постановлене сельекато схода.

Косноеть деревни препятствовала таинственнымь сау-
хамъ, которымъ вов безусловно вЪрили, вызвать какой нибуль
значительный эффекть. Люди удивлялись коварству помВщи-
ков (которыхъ. въ слову сказать, въ мартупольскомь уфздё
теперь совебмъ нЪттъ, а въ то время было не бол8е 5 или
6 человВкъ), и затбыъ, пошум$ въ иногда на сеходЪф, входили
в спокойную колею ‘деревенской жизни, не предпринимая
вичего дъйствительнаго и цвлесообразнаго относительно в9з-
мутившихь ихЪъ покой собыий.

ПБ: Ш ад

+

3 — 954 —

Но зато ма) польсмя дамы, почуявь таанствениое,
восиламенились къ Ильяшенко и старались поддержаль его
репутацию. Он потянули въ тюрьму кавъ на ‘богомолье.
Благо въ то время никакихъ. тюремныхь стротостей и фор-
мализма пе практиковалось: веяый кто хотфлъ могъ, безъ
веякато разрёщешя, посфщать тюрьму, котда ему было угодно
и сколько разъ ему было угодно. -

Боюсь, что неловфрчивый читатель сейчасъ спросить:
кажъ-же, при такомъ отсутстым надзора и строгостей, вс
арестанты не убЪфгали изъ тюрьмы. Замзчу на ато попро-
шелощему, что и я ставиль старожиламь такой же вопрост;
они отвьзчали мн, что не полагалось бЪгать и что въ то
время эрестантовь бфжало не больше, чВмыъ теперь. Всякий,
молъ, зпалъ, что начальство не дозволяеть бЪжаль.

Благодаря такимъ удобнымь услойямъ для посфщеня
тюрьмы, дамы стали самымъь щедрымъ образомъ наносить
визиты узниву. Ильяшенко ихъ принималъ, ни па минуту

. не измфняя своей пачальственной величественности, & награ-

диль многихь изь посзтительниць обфщан!емъ, что онъ ихъ
въ надежное время ВСПОМНИТЬ.

Одинъ изъ достов®риЪйшихь старожиловь объясниль
мнЪ, что дамы въ общемь дфалф ухаживанья за Ильяшенко
не ссорились между собою, ибо въ это ухаживаие ромьни-
чесвя чувства не входили. Даже, напротивъ, проявляли са-
мую усердную, солидарную заботливость объ Ильяшеако,
онф съ общаго вовхъв сотлаея, ввели принципь раздфленя
труда: барыни изъ у6зда узлами привозили для питант уз-
ника фрукты, овощи, маело, молоко и всякую живность,
городев1я-же дамы” ежедневно приносили въ тюрьму горяще
пирожки, обфды, сладости и т. п. Но особенно трогательно
выразилась заботливоеь дань въ доставленыи узнику всякаго
потребнато бфлья и костюмовъ. Въ этомъ отноптеня щедрость
Е

й
5
я

ихъ была такъ велика, что по свилфтелеству старожиловъ,
Ильяшенко, каждый разъ, вызываемый къ слфдователю нА
допроеь, являлся въ новомъ вкостюмЪ.

Паломничество дамъ въ тюрьму не особенно импони-
ровало тюремной администрати и начальнику тюрьмы. Но-
слъдшй смотржль на дамъ, вакь на народъ несерьевный,
большей частью не понимают того, что дВлаеть. Но зало
посъщен1е разныхь неизвфетныхь гоеподъ, прЁзжавшихь
издалека и усердно стремившихся и добивающихся азумении
у заключеннаго, продолжавшаго себя держать ©ъ величест-
венностью персоны наполняло ядомъ тревогь и сомнфый
бЪдную душу начальника тюрьмы.

_ Окончательно его добиль ныЫй полковникь и нёкй

`тенераль, посБтивиие Ильяшенко въ оцинъ и тотъ же день.

НолковниьЪ, послБ аущенщи, быстро вышель, вотрахнувъ
везиь тБломъ и быетро ушелъ, бросивъ одно только слово:
„удивительно“. Гонералъ-же на обращенный къ нему во-
просъ: „вакъ полагаете ваше превосходительство?“, отвфтилъь
только „поразительно... не сироста“... и, усФвшись въ свою
коласпу, запряженную тройкой добрыхъ донскихъ коней, ука-
тиль въ свое имзыюе на донскую сторопу.

Поелв этого начальникъ тюрьмы вналь въ такое безио-
койство, вакъ будто ожидаль, что ежесекундно налъь его
тюрьмой разразится небесный огонь. Мимо камеры ИльязшенЕо
онъ не ходиль иначе, какъ дрожа веБмь т$ломъ и на пы-
почках. Въ его голов зарождалась даже мыель, что не
лучше ли будетъ, если онъ явитея въ камеру узника и по-
вергнегь, къ его ногамъ во ключи тюрьмы. Во веякомъ
случаЪ съ каядымь днемъ пазальникъ тюрьмы все боле
поднадаль подъ воздфйстые укрБплявшихся слуховъ, полу-
чавшихь силу абсолютной достовфрности и создавшнхь Иль-
яшенко ореоль таинетвениаго зеличя. (Юращене за совЪ-
томъ кф овружеющимт, вабъ всстда въ такихь случаяхъ,
было безполезно. Дюли говорили много, умно, а въ итогв
дарили совсВмъ безплодпый совфть: будьте осторожны, емо-
трите въ оба, поступайте, какъ лучше... Оть столь рЪши-
тельнато шага, капт полнесене ключей тюрьмы, начальникъ
этой посяФдней былъ удержанъ только благодаря тому обето-
ятельству, что изъ Екалтеринослава прибыль произвести елвд-
стые особый чиновникъ губернатора, командированный своймъ
начальствоимъ съ этой спешальной пфлью.

По мЪрф тото, накь разросталось сядет, тюремная
алминистратя успокаивалась,

Идльятенко-же не издаль духомъ и на допроеЪ нпро-

полжалъ держать себя по прежнему, кавъ начальникъ. На
первомъ допроеВ онъ даже смутиль слблователя, когда на
предложене объяснить. чёмъ онъ занимался, отьфтиль: „р
нахожусь по службф, някому объяснить ве желаю, вром®
Госуларю Императору“. Но слЁдователь быль изъ опытныхь
и ве поддался первому впечатльню. Онъ -осторожно соби-
ралъ сВдетвенный матераль, который оказался убШетвеннымъ
для разыгрываемаго Ильяшенко фарса. Тфмъ не менфе это
слЗдстые не дфйствовало угнетающе па пашего героя. Бла-
тодаря заботлизости дамъ, опъ, за времз пребывав въ ма-
ртупольской тюрьм%, откормился, поздоровЪль и, прилично
одфтый, сталъ болфе походить на персону, ч5мъ въ тотъ
моменту, когда онъ въ длинныхь емазныхь сапотахъ и ста-
ренькомъ коричневомь потертомь пальто, впервые явился въ
канцелярю вачальника марупольской команды. Тавимъ 00-
разомъ, въ благополучномъ еостояни Ильяшенко оставалея
въ Маруполь до осепи 1863 года; затЪмъ его перевели въ
Алекеандровекую тюрьму, гдВ онъ сразу попалъ въ тяжелую
обстановку. Въ этой тюрьмф на него не обрацали внимания,
в онъ очутился в положени обыкновеннаго рядового аре-

—=

станта. Улруленный холенмъ пребывамемь въ оТюрьыт, Иль-
Ященко ухнатилея за послвВднее средетво; онъ началъ пиезль
длинныя прошеня ва Высочайшее вмя и въ Юкатеринослав-
скую палату уголовнато суда. Въ эхихь прошевях» онъ
жаловался на свое’ положене въ тюрьмЪ, на свое „неогра-
ниченное жертвоприношен!“ въ служеши государственнимь
цфлямъ, но ничего не помогало; судебрал воловита тянулась
евоимъ обычнымъ ходомь и только 96 мая 1964 гола
‚алевсандровеюмй уБзлный судъ обще съ тородевой ратушей“,
въ качеству сула первой инетанши, приступил въ раземо-
трёню лфла объ Ильяшенко и ето сообщнякахь: купиЪ
Поддубнв, МазинЪ, Волосеовскомъ и Пичахчи,

Кажь мы знаемъ, первые три изъ названных лиць
вотрьтилась еъ Ильященко въ №. Бердянск в два изъ нихъ
сопровождали ето въ Мартноть. Что касается Пичахчи, то
слЬлетыемь было установаено, что Ильяшенко его постав
наканунЪ своихь знамевательныхь дйстй въ г. Мартулоав.
По отеутетвно добстаточныхь уликъ, на основан которыхь
можно было-бы установить евязь въ дфйслыяхь Ильяшенко
0 вефми названными лицами, уЪздный сулъ въ евоемь вер-
диктз постановить означенныхъ лиць, .прихосновенаыми къ
сему двлу пе читать“. (СЛ. д. 42 тамъ-же) Что же васа-
етея Ильяшенко, то судъ, признавь его яВйствовазшимт въ
добромъ .здрайи и виновным, полвель его лВйетия подъ
большое количество твхъ статей изъ разнахь попцовъ бозь-
того ХУ т. св. завоновъ, по перечислении конхъ, слЗховало
роковое ваключене: & посему опредЪлиль; по лишении вофхь
правъ состоян!я, гослать Ильяшеняо въ каторжныя работы
безь срока. |

По счастью Ильяшенко рфшеше сула не было оконча-
тельно, и онъ воспользовайея своимъ правоуъ перенести
ДЪло. во 2-ю инстанщю, въ Икатеринославскую палату уго-

== 58. Не

зовнаго суда. Это судебное учреждев!е отнеслось къ участи
Ильяшенко ‘болфе гуманно: ‘оно обратило внимане на его
„неограниченное пертвоприношене“, на отсутегне корыстной
пли въ его дфйстыяхь, быть можеть также на его роль
Пемезиды, воздающей каждому по двламъ его, и подняло
вопроеъ о нормальности его душевной дфятельности въ мо-
менть совершеня преступленя. Правда, что ототъ воир®сь
не особенно уб®дительно согласовалея со везми обстоятель-
ствами лЪла, что и заставило Уфздный судъ отвергнуть пред-
положене о ненормальности подсудимаго; тфмъ (не менЪе,
при старомъ формальномь суд, таковъ быль единственный
путь для спаеевя Ильяшенко. Заря новыхь вфавй, нред-
шествовавшая ввеленшо суда присяжныхь и подготовлявшая
почву для правосумя, основывающагося на принцио сво-
бодной совфети, внутренвяго уб®еденшя, правосущя, судящаго
не только дВяшя, но и самаго преступвива — эта заря вЪ-
роятно бросила свой зарождающийся съЪтЪ и въ отживавиия
послёдше дни старыя судебныя учрежденя. Формализиъ въ
въ дВИствительности слабЪль, и это принесло спасене Иль-
яшенко. Еказеринославская палата, уголовнаго суда отнеслась
въ преступленлямь Ильяшенко не только съ точки зрашя
удовлетвореня капцелярсвихь, формальных» условй; она
тщательно разобрала внутреннюю сторону дфянЙ подеудимаго
и нашла, что Ильяшенко дЪйствоваль не въ здравомь ум,
а потому ршенемъ, состоявшимся 2 августа 1865 года
опреявлила: преступленя совершенныя отетавнымъ чертеж“
викомЪ Ильяшенко не вифнять ему въ виву.
Освобождеше Ильяшенко отъ всякаго уголовнаго нака-
зан!а вызвало приливъ бЪфшенства у стряпчаго Хартахая.
Его раздражала мысль, что судебное учреждене косвенно
признало, что не только всё Мартупольсыя власти, но и
онъ, Хартахай, стряпч, стражь закона, принимать сума-
50

сшедшаго за высокопоставленную персону. Воть почему,
воспользовавшись своими нрокурорекими правами, какъ стряц-
Шй, Хартахай разразился жалобой въ сенать на рёшеще
Ккатеринославской палаты уголовнаго суда. На многихь
листахъ дохазываль онъ правительствующему сенату, что у
Ильяшенко умь самый здравый, что всВ его поступки въ
МарулолВ верхь тфлесообразности и разумности, что едва
оправданный Ильяшенко тотчасъь же получилъ мЪ$ето съ го-
довымъ оБладомъ въ 500 руб. вь Екатеринославской гоерод-
ской дум ит. д. По веф эти доводы оказались тщетнями:
сенатъ оставиль протесть Хартахая безь уваженя, удержавъ
съ него въ пользу казны 3 руб. 60 кон. пошлинъ за не-
правильную жалобу. (Л. д. 156—158 тамъ же). До рёшешя
сенала Хартахай очень горячо выражаль свое негодоваше
противь рЬшешя палаты и всакаго встр$чнаго обдавалъ
фразой: „помилуйте, какой же онъ сумасшедиий, развЗ бы
я могь исполнять приказан!я сумаешедшего“ |... Увфдомден-
ный о рёшен1и сената, Хартахай примолкт, и только отиле-
вывалея, когда заходила рзчь объ Ильяшенко.

Но стряпчаго и весь составь суда ожидало еще боль-
шее огорчеве. Въ ннварЪ 1869 года утверждено опредЗлеше
палаты тавого содержаня: „Бывшимь членамъ мартупольскаго
греческаго суда: Попову, Газавджи, Ганжи, Охсюзову н
секрегарю Хартахаю виновнымъ въ бездВйстви власти сдЪ-
лать на основан 343 ст. улож. о нак. замВчане“.

„Воть такъ правда“, говорили осужденные. „ВмЪсто то-
го, чтобы пожалфть людей, надъ которыми потлумилея, наемял-
ел шарлалант, яхъ обвиняють въ престунленш но должности“!

Вфролтно, нвкоторыя лица спросять мена, зачЪмъ я
возстановиль всю эту маловфролтную, фантастическую исто-
— 60 —

ро, въ тБйствитольность которой викте бы не новфрилъ,
если бы Л въ своемъ изложети не опирался па безспорный
доказательства. ОтвЪчу кратко. Мвв хотЪзось этой истомей
показате прежде веего, наскозько городъ Мамуполь. пазуз
чивиий въ семидесятихь годахъ городегос самоунравлене
на началахь не исключительности, по равенства вебхв ио-
’редь закономъ, подвинулея вперещь въ своемь развийи. Че
только въ настояние дци, но уже въ конц® семидесятыхт
толовт, исторя Ильяшенко казалась ма\упольцамт гакимт-то
чудовииинамь сномъ; до тазой степени яваялосью бы совер-
нение невозможнымт повгорене чего либо полобнаго тому,
что продфлаль Ильяшенио: Но отмфчая такого рода прор-
реесъ, д въ тоже время хочу напомнить, что и по лынЪ въ
мартупольсвом» увздф, какъ и во’ многихь другихь уфзлахь,
можно вотрутить большия и малыя села, гдВ свободно и се-
годвя можно повторить оныть Ильяшенко, продзланвый им
31 зЪтъ тому назадт. Мн лично извФетенъ случай, когда
одинъ тосподинъ, получивь въ дарь отъ казны амфие за
свою службу и цоселивиись, но зыходв въ отставку, на
подаренной землЪ, заблагоразеудилт присвоить себ неогра-
ниченную власть въ смысл коптроля и управлемя дВлами
ближайших, сельскато и волостного правлений. Это случилось
за нфеколько ЛЁтЪ до введешя института земекихъ началь-
никовтЪ. Вновь поязвивнийся землевладлець, опиразеь на
газетные слухи © будущей роли земскихъ начальников и
ъесьма опгибочно объявивъ, что онъ именпо получить наз-
ванную должность, фактически присвоиль себф тажя права,
вавихъ не получили и будуше представители судебно-адий-
нистративной власти. Тавъ управлён1е этого начальника-во-
лонтера блатополучно продолжалось два года, до тВхь поръ,
лова не явился настояпий земскй начизьникъ, упразднивиий

его фактическую, въ предБлахь волости, государственную

28

ке

ео

=
а

..

р

дфательность. Нродфлка Ильяшенко и веб подобрыя невф-
роятныя шутки вытекают изъ одного общаго источника:
изъ полнаго отсутотыя въ нашемь обществ элемевтарны хи
прелставленай о дЪйствующихь учрежденяхь, законахъ пра-
вахь и обязанностаяхъ,

Но этоть факть я равьше подробно указывал в» с6о-

ихъ относащихел &ь этому предмету реботахт: равным

образомъ, я ущазываль на удачную полытку вфкоторыхь за-
падно-евролейскихь тосударетвь ввести. Элементарное озпако-
ылеше съ существующимь государствевнымь и юридическим
строемъ страны въ начальныхь пародныхь училищахь. Тогда
ще л объясниль, что настала пора осуществить ту же пр-
пытку и среди нашего крестьянскаго населеня. (П. В. На-
менсюй. Преподаваюе граждапекой морали вь народныхъ
школах. 1896 г. Харьковъ).

Наконецт. возстановляя историю Ильяшенко я считалъ,
что этоть маленьюмй опизодь интересенъ тВыъ, что онъ яв-
ляется прямымъ результатомь тёхъ вфян!Й, благодаря кото-
рымъ держалось убЪждеше, что для блага общежимя пужпы
правители, знающ!е тольхо исполнительность, дисциплину и
проникоцеся сознаюемъ, что опий „не могутъ смЪть свое
суждене ныть“,

Мысль, что оргавы Власта должны ‘быть, хотя бы въ
скромпой мЪрБ, проевьщенными людьми еъ устойчивыми мо-
рально-правовыми представлемями считалась не только заслу-
живающей внимания, но иросто вольнодумствомъ. Помнауйте,
говорили в® то время, (аи тенерь подъ часъ повторяютРь
таъ называемые ревните\н порядка). вочему эти громым
слова, и 60635 нихъ люди жили, — главное, чтобы чиновники
дфао дфаали и прежде него въ точности исполняли прика-
зая начальства безь хитростныхь размышиленй; тогда ната

общественная. жизнь быстро освободится оть тяготбющихь

Ето ==

надъ ней недостатковъ. Сердцу сторонниковъ такого взгляда,
долженъ быть очень любезенъ капитанъ Лисенко. Онъ вЪль
такъ былъ выдрессированъ своей эпохой, что только и жаж-
даль исполнять, не разеуждая. А вВль члохо могло придтиеь
мартупольевимъ обызателямъ, если бы Ильяшенко, ошал®вити
отъ удачнаго исполневя своей роли, приказалу, писциплини-
ровапному капитану колоть или стр®лять греческй судьи
именитыхь граждан. Бапитанъ Лисенво не замедлиль бы
показать столь желанную исполнительность; не даромь ка-
питанъ въ своихъ позднфйшихъ признаняхь соржествевно
и съ паеосомъ объявиль: „прикажи Ильяшенко стр®лятЕ,
веЪхъ-бы перестрфлялъ, приважи колоть, вофхь переко-
лолъ-бы“. Хорошо, что дфло ограничилесь лишь бритьему
головы.

и №

„Ликар - стоматолог

208
Ярнтия Эт

^я
