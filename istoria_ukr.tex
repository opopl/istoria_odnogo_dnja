
%title
%page1

\documentclass[a4paper,20pt]{report}


\usepackage{geometry}
%\usepackage[a4paper, total={6in, 8in}]{geometry}

\geometry{
 a4paper,
 total={170mm,257mm},
 left=20mm,
 top=20mm,
}

%\documentclass{article}
%\usepackage{iftex}
%\iftutex
%% For LuaTeX or XeTeX Use Google's 
%% OpenType Noto fonts for typesetting
%% Russian text
%\usepackage{fontspec}
%\defaultfontfeatures{Ligatures={TeX}}
%%\setmainfont{Noto Serif}
%%\setsansfont{Noto Sans}
%%\setmonofont{Noto Sans Mono}
%\else
%% For pdfTeX we must use old
%% 8-bit font technologies
%\usepackage[T2A]{fontenc}
%\fi
%Hyphenation rules
%\usepackage{hyphenat}
%\hyphenation{ма-те-ма-ти-ка вос-ста-нав-ли-вать}
\usepackage[english, russian, ukrainian]{babel}
\usepackage{csquotes}
\usepackage{xfrac}

\usepackage{fontspec}
\setmainfont{FreeSerif}
\setsansfont{FreeSans}
\setmonofont{FreeMono}

\usepackage{polyglossia}
\usepackage{hyperref}

%\setdefaultlanguage{russian}
\setdefaultlanguage{ukrainian}

%\setotherlanguages{ukrainian}
\setotherlanguages{russian}

%\babelprovide[import]{thai}
\babelfont{rm}{FreeSerif}
\babelfont{sf}{FreeSans}
\babelfont{tt}{FreeMono}

\newif\ifrus
\newif\ifukr
\newif\ifpages

\ukrtrue
\pagestrue



\begin{document}
\Large

П. В. Каменський

Історія Одного Дня

Достовірна Розповідь

Лікар-Стоматолог ВДОВ

Валентин Дмитрович

Єкатеринослав

Типо-літографія М. С. Копилова. Проспект, власний будинок

1900
%page2
Дозволено цензурою. Харьков, 25 сентября 1900 г.

%page3
\ifpages\section{Сторінка 3}\else\fi
Історія Одного Дня 

(Достовірна Розповідь)

Біля 20-х чисел березня 1868 року у місті Бердянську, у готелі ``Білий Лебідь''
орехівський третьої гільдії купець Поддубня, який торгував чаєм, дьогтем та
салом, розповідав двум своїм знайомим, Мазіну та Колосовському, про зловісні
пригоди, які стались із ним у місті Маріуполь. Серед слухачів був також третій
чоловік, невідомий Поддубні, що також слухав розповідь; його звали \textbf{\em
Григорій Власов Ільяшенко}. Він був молодий чоловік років 33, блондин,
зовнішньо нічим особливо собою не помітний, і ніяких, як мовиться у паспортах,
особливих прикмет не мавший.  Родом він був із міста Миколаїв: у час коли
трапилась ця історія він жив разов із своєю жінкою в місті Бердянськ, де
займався приватним чином креслярськими роботами у місцевого архітектора. До
того як він прибув у Бердянськ Ільяшенко знаходився на службі в
севастопольській інженерній команді морської будівельної частини як кресляр та
був нагороджений, як це було підтверджено офіційними документами, бронзовою
медаллю на Андріївскій ленті.

Розповідь Поддубні не була зв'язною. Він почав із пояснення, у чому полягало у
місті Маріуполь особливе грецьке управління, але його пояснення були мало
зрозумілі, бо вони зводились до повторення одних і тих же слів: каторжні
грецькі порядки, прокляте грецьке царство тощо. Зрозуміло, що ці повторювані
фрази, які переплітались із лайкою, ніяк не знайомили із установами, яким
підпорядковувалось у той час грецьке населення. Насправді, ознайомлення із цими
установами не є складною справою.

????  ????  ????  ????  ????  ????

%page5
Так як у склад суду потрапляли більшою мірою люди, які не тільки не мали адміністративного та суддівського досвіду,
але які були недостатньо освічені, поняття яких не йшли далі питань про ``паляницю та льон'', то, для навернення
їх на шлях законності та відповідного усвідомлення цих шляхів, влада призначала та надсилала із губернського міста 
Єкатеринослава спеціально до цього призначеного секретаря, який ще називався стряпчим та який мав відповідні
прокурорські права.
%Так как в состав суда попадали большей частью люди, не только не блиставшие административным и судебным опытом,
%но даже недостаточно грамотные, понятия которых не шли дальше вопросов о ``паленице и льне'', то,
%для наставления их на путь законности и достодолжного уразумения сих путей, правительством назначался
%и присылался из губернского города Екатеринослава нарочито к сему приспособленный секретарь, именовавшийся
%еще стряпчим и обладавший прокурорскими правами.

Із подальшої розмови, що відбулася між відвідувачами готелю ``Білий Лебідь'', можна було
дізнатись, що в кінці 50-х і на початку 60-х років членом грецького суду був якийсь
житель міста Маріуполь \textbf{\em Логафетов}; він відповідав за поліцейську частину, і був,
таким чином, адміністратором, в обов'язки якого входила задача поліпшення стану маріупольського
грецького округу та його жителів.
%Из дальнейшего разговора, происходившего между посетителями гостиницы ``Белого
%Лебедя'', можно было узнать, что в конце 50-х и начале 60-х годов членом
%греческого суда состоял некий обыватель гор. Мариуполя, Логафетов; он ведал
%полицейскую часть, был, таким образом, администратором, имевшим своей задачей
%споспешествовать благоустроению мариупольского греческого округа и его
%обывателей. 
В цій своїй ролі він не залишив нічого пам'ятного своїм нащадкам. Але як поліцейський чин,
який керував обходами під час ярмарок у Маріуполі, Логафетов став широко відомим; 
чутки пов'язували його ім'я із цілим рядом невеликих та більш суттєвих злочинів.
%В этакой своей роли он ничего памятного потомству не оставил. Но за
%то, как полицейский чин, руководивший обходами во время ярмарок в Мариуполе,
%Логафетов приобрел широкую известность; молва соединяла его имя с целым рядом
%мелких и крупных преступлений. 
Славнозвісні логафетовські обходи, які надовго збереглися у пам'яті нащадків, зазвичай полягали
у наступному: під своїм керівництвом Логафетов збирав загін із сторожів грецького суду і тих молодих
людей із міста, які під час ярмарок приходили в його загін у якості волонтерів. Із таким от загонов 
Логафетов і відправлявся на ярмарок ``наглядати за порядком''. Але насправді вся ця банда віддавалась
розпусті, гульбощам та обжерливості, але що було добре, у ярмарочних шинках та трактирах цих охоронителів
порядку безоплатно годували та поїли, пам'ятаючи про
%Знаменитые логафетовские обходы, надолго
%сохранившиеся в памяти потомства, обычно состояли в следующем: под своим
%начальством Логафетов составлял отряд из сторожей греческого суда и тех молодых
%людей из города, которые во время ярмарок поступали в его отряд в качестве
%волонтеров. С таким отрядом, заседатель греческого суда Логафетов отправлялся
%на ярмарку ``смотреть за порядком''. На самом же деле вся эта банда предавалась
%разврату, кутежам и обжорству, благо, в ярмарочных кабаках и трактирах этих
%охранителей порядка безвозмездно кормили и поили, в виду
%page6
їх значення як начальників. Але шукати розправи і суда шукати було ніде, по дальності відстані
від високого начальства, губернатора і по безплодності результатів розслідувань, які виконувались
надсилаємими чиновниками-ревізорами.
%их начальственного значения. Расправы же и суда искать
%было негде, по дальности расстояния от высшего начальства, губернатора и по безплодности
%результатов расследований, производимых присылавшимися чиновниками-ревизорами.
Одного разу, під час такого начальствованія на ярмарку із своїм обходом, Логафетов
зустрів козака Капсуленко, який різко відповів на вимогу не курити. Відбулась свалка; 
Капсуленко впав, отримавши смертельну рану у бік, яка була нанесена йому дротиком, який
Логафетов носив замість палки. Хоча смертельний удар і не був нанесений особисто
Логафетовим, але одним із його поплічників, тим не менш, у місті Маріуполь усі передавали
за достовірне, що засідатель - вбивця козака, що розслідування по справі про цей злочин
не дало ніяких результатів, внаслідок того, що Логафетов встиг схилити людей свого обходу до дачі
брехливих свідчень, які його обілили. Чутки вказували на виверткість Логафетова перед старою слідчою
владою та додавали, що Бог покарав деяких нечестивих, які брехливо свідчили після присяги; запевняли, що
лжесвідки були уражені раптовою смертю зараз, після дачі хибних свідчень. 
%Однажды, во время такого начальствования на ярмарке со своим обходом, Логафетов повстречал казака
%Капсуленко, который резко ответил на требование не курить.
%Произошла свалка; Капсуленко пал, получив смертельную рану в бок, нанесенную ему дротиком,
%который Логафетов носил вместо палки. Хотя смертельный удар и не был нанесен лично
%Логафетовым, а одним из его споспешников, тем не менее, в гор. Мариуполе
%все передавали за достоверное, что заседатель - убийца казака
%расследование по делу об этом преступлении не дало никаких результатов, вследствие того, что
%Логафетов успел склонить людей своего обхода к даче ложных показаний, которыми его обелили.
%Молва указывала на изворотливость Логафетова перед старой следственной властью и добавляла, что Бог
%покарал некоторых нечестивцев, лжесвидетельствовавших после присяги: уверяли, будто
%лжесвидетели были поражены внезапной смертью сейчас,
%после дачи ложных показаний.
Про всі ці викладені обставини Поддубня розповідав своїм слухачам. Одночасно
передав він і їм і те, що особисто над ним проробив Логафетов у 1861 році. ``Приїхав
я, значить, із товаром на ярмарок. - каже Поддубня, - розторгувався на перший сорт, спустив усякого
товару. Отримав чистоганом на гроші 5,015 рублів, сховав гроші під жилетку і, звичайно, напився п'яним, почав 
вимагати музику''...
%Обо всех этих изложенных обстоятельствах рассказывал Поддубня своим слушателям. 
%Одновременно передал он и им и то, что лично над ним проделал Логафетов в 1861 году.
%``Приехал я, значит, с товаром на ярмарку. - 
%говорит Поддубня, — расторговался на первый сорт,
%спустил всякого товару. Выручил чистоганом на деньги
%5,015 рублей, запрятал деньги под жилетку и, конечно,
%напился пьян, потребовал музыку''...

%page7
Коли в забрудненому сміттям ярмарковому трактирі три обірваних музиканта, які
були озброєні двума інструментами, які були схожі на скрипки та бубном, тішили
слух п'яного Поддубні, у іншому кутку того же трактиру, Логафетов із своїм
загоном віддавався даровому обжерству та пияцтву. Затуманілий надмірної
кількістю їжі та надмірно випитим, Логафетов почав бешкетувати та вимагати,
щоби музику, яка тішила Поддубню, негайно почали би грати перед ним. І як би не
був п'яний Поддубня, але він виступив захисником своїх прав, кажучи, що він
заплатив за музику наперед, і що він не відпустить, ``аж поки вона не відіграє
свого''.

%Когда в загрязненном сором ярмарочном трактире три оборванных музыканта,
%вооруженных двумя иструментами, похожими на скрипки и бубном, услаждали пьяный
%слух пьяного Поддубни, в другом углу того же трактира, Логафетов со своим
%отрядом предавался даровому обжорству и пьянству. Отуманенный чрезмёрной едой и
%излишне выпитым, Логафетов стал буйствовать и потребовал, чтобы музыка,
%услаждавшая Поддубню, немедленно стала бы играть перед ним. Как ни был бы пьян
%Поддубня, но он выступил защитником своих прав, говоря, что он музыке заплатил
%вперед, и что он не отпустит, ``пока она не отыграет своего''.

Виникло непорозуміння, в результаті якого Поддубня виявився викинутим із
зав'язаними назад руками у якийсь напівтемний чулан, названий арестантською
камерою. Тут Поддубня заснув міцним п'яним сном, від якого прокинувся годин
через п'ять, коли якийсь невідомий чоловік почав виштовхувати його на свободу.
Жах охопив в'язня, коли він, намацавши у себе під жилетом, виявив відсутність
грошей. Він одразу протверезів і зрозумів, що в кінец є розорений, оскільки там
під жилетом знаходилась виручка за увесь повністю проданий товар, він тепер
зрозумів, що тепер не було вже ні товару ні грошей. Залишилась злиденність.
Поддубня звернувся за допомогою до місцевої влади, але це виявилось марним.
``Там прокляте грецьке царство'' - пояснював він своїм слухачам, - ``Логафетов
багатий, у нього велика рідня, знайомства, наш брат тут нічого не вдіє''...
Писав Поддубня скаргу і губернатору, що визвало розслідування особо
командированого чиновника, який, списавши багато паперу, так і не прийшов ні до
якого результату. Загалом же скарги на Логафетова Поддубні та інших осіб,
%page8
%Произошло недоразумение, в результате которого Поддубня оказался выброшенным с
%завязанными назад руками в какой-то полутемный чулан, названный арестантской
%камерой.  Здесь Поддубня заснул крепким пьяным сном, от которого очнулся часов
%чрез пять, когда какой-то неизвестный человек стал выталкивать его на свободу.
%Ужас охватил узника, когда он, ощупав у себя под жилетом, заметил отсутствие
%денег. Он сразу отрезвел и понял, что в конец разорен, так как там под жилетом
%находилась выручка за весь полностью проданный товар, он понял, что теперь не
%было ни товара, ни денег. Осталась нищета. Поддубня обратился за помощью к
%местным властям, но это оказалось бесполезным. ``Там проклятое греческое
%царство'' - объяснил он своим слушателям, - ``Логафетов богат, у него большая
%родня, знакомства, наш брат ничего не поделает''...  Писал Поддубня жалобу и
%губернатору, что вызвало расследование особо командированного чиновника,
%который, исписав много бумаги, ни к какому результату не пришел. В общем же
%жалобы на Логафетова Поддубни и других лиц, вызвали
призвели його до усунення обов'язків члена грецького суда. Від цього, звісно,
не було легше Поддубні: гроші його пропали. ``По світу спустив, проклятий...
грабіжник'' ... - закінчив свою розповідь Поддубня, прикрасивши своє заключне
слово тією багатою по своїм відтінкам лайкою, яка не допускається в печаті,
хоча й тепер її нерідко можна почути на вулицях у великих навіть містах. Коли
Поддубня закінчив, Ільяшенко втрутився у розмову. Він запропонував Поддубні
``узятись'' за його справу і пояснив, що раніше з успіхов вів перед начальством
справи молокан. І дійсно, безперечними офіціальними даними встановлено, що
Ільяшенко  відвідував колонії молокан, оглядав їх на манер ревізора-чиновника,
виражав їм іноді своє благовоління, похвали і обіцянки про клопотання перед
високим начальством у Петербурзі, якого, звісно, він ніколи не виконував і
виконати не міг. При цьому варто зауважити, що Ільяшенко у цьому випадку діяв
``без усілякої користолюбної цілі і навіть без особливої спонукальної
причини''.  (Див. в архіві єкатеринославского окружного суда: журнал рішень за
1965 р. л. 29 л. д. 14 об).

%page8
%устранение его от исполнения обязанностей члена греческого суда. От этого,
%конечно, не было легче Поддубне: деньги его пропали. ``По миру пустил,
%проклятый... грабитель'' ... - заключил свой рассказ Поддубня, уснастив свое
%заключительное слово теми богатыми по своим оттенкам ругательствами, которые,
%(хотя еще и теперь нередко слышатся на улицах в больших даже городах) нетерпимы
%в печати.  Когда Поддубня кончил, Ильяшенко вмешался в разговор. Он предложил
%Поддубне ``взяться'' за его дело и пояснил, что раньше с успехом вел пред
%начальством дела молокан.  Действительно, безспорными официальными данными
%устанавливается, что Ильяшенко посещал колонии молокан, осматривал их на манер
%ревизора-чиновника, выражал им иногда свое благоволение, похвалы и обещания
%предстательства пред высшим начальством в Петербурге, какового, конечно, он
%никогда не исполнял и исполнять не мог. При
%этом достойно замечания, что Ильяшенко в этом случае действовал ``без всякой
%корыстной цели и даже без особой побудительной причины''. (См. в архиве
%екатеринославского окружного суда: журнал решений за 1965 г. л. 29 л. д. 14
%об).
На пропозицію Ільяшенко узятись за справу, Поддубня відповів відмовою, зауваживши...
``із чем же вести справу, коли я залишився голим як сокіл!'' ``ну так знайте'', заперечив Ільяшенко,
``я вам цього самого Логафетова у кандалах через Бердянськ відправлю у єкатеринославську тюрьму!''
До сих пір розмова відбувалась відносно спокійно і на ви, але, після слів, сказаних Ільяшенко, Поддубня 
впав у патетичний тон і перейшов на ти:
``будь благодійником, замкни його у тюрьму, останню сорочку зніму, віддам!'' ... закричав Поддубня. ``Нічого не треба,
тільки поставиш могорич'', удавав із себе великодушного Ільяшенко,
%На предложение Ильяшенко взяться за дело, Поддубня ответил отказов, заметив...
%``с чем же вести дело, когда я остался гол как сокол!'' ``ну так знайте'', возразил Ильяшенко, ``я вам этого самого 
%Логафетова в кандалах чрез Бердянск отправлю в екатеринославскую тюрьму!''
%До сих пор разговор происходил сравнительно
%спокойно и на вы, но, после слов, сказанных Ильяшенко,
%Поддубня впал в патетический тон и перешел на ты:
%``будь благодетель, запри его в тюрьму, последнюю сорочку сниму, отдам!''... завопил Поддубня. ``Ничего не надо,
%только поставишь могарыч'', великодушничал Ильяшенко, 
%page9
після чого раптово познайомлені обійняли один одного, і стали пити воду із знову принесеної пляшки. Біля цього
сосуду об'єдналась уся компанія із чотирьох чоловік, і дружньо вела голосним шопотом якусь довгу розмову.
%после чего внезапно познакомившиеся заключили друг друга в объятия, 
%и стали пить воду из вновь принесенной бутылки. Возле этого сосуда объединилась вся компания из
%четырех человек, и дружно вела громким шопотом какой-то длинный разговор.
День 5 квітня 1863 року був одним із тих весняних днів, коли зима робить
останню вилазку супроти наступаючого літа. Було ``сиверко'', як каже наш
простий народ; небо заволокло сірими хмарами, які не міг прогнати пронизаючий,
ні на одну хвилину не ослаблюючий своєї поривчастості північно-східний вітер;
степ не встиг ще покритись зеленню, і разом із сірим небом, представляв собою
як би суцільну сіровато-буру масу. Якщо би не довгі дні, та не особливе весняне
освітлення, яке, не дивлячись на хмари, що затулили сонце, таки відчувалось, то
можна було би подумати, що пори Роки повернули назад, що наступив листопад
місяць. Ледве з'явившийся ``накат'' по дорогам зіпсувався, завдяки дощу, що
безупинно падав на землю. Посеред такої обстановки, у вказаний день 5 квітня,
по дорозі із Бердянська в Маріуполь, верстах у двадцяти під останнього міста,
їхала однокінна підвода, на якій сиділи вже відомі нам особи: купець Мазин,
міщанин Колосовський та відставний кресляр Григорій Власов Ільяшенко. (Див. в
архіві єкатеринославського окружного суда справу, що вступила в палату із
олександрівського повітового суда разом із міскою ратушею про відставного
кресляра із обер-офіцерських дітей Гр. Вл. Ільяшенко л. д. 12).
%День 5 апреля 1863 года являлся одним из тех весенних дней, когда зима делает
%последнюю вылазку против наступающего лета.  Было ``сиверко'', как говорит наш
%простой народ; небо заволокло серыми тучами, которых не мог прогнать
%пронизывающий, ни на одну минуту не ослаблявший своей порывистости
%северо-восточный ветер; степь не успела еще покрыться зеленью, и вместе с серым
%небом, представляла как бы сплошную серовато-бурую массу. Если бы не длинные
%дни, да не особое весеннее освещение, которое, несмотря на закрывшие солнце
%тучи, таки ощущалось, можно было бы подумать, что времена Года повернули
%вспять, что настал ноябрь месяц. Едва образовавшийся ``накат'' по дорогам
%испортился, благодаря безпрерывно сыпавшемуся на землю мелкому дождю.  Среди
%такой обстановки, в указанный день 5 апреля, по дороге из Бердянска в
%Мариуполь, верстах в двадцати от последнего города, тащилась одноконная
%подвода, на которой сидели уже известные нам лица: купець Мазин, мещанин
%Колосовский и отставной чертежник Григорий Власов Ильяшенко. (См. в архиве
%екатеринославского окружного суда дело, вступившее в палату из александровского
%уездного суда обще с городовой ратушей об отставном чертежнике из
%обер-офицерских детей Гр. Вл. Ильяшенко л. д. 12).
Вони їхали більш ніж 50 верст по невеселій дорозі; їм залишалось до Маріуполя
ще верст 25 такого ж невеселого шляху.  Перед подорожуючими простирався ще той
же степ - однорідний, сірий, без деревця, без будь якого пейзажу, на якому можна було б зупинити око.
%Верст более 50 по невеселой дороге ехали они; им оставалось до Мариуполя еще верст 25 такого же невеселого пути. 
%Пред путниками простиралась все та же степь
%однородная, серая, без деревца, без какого то ни было пейзажа, на котором можно
%остановить глаз. Но наши путешественники относились к этому равнодушно и сидели
%page10
Але наші мандрівники відносились до цього байдуже і сиділи на своїй жалюгідній
повозці, бувши в тій апатії, що оволодіває мандрівником, добре наперед знаючим,
що ніякими зусиллями, хитрощами, благаннями не скоротити і не
змінити довгих виснажуючих годин шляху посеред мовчазного степу.
%на своей невзрачной повозке, проникнутые той апатией, которая по необходимости
%овладевает путником, хорошо наперед знающим, что никакими усилями, хитростями,
%мольбами, едущему не сократить, ни изменить долгих томительных часов пути среди
%безмолвных степей.
Біля чотирьох годин після полудня Ільяшенко та його товариші під'їхали до Маріуполя
і зупинились у передмісті міста, на Маріїнській стороні, в одній із непомітних хатинок;
готелів тоді у Маріуполі не було і приїзджаючі зупинялись у приватних квартирах. Того ж дня ввечері,
коли вже стемніло, Ільяшенко розшукав будинок Логафетова і приїхав до цього останнього.
(журнал рішення єкатеринославської палати уголовного суду за 1865 р., л. 29). Як ми вже знаємо, 
Логафетов у той час не був владою: він був усунений від посади члена грецького суда. Тим не менш він
жив у своє задоволення: був холостий, багатий, не особливо старий, йому було десь 50 років, хворобами
не страждав, у місті мав багатих та сильних родичів, серед яких він був свій та дорогий їх серцю чоловік.
Словом жилось йому добре, безтурботно, спокійно; його службові подвиги його не тривожили, бо ж усі розслідування
про його минулі діяння скінчились зовсім щасливо: ні про які судове переслідування не було й мови. Ось чому Логафетов
зверхньо віднісся до Ільяшенко, коли той став йому пояснювати, що йому загрожують неприємності
по скаргам Поддубні, і коли Ільяшенко запропонував йому свою допомогу для владнання цієї справи. В кінці кінців
Логафетов грубо вирядив Ільяшенко та закриваючи за ним двері, пропустив повз вуха звернений до нього
вигук: ``ти мене ще згадаєш!''... Зрозуміло, що цим словам Логафетов не надав рівно ніякого значення.
%Около четырех часов пополудни Ильяшенко и его товарищи подъехали к Мариуполю и
%остановились в предместьи города, на Марьинской стороне, в одной из невзрачных
%изб; гостиниц тогда в Мариуполе не было и приезжающие останавливались в
%частных квартирах. Того же дня вечером, когда уже стемнело, Ильяшенко разыскал
%дом Логафетова и явился к этому последнему.
%(журнал решения екатеринославской палаты уголовного суда за 1865 г., л. 29). Как мы уже знаем, Логафетов в
%то время не был властью: он был устранен от должности члена греческого суда. Тем не менее он жил
%припеваючи: был холост, богат, не особенно стар, ему
%было около 50 лет, недугами не страдал, в городе имел
%богатых и сильных родственников, среди которых он
%был свой и дорогой их сердцу человек. Словом жилось
%ему хорошо, беззаботно, спокойно; его служебные подвиги
%ето не тревожили, ибо все расследования о его прошлых деяниях 
%кончились совсем благополучно: ни о каких судебных 
%преследованиях не было и речи. Вот почему Логафетов свысока отнесся к Ильяшенко, когда тот стал
%ему объяснять, что ему грозят неприятности по жалобам
%Поддубни, и когда Ильяшенко предложил ему свое содействие для улаживания этого дела. В конце концов
%Логафетов грубо выпроводил Ильяшенко и закрывая за ним 
%page11
%дверь, пропустил мимо ушей обращенное к нему восклицание: ``ты меня попомнишь!''...
%Понятно, что этим словам Логафетов не придал ровно никакого значения. 

\par\noindent\rule{\textwidth}{0.4pt}

\textbf{\em Mihi non pudet fateri nescire, quod nesciam} и мені не соромно зізнатись, що у 
мене немає вміння писати творчо, створюючи художні типи, частково розкидані у життю. Якщо би я володів таким 
вмінням, я надав би перевагу зображенню описаних подій у формі комедії або драми, і в цьому місці зображення дійсності
опустив би завісу, закінчивши I-й акт.

Потім я почав би 2-й і останній акт, таким чином внесши більше інтересу і життя у цю роботу;
але, на жаль, це не під силу мені; я є лише смиренний літописець-протоколіст, і відновлюю забуту і,
на мій погляд повчальну історію по тому способу і прийому, як складають
звичайний судовий або поліцейський протокол. При цьому я використовував безсумнівні документи, що зберігаються
в архіві єкатеринославського окружного суду, дозвіл на переглядання яких був мені люб'язно наданий поважним
головою названого суду І. Ц. Патоном; потім я маю у своєму розпорядженні знайдену моїм шанованим 
товаришем уїздним членом таганрогзського окружного суду В. С. Станкевичем кореспонденцію ``Сина Вітчизни''
за 20 травня 1869 року та передану ними мені; накінець я широко користувався свідчими показаннями деяких старожилів міста Маріуполь,
які були не тільки безпосередніми свідками, але навіть учасниками описуваних мною подій.
%\textbf{\em Mihi non pudet fateri nescire, quod nesciam} и мне не стыдно сознаться, что у
%меня нет умения писать творчески, создавая художественные типы, частично
%разбросанные в жизни.  Если бы я обладал таким умением, я предпочел бы
%изобразить описываемые события в форме комедии или драмы, и в этом месте
%изображения действительности опустил бы занавес, закончив I-й акт.

%Затем я начал бы 2-й и последний акт, так внесено было бы более интереса и жизни в настоящую работу;
%но, к сожалению, это не под силам мне; я только смиренный летописец-протоколист, и восстанавливаю
%забытую и, на мой взгляд поучительную историю по тому способу и приему, как составляют обыкновенный судебный или
%полицейский протокол. При этом я пользовался бесспорными документами, хранящимися в архиве екатеринославского
%окружного суда, просмотр коих любезно был мне предоставлен почтенным председателем
%названного суда И. Ц. Патоном; затем я имею в своем распоряжении разысканную моим уважаемым 
%товарищем уездным членом таганрогского окружного суда В. С. Станкевичем корреспонденцию 
%``Сына Отечества'' от 20 мая 1869 года и переданную ими мне; наконец я широко воспользовался свидетельскими 
%показаниями некоторых старожилов г. Мариуполя, которые были не только непосредственными 
%свидетелями, но даже отчасти участниками описываемых мною событий.
%page12
Користуюсь нагодою приношу мою глибоку вдячність цим старожилам, із яких
деякі не побажали, щоби я згадував їх імена; я їм дякую за їх довгі, детальні свідчення та головним чином
за вражаючу вірність та точність їх свідчень, у чому я переконався співставлюючи їх покази
документальними даними, які зберігаються в архіві єкатеринославскього окружного суду. Після цього пояснення
продовжую перерваний протокол.
%Пользуюсь случаем принести мою глубокую благодарность этим старожилам, из
%которых некоторые не пожелали, чтобы я упомянул их имена; я их благодарю за их
%длинные, подробные показания и главным образом за поразительную верноеть и
%точность их свидетельств, в чем я убедилея сопоставляя их показания
%документальными данными, хранящимися в архиве екатеринославского окружного
%суда. После этого объяснения продолжаю прерванный протокол.
6 квітня 1863 року писар, що завідував канцелярією 
\textbf{\em ``Начальника Маріупольської команди внутрішньої стражі''}, в
установленому порядку, у 8 годин ранку, з'явився у невелику кімнату, на дверях
якої був прибитий напів-лист паперу із криво виведеним великими літерами
надписом: ``Канцелярія''. Писар був із молодих; цю посаду він займав тільки із
1862 року. Раніш він входив у склад команди, що була сформована у місті
Єкатеринославі із 40 рядових; команда була сформована в 1858 році, і в тому ж
році була надіслана у місто Маріуполь під головуванням її постійного начальника
капітана Лисенко. В команду, про яку йде мова, входили солдати найкращої
репутації, що мали нашивки. По свідченням старожилів у той час панувала
найсуворіша воєнна дисципліна: ``достатньо було чихнути во фронті, щоби
отримати в морду від начальства'', привожу справжні слова одного бувшого
солдата команди. Варто зауважити, що, даючи такі свідчення, старожили із бувших
солдат тим не менш додають: ``жити було прекрасно''. Потім вони пояснюють що
всі знали одне одного, що ``не було крадіжок, грабунків, шахрайства, готелів та
візників не було''. Очевидно, значить, що витівки та злочини Логафетова не йшли
у рахунок: ймовірно, їх ігнорували через повагу до його стану як начальника.
%6 апреля 1863 года писарь, заведывавший канцелярией 
%\textbf{\em ``Начальника Мариупольской команды внутренней стражи''},
%по заведенному порядку, в 8 час. утра, явился в небольшую комнату, на дверях которой был прибит полу-лист
%бумаги с криво выведенною большими буквами надписью: ``Канцелярия''. Писарь был из молодых; эту должность он занимал только с 1862 года.
%Раньше он входил в состав команды, сформированной в г. Екатеринославе из 40 рядовых; команда была сформирована в 1858 году, и в тот же год была прислана в город Мариуполь под начальством ее постоянного начальника капитана Лисенко. 
%В команду, о которой идет речь, входили солдаты лучшей репутации, имевшие нашивки. По свидетельству старожилов
%в то время господствовала строжайшая военная
%дисциплина: ``достаточно было чихнуть во фронте, чтобы
%получить в морду от начальства'', привожу подлинные
%слова одного бывшего солдата команды. Замечательно при
%этом, что, давая такие показанля, старожилы из бывших
%солдат тем не менее добавляют: ``жить было прекрасно''.
%Далее они объясняют что все знали друг друга, что 
%``воровства, грабежа, мошенничества, извозчиков и гостиниц
%не было''. Очевидно, значит, что проделки и преступления Логафетова в счет не шли: вероятно, их
%игнорировали из уважения к его начальственному состоянию.
%page13
Писарь войсковой команды отличался как образцовый ``службист'' и за образцовое
исполнение дисциплины, в 1860 году, был произведен в унтер-офицеры. Обязанности
писаря он исполнял с тем же неизменным усердием, с тем же строгим исполнением
дисциилины, как и обязанности рядового.

Прийдя 6 апреля в свою канцелярию, писарь подошел к большому, неуклюжему, грубо сделанному,
косому шкафу, достал из него несколько толстых тетрадей в потертом переплете и положил их на тут же
стоявший большой стол, такой же неуклюжий, как и шкаф.

Затем, обмакнув в пузырек с чернилом гусиное перо, он стал этим скрипучим
орудием выводить буквы на серой, немного мохнатой и похожей на войлок бумаге,
из которой состояли книги для входящих и исходящих бумаг.

``В то время'' (так показывает бывший писарь, ныне благополучно проживающий в гор. Мариуполе), 
\begin{quote}
\bfseries\em
твердым шагом, не спеша, но и не медля, входит приличный, 
солидный, с бравым видом господин лет тридцати. Роста он был выше среднего, блондин, одет в плохонькое пальто
светло коричневого цвета, заметно потертое; в такие пальто обыкновенно наряжены приказчики, стоящие за прилавком
\end{quote}

Неизвестный, войдя в канцелярию, поздоровался с писарем; последний встал,
вытянулся, поклонился, молча сел и стал опять усердно выводить буквы, но душа
его была уже охвачена какой-то неясной тревогой.

Писарь почувствовал, что необычайное появление нензвестного господина
неспроста. 

Между неизвестным и писарем в это время завязался следующий диалог.

\textbf{\em Неизвестный}: - Кто начальник?
%page14
\textbf{\em Писарь}. - Штабс-капитан Лисенко...

\textbf{\em Неизвестный.} - Хорошо-ли обращается с солдатами?

\textbf{\em Писарь.} - Так точно, все обстоит благополучно, позвольте доложиться начальн...

\textbf{\em Неизвестный} (резко обрывая). - Не надо останься...  Есть писчая бумага?

Уже после второго вопроса писарь понять, что предчувствие его не обмануло; 
ему показалось очевидным, что пред ним какой-то большой начальник, \textbf{\em ``ибо никто иной,
как только начальник станет спрашивать, как обращаются с солдатами''}; от этой мысли он заволновался,
\textbf{\em ``весь затряеся''} и дрожащею рукою положил на стол 6 листков белой бумаги.

Между тем, ``большой начальник'' уже отдал сухо приказ:

— Напиши на этой бумаге, что я скажу...

Оторопь охватила писаря, в голове мелькнула мысль: удрать, убежать, но это была
только мысль без решимости; на самом деле писарь боялся тронуться с места. Он
только смог произнести тихим, умоляющим голосом:

— Позвольте доложить начальнику команды.

Но на эту мольбу неизвестный еще более сухо, повысив голос, ответил:

— Не надо... Пиши!

Писарь покорно сел перед листом белой бумаги.  Неизвестный, поглядывая в
памятную книжку, вынутую им из внутреннего кармана своего верхнего пальто, стал
диктовать, а писарь записывать следующее:

\begin{quote}
\em\bfseries

По данной мне власти Государя Императора Всероссийского Александра Николаевича,
в Царском Селе, 18 мая 1862 года, лицом и именем которого повелеваю:
бывшего заседателя сего суда Николая Логафетова за грабеж и
смертоубийство
\end{quote}

%page15

Но силы писаря ему изменили. Уже когда он выводил слово: \enquote{Государя
Императора}, ему сперло дух и он чувствовал, как земля уходит из под его ног.
Когда же были упомянуты преступления Логафетова, о которых шопотом и почему-то
со страхом говорилось в городе, как будто бы все были соучастниками его
преступлений, писарь всем своим существом постиг, что перед ним - великую
власть имущая персона, и так заволновался, что его руки с гусиным пером среди
стиснувшихся пальцев запрыгали на бумаге. Начатый лист был испорчен появившейся
на нем чернильной кляксой и до неузнаваемости безобразно выведенными буквами
последних слов, отражавшими пляску руки на бумаге.

Неизвестный прекратил диктант и повелительно сказал писарю:

— Встань, оправься, пройдись по комнате, не надо дрейфить!

Пока писарь оправлялся, неизвестный стал около двери, отрезав путь к бегству. 

— Ну, теперь пиши, - опять приказал неизвестный, и, после вписания на чистый
лист уже написанного, продолжал диктовать следующее:
\begin{quote}
\em\bfseries
и вообще за все злоупотребления лишить всех прав состояния со ссылкою на Алтайские заводы 
в вечные работники, имение продать с публичного торга и удовлетворить всех должников и претендентов; а остальное затем
должно поступить в казну.
\end{quote}

Просмотрев написанное неизвестный собственноручно ``подписывает'' его так: 

\textbf{\em ``Полномоченный Государя моего, верноподанный Григорий Власов Ильяшенко. Мариуполь, апреля 6 дня, 1865 года''}
Подпись эта не произвела впечатления
%page16
на писаря, который, после перенесенных тревог, перестал временно реагировать и
только пассивно продолжал под диктовку, еще два приказа, текст коих я приведу в
точности несколько далее.

Все 3 приказа, написанные на 3-х отдельных листах, Ильяшенко спрятал во
внутренний карман своего пальто: затем, став вплотпую пред писарем и сказав
``cмотри'' расстегнул верхние пуговицы жилета и вынул из под него складной
медальон из красной меди, висевший на Андреевской ленте.  В этом медальоне был
портрет Государя и под крышкой кусочек белой бумаги на коей имелась надпись:
\textbf{\em ``быть по сему, Александр 2-й, Царское село, 17 мая 1862 года''}.
Эти слова были написаны обыкновенным почерком Ильяшепко, как это удостоверено
судебной экспертизой и как это было очевидно впоследствии для всякого
обозревавшего настоящую подпись. (Журнал решений Екатер. Палаты угол. суда за
1865 года, л. 29)

``Смотри'', говорил Ильяшенко; наступая на писаря:
\begin{quote}
\em\bfseries	
ты знаешь, кто я таков, видишь, чем я награжден от
Государя. Знай, что если ты известишь начальника команды до ревизии мною суда,
сегодня же будешь повешен!
\end{quote}

С этими словами, произнесенными с зловещим грозным шепотом, Ильяшенко быстро
вышел из канцелярии.  У писаря отнялся дух, опять потемнело в глазах, застучало
в голове: не донести начальнику - беда, донести еще хуже, мелькнуло в его
сознании; недвижимый, словно окаменелый, остался он в своей канцелярии, как
будто тысяча железных цепей приковали его к месту.

\par\noindent\rule{\textwidth}{0.4pt}

Николай Фотиевич Логафетов, не обратив никакого внимания на угрозу почти
вытолкнутого им из своего дома Ильяшенко, на утро совершенно о нем позабыл. 
%page17
По обыкновению хорошо выспавшись, он встал рано не спеша стал одеваться, поверхностно 
умылся и предался спокойному чаепитию.

Часов около 10 утра он отправился в соседнюю лавку своего приятеля К...нова, где он бывал каждое утро
и где каждое утро приятели вели один и тот же разговор.
Во 1-х, они говорили, что жилось бы гораздо лучше, если
бы всегда можно было быть уверенным купить товар дешево и продать дорого. Во
2-х, оба они печалились друг другу о том, что в Мариуполе начинают селиться пришельцы из разных концов Империи, не греки;
за ними, говорили между собою приятели, и на базар не успеешь ничего купить;
при этом они вспоминали, как остроумно их общий друг, некий местный обыватель, выразил протест 
против такого вопиющего положения вещей. Этот их
друг, идя на базар купить курицу, увидел, что одна из
вновь поселившихся русских женщин возвращалась с базара и несла в корзине курицу. 

Такое предупреждение намерений греческого поселенца возмутило его дух; он остановил своего конкурента в деле покупки курицы
и, выхватив из корзины птицу, бросил женщине 20 копеек, причем был настолько галантен, что представил ей объяснение своего поступка:
``за вами чертями ничего не успеешь купить!''
``Еще 20 коп. уплатил'', добавлял Логафетов тем тоном, из которого 
безусловно явствовало, что он лично ограничился бы только отобранием
курицы без всяких дальнейших действий. В особенности
же Логафетов и К-в в своих собеседованиях возмущались тем обстоятельством, 
что пришельцы открывали в
городе лавки и являлись в торговле опасными конкурентами. В то сравнительно
недавнее время уеловия общественной жизни в Мариуполе не походили на сегодняшние.
%page18
Сегодня смешно говорить о Мариуполе, как исключительно греческом городе; в нем
нет серьезного различия по национальностям: все обыватели этого города
постененно смешиваются и во всяком случае положение всех обывателей равны перед
законом и властями. Тогда по свидетельству старожилов, дело обстояло иначе.
Пришельцы не греки, составляли ничтожное меньшинство, которое побаивалось
греков, представлявших большинство, силу, с которой не могли бороться новые
поселенцы. Кроме того, греки имели свое самоуправление, свою обусловленную
привилегированную организацию, которая не распространялась на невыходцев из
Крыма. Последние подчинялись общим дореформенным учреждениям, под сенью которых
жилось не легко. Так продолжалось до начала семидесятых годов, когда введено
было городовое положение на общих основаниях. До этой же поры греки, выходцы из
Крыма, занимали господствующее положение, а остальные обитатели угнетенное.
Греки были недовольны пришельцами, а последние их побаивались и питали к ним
недобрые чувства.

И так Логафетов и К-ов затянули свой обычный разговор; на этот раз он был скоро
прерван. Из греческого суда прибежал старик сторож и на турецком наречии
(многие из греков поселенцев и теперь сохранили это наречи) передать
Логафетову, чтобы он поспешил придти в суд, что его требует председатель Попов.

Логафетов направился в суд совершенно спокойно.
Случалось и раньше, что его требовали иной раз в присутствие
для некоторых неважных разъяснений, которые он мог представить, как бывший член суда;
правда, что чаще эти объяснения переводились с места на собеседования по душе, которыми совершенно
затушевывалась главная цель вызова. Вот почему, даже не без некоторого удовольствия, Логафетов направился 
%page19
в присутствие, предвкушая хотя всегда один и тот же, но тем не менее всегда
одинаково для него интересный разговор о торговле, о ``пашенице'', о зловредных
новых людях, поселившихся недавно в Мариуполе.

В это время один из таких новых поселенцев,
почтенный купец -ов (ныне благополучно проживающий
в Мариуполе, развивший и многократно возвеличивший свое
предприятие) пришел в свою лавку и сел по обыкновению
на табурете, стоявшем у двери, выходившей на улицу. 

Моросил мелкий дождь, но ветер, свирепо дувший накануне с северовосточной
стороны, изменил направление, стал дуть с юго-востока и в воздухе начала
разливаться весенняя, оживляющая мир, мягкость; предшествующий холодный день
оказался на самом деле последним приступом зимы, утомившей непривыкших к холоду
южан своими длинными бездеятельными днями.

Купець -ов, смотря на улицу, среди мертвой тишины, 
громко передавал себе свои впечатления: 

\begin{quote}
\em\bfseries

И куда это
Иван Павлович (начальник инвалидной команды, капитан Лисенко) поехал в дрожках?
А вот за ним четыре солдата бегут рысцой, ружья держат под шинелью, чтобы дождик не замочил. А среди-то всех во как
ковыляет ногами старый солдат цирюльник: ему то сердечному, старому и хилому, не легко поспевать за строевыми!
\end{quote}

Так в мыслях купца -ова отразилось виденное
им на улице и он осталея сидеть спокойно на своем
месте, праздно и разсеянно размышляя на тему: куда бы это
поехал Иван Павлович?

Эти размышления были по прошествии некоторого промежутка времени, прерваны появлением мастерового 
Анохина, имевшего ошалелый вид. В согласии с его необычным внешним состоянием были и его действия: войдя в лавку, 
он начал, без всяких к тому побудительных причин, учащенно и усердно креститься.
%page20

— Чего крестишься? — вопросил удивленный купец -ов.

В ответ посыпались отдельные слова: \emph{Чудеса Капитан! Саблю наголо!..
Четыре солдата?.. Логафетов и т. п.}

— Да ты с ума сошел. — прервал бессвязный поток слов купец —ов и увел расстроенного 
посетителя в следующую комнату, где шепотом Анохин продолжал столь странно начатое  
повествование.

Я уже заметил, что когда речь заходила о деяниях Логафетова, то мариупольские
обыватели говорили шепотом и со страхом, как будто они были
соучастниками Логафетова. Анохин же не только усвоить эту общую всем
привычку, но сверх того был приведен в трепет виденным и опасался как
бы не быть в ответе за то, что он рассказывает виденное... Впрочем,
восстанавливать события по отрывочным словам и фразам Анохина я не берусь
и предпочитаю возвратиться к имеющемуся у меня точному, достоверному
материалу.

Ильяшенко, оставив писаря в канцелярии, направился
в греческий суд; там он застал одного столоначальника
уголовного отделения суда, Могулянского. Последнему, как
говорят, Ильяшенко сразу поставил вопросы ребром и
налег на него со стремительностью. ``Где председатель и
члены суда''? — ``Состав суда выезжает сегодня для разбора
дел в Ялту''. — ``Вы коронный''? (т. е. состоите ли на 
коронной службе или выборный). - ``Да, коронный''... Ильяшенко
показаль портрегь Государя, после чего столоначальник
превратился в воплощенный вопросительный и восклицательный знаки. ``Ни слова''! - грозно крикнул Ильяшенко. —
%page21
``Немедленно послать за председателем и членами''! Столоначальник пустился
бегом исполнять приказ. Состав суда (председатель и три члена) в это
время состоял из людей основательных, уже немолодых, умудренных, так
сказать, годами. Один секретарь был молод, но зато он был умудрен
знанием на память некоторых законов и наипаче циркуляров; это
признавали все и более всех уверенно сам секретарь.

Члены греческого суда, надо сознаться, при всей солидности, страдали полным
неумением составлять суждение о делах, подлежащих их рассмотрению, но
это не препятствовало функционированию названного учреждения, ибо в
этом случае выручал секретарь со своими циркулярчиками и законами. Он
писал определение, ссылаясь на проставленные им статьи, а председатель
и члены суда медленно и старательно их подписывали, не постигая их
содержания, что, однако, не задерживало течения дела. В остальном
секретарь ничем достопримечательным не выделялся.  Можно только
отметить, что уже прежде познания законов он постиг, что как законы,
так и циркуляры существуют для угождения начальству.

Не прошло и двадцати минут после ухода Могулянского, как запыхавшись, но уже в
настоящем одеянии и при надлежащем знаке, явились: председатель, три
члена суда и секретаръ. Ильяшенко продолжаль действовать с тою же
стремительностью, с какою он обратился к столоначальнику. Едва
поздоровавшись, он вручил председателю тот приказ, который он
продиктовал писарю и текст коего приведен выше.

Председатель навел глаза на врученную ему бумагу; соображал он туго, медленно, но все таки довольно
скоро понял, что случилась беда, и в душе у него похолодело;
%page22
он не знал, что сказать, что делать. Ильяшенко вывел
его из состояния нерешительности, потребовав дать ему
надежного человека, с которым он мог бы послать важную и секретную бумагу
начальнику команды, капитану Лисенко.

Председатель позвал одного из сторожей, которому Ильяшенко и вручил в
запечатанном конверте второй приказ, зараннее написанный под его диктовку,
писарем инвалидной команды. Председатель и по русски, и на турецком наречии
напутствовал посланного наставлениями. Ильяшенко затем объявил председателю,
что суд не поедет сегодня в Ялту, так как ему предстоит заняться рассмотрением
чрезвычайного государственного дела о Логафетове, которого Ильяшенко потребовал
немедленно призвать в присутствие суда. Остальные неопределенные, но
наставительного характера, общие указания весь состав суда выслушал, стоя и
молча, находясь в состоянии людей, которых пришибли сильным ударом по голове.

\par\noindent\rule{\textwidth}{0.4pt}

Капитан Иван Павлович Лисенко был старый служака: еще до севастопольской войны
он служил в кантонистах и за усердную службу был назначен фельдфебелем
селенгинского пехотного полка. Во время севастопольекой
войны он храбро сражался, был ранен и произведен в
подпоручики со старшинством. Когда в 1858 году была
сформирована команда из сорока рядовых, четырех унтер-офицеров, к которым впоследствии был прибавлен один
барабанщик и один пирульник, Лисенко, в чине поручика, был назначен начальником этой команды. В том 
же 1858 году команда из Екатеринослава совершила переход в Мариуполь, 
%page23
где и оставалась под начальством
Лисенко, в скорости произведенного в капитаны. Таким
образом в мариупольском округе капитан Лисенко оказался единственным и 
властным представителем военной
власти. С виду Лисенко был настоящий богатырь: в плечах сажень, 
грудь как большой котел, а ростом он
был выше на голову каждого из нижних чинов своей
команды и даже каждого из обывателей города Мариуполя;
голос его вызывал содрогание даже неодушевленных предметов. 
Лисепко мог пить, но всегда был трезв. К своей
службе относился строго; знал и исполнял дисциплину
в совершенстве. На парадах, при маршировке, при фронте
священнодействовал. К солдатам относился строго, справедливо и заботливо. Соблюдение дисциплины требовал
неумолимо и карал за всякое малейшее ее нарушение, в конец расстраивающее расположение его духа. Даже 
лучшего и образцового унтер-офицера он наказал на первый день Светлого Праздника за то, что он 
на параде, поставленный против ярко-весеннего солнца, ``натужился'' и чихнул, после чего Лисенко, разгневавшись, сейчас
же прекратил парад, считая дело испакощенным. 

Лисенко привык, умел и любил исполнять безпрекословно приказания и при этом
знал все, что от него требовалось. Одного он не знал и не умел за полным
отсутствием  практики в этом отношении: он не умел думать, и всей душой
ненавидел такое положение, когда ему приходилось отвечать на вопрос как
поступить, ибо он мог только поступать, но не обсуждать, как поступать. Так,
например, если бы он получил приказ перебить мариупольских обывателей, ему
легче было бы исполнить такое требование, чем обдумывать и размышлять над
вопросом, следует ли, возможно ли исполнить такой приказ.
%page24

Жена Лисенко, Дарья Кондратьевна, была основательная дама, сорока восьми лет 
(Арх. екатер. окружн. суда, указанное выше дело, л. д. 64). Грамоты она совсем не знала.
По крайней мере, под ее показанием, данным следственной власти,
 вместо ее подписи значится отметка следователя:
``не грамотная''. Зато она была несравненная хозяйка,
жизнь которой хлопотливо протекала среди неизменно каждодневно чередующихся 
забот о кухне, погребе, о соленьях,
о борще и т. д. Гигант капитан Лисенко по внешности
был под пару своей жене. Как почти все мужья, капитан слегка побаивался своей жены; 
последняя питала почтение, главным образом, к начальственному положению своего мужа и менее к его
личности. Несмотря на свое умственное неразвитие, капитанша вовсе не относилась к службе и деятельности
своего мужа с тем презрительным высокомерием, которым угощают нередко внешне приличные, но в душе полудикие
женщины, вышедшие замуж за писателей, художников, мыслителей и т. п. выдающихся людей.
Дарья Кондратьевна была неразвита, но не обладала дикой душой, стремящейся к разрушению.
Поэтому она мирилась, как с необходимостью, и с фронтом, и с маршировкой и с приносимыми из канцелярии
бумагами на квартиру в капитану. Большим необходимым
злом капитанша считала приносимые на дом бумаги, ибо
они всегда вызывали некоторое беспокойство и тревогу капитана, 
а через то и некоторый безпорядок в дом: капитан, поглощенный подписом бумаг, 
запаздывал к обеду, кричал, звал, отдавал распоряжения и т. д. Мирясь
со всем этим, Дарья Кондратьевна, хотя и не высказывала,
все таки в глубине души смотрела на все эти бумаги и беспокойства,
как на дело пустое; она понимала, что из всего этого не
выйдет ни хорошей начинки для пирогов, ни солений, ни борща.

%page25
Вот почему, когда она первая встретила посланного
из греческого суда с приказом Ильяшенко, она пренебрежительно равнодушно приняла бумагу
двумя пальцами в
тех видах, чтобы на бумаге осталось поменьше бурякового
кваса, в который были смочены ее руки, и бросила эту
бумагу на пропитанный жиром кухонный стол. Об этом
моменте вот как свидетельствует сам капитан Лисенко.
Привожу доподлинное его показание, сохраняя его стиль и орфографию:

\begin{quote}
\em\bfseries
	
Приказ Ильяшенко прислан из суда неизвестным человеком, передан в руки моей
жены на крыльце во время бытности ее в кладовой... сказала мне, что
требуют в суд, сама же как говорит, понесла продукты в кухню, в то
время я был в другой комнате переодевался из одежды (Л. д. 60
указанного выше дела).

\end{quote}

Текст этого приказа, написанного как мы знаем, писарем
команды, под диктовку Ильяшенко, был следующий: 

\begin{quote}
\em\bfseries

Государь Император Высочайше повелел соизволить, в Царском селе, 13 мая 1862
года, приказал вам исполнить мое требование, в отношении спокойствия и
прекращения злоупотребления в Новороссийском крае. Ныне поручаю вашему
благородию приготовить с вверенной вам команды четыре человека с одним
унтер-офицером и одного нестроевого цирюльника, при которых должны
находиться пара кандалов и арестантское платье. Все прописанные нижние
чины и вы сами лично обязаны явиться в тот час и минуту как получите
сей приказ, в мариупольский греческий суд, где, приняв преступника, с
первым этапом отправить к месту назначения за строгим караулом, а также
поручаю вам на будущее время, в случае важных происшествий в Мариуполе
и ближайших ему окрестностей, немедленно донести министру внутренних
дел, статс-секретарю Валуеву минуя свою прямую дистанцию.
	
\end{quote}
Следовала
%page26
затем подпись Ильяшенко, тождественная с подписью на
первом приказе, только после нее рукою Ильяшенко было
добавлено: \textbf{\em ``В десять часов одиннадцать минут, город
Марbуполь, шестого апреля, 1863 года''}.

В своих дальнейших показаниях капитан сознается, 
что он не обратил внимания на то обстоятельство,
что приказ написан хорошо ему известным почерком
писаря его команды. Произошло это вследствие того, что
слова приказа: \emph{``Государь Императоръ Высочайше повелел
соизволил''} сильно \emph{``ветревожили''} капитана.

По показанию старожилов, внутренная тревога Лисенко
проявилась наружу следующим порядком: голосом, способным на смерть ушибить 
не подготовленного слушателя,
капитан стал кричать: \emph{``одеваться, олеваться''}!... Многократным
повторением этого слова капитан хотел объяснить, 
чтобы ему помогли скорее облачиться в парадную форму.

Дарья Кондратьевна, хотя и была привычна к голосу своего супруга, однако,
подскочила несколько раз на месте, как токующий тетерев, испуганная неслыханным
еще столь сильным криком капитана: она даже обронила на земле те ``продукты'',
что держала в руках; тем не менее она скоро оправилась и устремилась к
капитану, который, спеша, никак не мог справиться с переодеванием в парадные
узкие брюки.  Ее помощь была как нельзя боле кстати задыхавшемуся от волнения
ее супругу.  Пока он окончательно подправлял себя, Дарья Кондратьевна успела
отдать приказание кучеру, чтобы он запряг лошадь и позвал четырех солдат,
цирюльника и барабанщика.

Хотя волнение и спех и помешали действовать с той
быстротой, какая требуется военными людьми, тем не менее
%page27
капитан достаточно скоро появился во двор в надлежащем виде и, не теряя ни
минуты, стал во главе четырех застывших пред ним солдат, цирюльника, и
барабанщика, с которыми и двинулся в путь в том порядке, в каком их видел
купец —ов, когда он из своей лавки смотрел на улицу. Капитан, очутившись на
улице и освеженный слегка моросившим дождем, оправился от волнения и чувствовал
в своей душе ту отвагу, ту решимость немедленно действовать, какой он отличался
в геройские дни севастопольской войны, когда надо было сходиться грудь с грудью
с неприятелем.

\par\noindent\rule{\textwidth}{0.4pt}

После того как, по приказанию Ильяшенко, из греческого суда 
был послан один надежный человек с
известным нам приказом к капитану Лисенко, а другой - за Логафетовым, 
сам Ильяшенко не терял времени. Прежде всего он приказал председателю про себя
внимательно вчитаться в переданный ему приказ № 1-й,
коим, как нам известно, Ильзшенко определял сослать
Логафетова \textbf{\em ``на алтайские заводы в вечные работники''.} 
Затем он приказал смущенному и в глубине души 
разъедаемому сомнениями секретарю суда, стряпчему Хартахаю,
составить от имени греческого суда определение об осуждении на каторгу Логафетова.

Секретарь пугался и от необычайного составляемого
им приговора, и от казавшегося ему несогласия определения суда с известными
ему законами и циркулярами, и, главное, от стремительности действий суда, упразднявшей всякую судебную
логику.

Но Хартахай в этот трудный момент не забыл
надежно сидевшей в его душе идеи, что все должно совершать для угождения начальству,
и он решил не противоречить Ильяшенко, тем более, что последний прибег
%page28
к довольно грозным покрикиваниям и понуканиям, когда
секретарь обнаружил замешательство. Вот почему определение быстро продвигалось вперед,
сопровождаемое иногда и одобрительными восклицаниями, в роде: ``так, верно,
пиши дальше''.... и было окончено к приезду капитана Лисенко.

Случилось так, что в присутствие греческого суда капитан Лисенко вошел почти
одновременно вслед за Логафетовым. Таким образом, все главные действующие лица
настоящей драмы оказались собранными вместе друг против друга. Такое их
стечение оказалось в высшей степени благоприятным для миссии Ильяшенко.
Это явствует из данных впоследствии объяснений действующих лиц по вопросу о возникновении \emph{``слепого верования''}
в \emph{``лжеуполномоченного Ильяшенко''}.

Капитан верил потому, что был требован в присутственное место, где все исполнялось 
по приказанию Ильяшенко, (Там же л. д. 17 об.). Председатель суда окончательно доуверовал, 
благодаря
\emph{``внезапному появлению начальника инвалидной команды капитана Лисенко, 
да еще с солдатами и в парадной форме''};
члены суда уверовали \emph{``смотря на председателя и секретаря
Хартахая, повиновавшихся привазаниям Ильяшенко''}. (Там
же л. п. 16 об.). Наконец председатель Попов, в последних своих объяснениях еще
указывал, 
что укреплению веры в ``лжеуполномоченного'' содействовал и Логафетов, который ``стоял с повинной головой, как настоящий
виновник''.
Правда, такое объяснение вызвало негодование
Логафетова, который подал жалобу на Понова и писал,
что эти слова Попова, он ``принимает за преднамеренную
обиду на письме принесенную; невольно влекусь к подозрению'' (там же л. д. 69—72), - продолжал он 
жаловалься, что Попов действовал неспроста, что у него
были задние мысли... Но последние, по тщательному розыску,
%page29
найдены не были.

Едва капитан Лисенко вошел в присутствие суда,
вслед за Логафетовым, как Ильяшенко, указывая пальцем на 
Логафетова, твердым и решительным голосом
обратился к капитану: \emph{``именем закона приказываю вам
арестовать Логафетова''}. Решимость действовать проявилась
наружу: капитан молодецки выхватил саблю из ножен
и, держа в руке это оружие поднятым вверх, громовым
голосом скомандовал солдатам стать с ружьями у дверей
и никого не впускать и не выпускать из присутствия
суда. Наступила пауза, Ильяшенко один нарушил воцарившуюся 
гробовую тишину, объявил секретарю, что он
должен приготовиться к прочтению определения суда. Хартахай сейчас 
поднялся и начал, не ожидая дальнейших
приказаний, без всякой торжественноети, монотонно читать
написанный им приговор. Ильяшенко с места его оборвал 
и грозно возопил: \emph{``Мальчишка, службы не понимаешь!
Не знаешь как, когда следует читать!''}... Мгновенно восстановилась 
мертвая тишина. Ильяшенко достал из-под
сюртука медальон с портретом Государя и, держа его
перед собою, скомандовал: ``На караул!'' Сабля капитана
Лиеенко сделала вольт в воздухе и он сам, равно как
и его солдаты, превратились в застывшие статуи. — \emph{``Теперь
можешь читать, только смотри толково читай''}, - строго обратился
Ильяшенко к секретарю; последний на этот раз
блатополучно прочитал приговор, который корреспонденту
``Сына Отечества'', в номере от 20 мая 1869 г., передает \emph{``на память потомству, как замечательный документ
в летописях суда''}. Привожу это определение суда дословно, без сокращений. Вот оно:
%page30
\begin{quote}
\em\bfseries
	
1863 года, апреля 6 дня. По Указу Его Ииператорского Величества, мариупольский греческий суд слушали
приказ уполномоченного Государя Императора Всероссийского,
Григория Павлова Ильяшенко, от 6 апреля, в коем объявлено 
по данной ему Государем Императором власти в
Царском Селе, 18 мая, 1862 года, лицом и именем
Его Величества повелевает: бывшего заседателя сего суда
Николая Логафетова, за грабеж и смертоубийство, и вообще
за все злоупотребления, лишить всех прав состояния с
ссылкою в алтайские заводы в вечные работники, а имение
его продать с публичного торга и удовлетворить всех
должников и претендателей, а остальное затем должно
поступить в казну. Приказали: решение это объявить бывшему заседателю сего суда 
Логафетову и передать его, согласно личному приказанию господина уполномоченного, в
распоряжение начальника мариупольской инвалидной команды,
штабс-капитана Лисенко, а исправляющему должность заседателя Ганжи 
предписать описать и оценить все имение
Логафетова и опись представить в суд. Председатель Николай Попов. 
За секретаря А. Ганжи, заседатель Оксюзов,
исправляющий должность секретаря и стряпчего Хартахай.
Не согласен на исполнение и буду телеграфировать г. начальнику губернии. 
Исправляющий должность стряпчего Хартахай. В присутствии мариупольского греческого суда постановление 
утверждаю. Полномоченный Государя Императора
верноподанный Григорий Власов Ильяшенко. 6 апреля 1863
года в $11 \sfrac{3}{4}$ часов слушали и подписали.

\end{quote}

Обстоятельства, которые будут указаны в дальнейшем
изложении обнаружат, что приписка Хархатая: ``несогласен
на исполнение и буду телеграфировать начальнику губернии'',
позднейшего происхожденя. Первоначальное определение суда
было написано без таких либеральных оговоров и, после
%page31
скрепления этого приговора подписью Ильяшенко и его громким возгласом: 
\emph{``быть по сему''}, начался немедленно обряд приведения 
приговора в исполнение.

— \textbf{\emph{Цирульника, бритву, барабан!}} — быстро и громко скомандовал Ильяшенко.

Цирульник, тут же стоявший возле солдат, побледнел: бритва была им забыта
впопыхах на квартире команды.

Барабан только числился при команде, но в натуре его не было.

Не давая объяснений, цирульник вылетел из присутствия суда и что есть мочи
устремился на квартиру за бритвой. В два конца ему пришлось побежать более
версты, но он себя не щадил и через каких нибудь 10 минут вернулся еще более
бледный, без дыхания, держа в дрожащих руках какую-то ободранную, изъявленную
зазубринами бритву.

Пока шли краткие приготовления к обряду исполнения приговора, Логафетов
переживал тяжелые минуты. В первое мгновение он не полностью постигал ожидавшую
его участь: он хотел верить и верил, что случилось какое-то неприятное
недоразумение, которое сейчас само собою разъяснится; но ход событий с его
стремительным натиском подрывал эту веру. Страх и отчаяние в конце концов
охватили бывшего заседателя суда; он не знал, что делать, что сказать в свою
пользу и в этом подавленном состоянии, на местном турецком наречии обратился,
в поисках за помощью к председателю суда, но тот не оказался добрым утешителем:
указывая на Ильяшенко, председатель только сказал:

— Власть его выше закона.

Взгляд Логафетова случайно упал на принесенную
бритву и это вызвало у него мысль заявить Ильяшенко,
\textbf{\emph{``что, кажется, теперь не велено брить''.}}
%page32

Стряпчий Хартахай, привычный указывать греческому
суду законы и давать указания, добавил от себя:

— Брить отменено.

Этого Ильяшенко не мог простить и он сердито разнес стряпчего, 
приведя его в состояние настоящего смятения.

— Мальчишка! — кричал Ильяшенко, — ничего не понимает; отменено брить содержащихся, 
а уголовных преступников, ссылаемых по конфирмации куда следует,
должно брить по прежнему.

— Подожди, — переводя дыхание добавил строгий ревизор, — доберусь и до тебя потом и тебе не миновать Сибири!

Последине слова как-то особенно глубоко запали в
душу стряпчего; он подумал: ``Кто не без греха... Неровен час... всяк человек 
в воле начальства''.

Шансов на сопротивление у Логафетова не оставалось
никаких, ибо капитан Лисенко был в той фазе решимости действовать, 
когда не оставалось таких препятствий,
которых бы он не сокрушил ради исполнения приказа начальства. Он сам впоследствии передавал 
своему знакомому, нынешнему секретарю мариупольского съезда, г. Б — му,
что в тот торжественный момент он так уверовал в
Ильяшенко, что мурашки у него бегали по спине и волос
с затылка поднимался. \emph{Если бы Ильяшенко}, — передавал
всегда восторженно и с экстазом капитан, — \emph{в тот момент приказал ``коли'' — всех бы переколол, еели бы
скомандовал ``пли'' — всех перестрелял бы.}

По прошествии многих годов капитан всегда приходил в экстаз при воспоминании 
о знаменательном моменте; предоставляю всякому судить, какой силы был подъем духа 
капитана в самый момент действия.

По кивку головы Ильяшенко капитан Лисенко и два
создата приблизились к Логафетову и, не встретив ни
%page33
малейшего с его стороны сопротивления, посадили его, за
отсутствием барабана на сундук; найденным в присутствии полотенцем
ему связали назад руки. После этого
ноги его одели в кандалы. Затем к нему подступил
дрожащий цирульник; в правой руке он держал наготове 
ту единственную, изъязвленную зазубринами бритву,
которая числилась в инвентаре, находящемся при команде.
Старожил Мариуполя, передававший мне порядок бритья
половины головы, пояснял, что цирульник, после беготни
и со страха, исполнил бритье неправильно. 
``Сами подумайте'', - говорил рассказчик, -  ``начал брить с затылка
против шерсти. Ну, конечно, задрал кожу; он хотя и
примазал, а Логафетову все таки было неприятно, и кровь
текла как из резаного''. Эта мелочь ни на минуту не
остановила обряда и бритье было закончено безпрепятственно;
после чего на голову осужденного солдата натянули арестантскую шапку; 
в заключение бывшему заседателю развязали руки и одели его в арестантский халат. Логафетов,
считавшийся опасным человеком, которого всяк еще так
недавно побаивался, в виду его крутого нрава, вдруг сделался
кроток, как ягненок и старательно исполнял все,
что от него требовали. В произведенном алекс—им исправником 
следствии этот учиненный над Логафетовым
обряд назван ``делом о невинном истязании мещанина
Николая Логафетова в присутствии мариупольского греческого
суда'' (л. д. 158, там же). Относится ли слово невинный
к истязанию или Логафетову остается не выясненным. Но
когда происходили указанные выше действия над Логафетовым, 
никто не думал об истязании; все знали только,
что приговор приводится в исполнение на законном основании. 
Вот почему капитан Лисенко не замедлил поставить Логафетова между солдат, из коих два стояли
сбоку, один спереди и один сзади. 
%page34
Сам же капитан, получив от Ильяшенко приказ отвести ``преступника'' в
тюрьму, стал впереди и, держа саблю наголо, приказал
отряду следовать за собой. Вскоре собравшийся на улице
народ с любопытством смотрел на шествие отряда, в
котором среди солдат, путаясь в кандалах, шествовал
бывший начальник, еще так недавно наводивший страх
на местных обывателей. Все смотрели на удалявшегося Логафетова со страхом,
некоторые с грустью; более же всех
уныло смотрел на кортеж стряпчий Хартахай, из смущенной души которого 
под тяжелым впечатлением недавнего
разноса вырвался следующий меланхоличесвй возглас: \emph{``все
мы пропали, скоро все пойдем туда-же''}...

События этого дня в Мариуполе, как видит читатель, текли с поразительной
быстротой; около 10 часов утра Ильяшенко впервые открылся писарю команды в
канцелярии начальника команды; к 12-ти часам дня произнесен над Логафетовым
приговор и моментально утвержден, и приведен в исполнение. В 12 часов дня
капитан Лисенко ведет осужденного окруженного четырьмя солдатами в тюрьму.
Несмотря на такую стремительность событий, толпа успела осведомиться о
чрезвычайном событии и, собравшись в большом количестве, длинным хвостом
замыкала шествие небольшого отряда, предводимого капитаном Лисенко.  Коренные
обыватели Мариуполя, хотя и хорошо были осведомлены о служебных подвигах
Логафетова, тем не менее не только не говорили: \emph{``по делом вору и
мука''}, но видимо были смущены и сконфужены. Надо помнить, что с самого
момента своего переселения из Крыма в Мариуполь, греки находились в
исключительных условиях и пользовались такими правами, каких до 60-х годов не
имело население Российской Империи, за исключением дворянства.
%page35

Не говоря о специальных привилегиях, заметим только, 
что греки пользовались правом избирать из своей
среды состав греческого суда, являвшийся учреждением,
которое полностью совмещало судебную, 
полицейскую и хозяйственно-административную функцию во всех делах, касавшихся
греков. Это учреждение было излюбленным детищем
греческого общества и в свою очередъ являлось доброй
матерью для всякого грека и самой злой мачехой для всякого
постороннего. Греки любили, ценили и гордились своим
греческим судом и иначе не называли его как \emph{наш}
суд, делая ударение на слово \emph{наш}. Понятно поэтому, что
быстрая расправа над бывшим судьей греческого суда,
являлась достаточно ощутимым афронтом для всего греческого населения.  За то
новые поселенцы Мариуполя, не греки, терпевшие притеснения от греков, ликовали
столь шумно и неудержимо, как это требуется при проезде начальства или при
установленных приказанием начальства праздневствах. Неизвестно почему кто-то в
этот момент встретившихся ликований и смущений пустил слух, что явившийся
ревизор ни кто иной, как сам Великий Князь Михаил Николаевич. Слух моментально
распространился и ему так поверили, что всякий, кто осмелился бы заявить
сомнение, наверное, если бы не быль избит, был обруган дураком или еще хуже
приписан к бунтовщикам, к неблагонамеренным, опасным людям, отрицающим власть.
Итак, одни были сконфужены, другие ликовали, а Логафетов, со скверно обритой
головой, в арестантском халате, улрученный, сидел в тюрьме.

Председатель Попов, согласно доброй, старой традиции
и поныне к счастью не умирающей, рассудил, что как
бы важны нн были государственные дела, тем не менее
они не умаляют существенной важности обеда. Поэтому,
%page36

после того как Лисенко отрапортовал, что престулник
завлючен в тюрьму, Нонов обратился к высокому ревизору с почтительной 
просьбой сделать честь откушать хлеб-соль. Предложение было принято. 
Тогда Попов пригласил
также обедать весь состав суда, секретаря и капитана
Лисенко, которые весьма рады были и поесть, и побыть в
обществе высокопоставленного лица. Обед состоял из местных греческих явств, и, главным образом, отличался
обилием разного материала, подлежащего еде; желающие
могли, не обирая соседей, есть до достижения отвращения к
пище и все таки не одолеть всего обеда. Шампанское в
то время в Мариуполе не водилось; заздравных тостов
произносить не умели и поэтому ревизор не был осыпан
речами с исчислением его добродетелей. Правда, за водкой,
стуча рюмкой о рюмку, участники обеда говорили: \emph{``за ваше
здоровье! по чаще! по больше!''}... Но дальше этого красноречие не шло.
После нескольких рюмок, когда настроение людей становится более откровенным, Попов обратился к
высокому гостю с просьбой ``по душе'' сказать; верит ли
он полной виновности Логафетова, на что Ильяшенко ответил, что в виновности этого последнего
не может быть ни малейшего сомнения. Также не отказался Ильяшенко ответить некоторым из 
присутствующих, интересовавшихся узнать, каким образом он получил свои чрезвычайные
полномочия от Государя. Не стесняясь, Ильяшенко рассказал 
какую-то фантастическую историю об его участии в
охране священной особы Государя Императора, после чего
Государь его лично узнал и осчастливил особым вниманием и доверием.
Когда же затем в достаточно обнаженной форме высказывалось, что настоящее дело такого рода,
что 16.000 рублей ассигнациями такая сумма, которая не
велика, чтобы вычеркнуть все дело Логафетова и уничтожить
%page37
все его следы, то Ильяшенко прямо ответил: \emph{``и рад бы,
но не могу, ибо лично уполномочен Государем Императором!''} Обед не затянулся; через какой нибудь час
времени все чувствовали себя в таком насыщенном состоянии,
что продолжать еду было бы для них мучением.
Ревизор выразил желание отправиться к себе на квартиру,
при этом запретил кому либо его сопровождать. Хозяин
дома приказал заложить лошадь в дроги — единственный
в то время известный обывателям Мариуполя экипаж.
Пока закладывали лошадь, Попов и его сослуживцы просили Ильяшенко 
объяснить, где он остановился; они считали себя обязанными явиться к ревизору на дом \textbf{\em in corpore},
так сказать, еще раз представиться и откланяться. Но
Ильяшенко категорически отклонил всякие такие проекты,
не указал своей квартиры и освободил мариупольские власти
от явки для представления.

Расставаясь с хозяином и его гостями Ильяшенко
отозвал в сторону капитана Лисенко и вручил ему третий
и последний приказ, написанный также, как мы знаем,
рукою писаря. Этот приказ был более краток; текст
его следующий: 

\begin{quote}
\em\bfseries
	
Уголовный преступник Николай Логафетов по конфирмации подлежит ссылке в
Алтайские заводы в вечные работники и должен следовать чрез
Екатеринославскую губернию, куда немедленно его отправить за вверенным
Вам Его Величества караулом до города Бердянска, и оттуда через г.
Александровск в г. Екатеринослав, где местное начальство примет свои
меры для дальнейшего отправления. По данной мне власти 18 мая 1862 года
в Царском Селе сим повелеваю полномоченный Государя моего и
верноподданный Григорий Власов Ильяшенко. Г. Мариуполь, 6 апреля 1863
года, в час пополудни.

\end{quote}
%page38

Несмотря на выраженное ревизором желание, чтобы
никто не утруждал себя нанесением прощального визита,
Попов, тем не менее, солидно рассуждая, что и излишнее
усердие к начальству только приносить пользу, тайно приказал 
своему кучеру заметить дом, у которого остановился
начальник. Когда Ильяшенко расстался с Поповым, остальная компания, 
посоветовавшись с хозяином, решила, что
общая явка к внезапно прибывшему высокопоставленному
лицу безусловно необходима. Порешили, что должны явиться
купно все власти, имея во главе военную власть, капитана
Лисенко; затем находили, что было бы весьма великолепно,
если бы шествие властей замыкали главные именитые
граждане г. Мариуполя. Оповестить последних обязались
члены суда, которые и не замедлили разбежаться по разным концам города. Итак
было условлено, что часа через два, т. е. между тремя и четырьмя часами дня,
все соберутся к Попову и оттуда совместно направятся во временную квартиру г.
ревизора. До тех пор каждому предоставлялось идти домой
для приведения себя в достодолжный порядок. Стряпчий Хартахай, пользуясь
свободным временем, зашел к своему знакомому, Д—чу, сообщить о чрезвычайном
происшествии дня. Д—ч был известен всему Мариуполю, как человек образованный,
развитой; сверх того совершенно справедливо его считали весьма сведущим в
законах.  На службе он не состоял, проживал в Мариуполе частным человеком,
владея невдалеке от этого города землей. Люди с злым языком уверили, что Д—ч
был единственный умный человек во всем городе. Вероятно, к такому заверению
нужно внести поправку и сказать, что Д-ч был самый умный среди окружающих его
обывателей г. Мариуполя.
%page39

Угнетенный событиями дня, стряпчий Хартахай, вновь воспринял смущающий дух
удар, когда Д-ч на его слова заметил: \emph{``се дило таке, що не Логафетова, а
вас всих за ваш суд погонять на алтайские заводы''}. Д—ч любил уснащать свою
речь малоросейскими словами и фразами и его слова, как по определенности
содержания, так и по способу их выражения, всегда производили на слушателя
сильное впечатление.  Перепуганный стряпчий заметался от этой новой точки
зрения на дело. Что же делать? — попросил он совета. Сейчас телеграфировать
губернатору, — определенно решил Д—ч.

Этот совет прояснил мысли Хартахая; он сразу
сообразил, что телеграфировать по службе начальством
разрешается и что телеграмма не вызовет гнева ревизора,
раз он составит телеграмму в смысле доношения его превосходительству о прибытии чрезвычайного
уполномоченного.
 
В таком виде он и послал телеграмму. Из изложенного сейчас ясно, почему я выше
заметил, что протестующая приписка Хартахая на приговор суда была сделана не в
момент подписания приговора, а позже. В тот момент стряпчий не посмел бы
противоречить грозному ревизору. Такое мое заключение подтверждается еще
следующим местом из позднейшей жалобы Логафетова, который, по конец своей жизни
остался недоволен действующими лицами греческого суда.
Логафетов писал: 

\begin{quote}
\em\bfseries

Хартахай телеграфировал в Екатеринослав, не только по совершении надо мною
истязания, но едва ли не после открытия шарлатанства Ильяшенко, в чем
можно удостовериться справками на телеграфе.  

\end{quote}

Сверх того, Логафетов указывал, что Хартахай в телеграмме только доводит до сведения начальства о появлении уполномоченного
(там же л. д. 41).
%page40

Вот почему, говоря словами из жалобы Логафетова,
\emph{``я невольно влекусь к подозрению''}, что приписка Хартахая
на приговор греческого суда, сейчас после подписи: \emph{``несогласен на исполнение и буду телеграфировать 
г. начальнику губернии''}, — сделана впоследствии, после совещания с
Д-чем или, вернее, после получения ответной телеграммы
губернатора, которая была доставлена как раз в то время,
когда всё собрались у Попова, чтобы купно шествовать откланиваться к отъезжающей персоне.

Необходимо пояснить, что в то время телеграммы ходили не так, как теперь. 
Тогда люди меньше чем теперь пользовались этим способом сношения друг с другом,
поэтому, при соответствии числа телеграфным аппаратов с
количеством посылаемых телеграмм, последние не залеживались на телеграфных станциях 
и передаточных пунктах в роли кандидаток, долго ожидающих своей очереди.
Словом, практика показывает, что теперь телеграммы ходят у вас, на юге России, 
со скоростью от 8 до 10 верст в час и не изменяют этой, практикой установившейся скорости, 
даже в экстренных случаях: так, например, во время недавних безпорядков на 
мариупольских заводах, телеграфное сообщение торжественно сохраняло
принцип равноправия и телеграммы начальства аккуратно
прибывали, после приезда лиц, извещавших о своем прибытии
и делавших свои предварительные распоряжения. Но
в описываемое время таких прогрессивных новшеств не
было; телеграммы начальнику губернии от него передавались 
почтительно и без замедлений. Конечно, нам с нашим настоящим каждодневным 
опытом трудно поверить,
чтобы существовало такое благодатное время, когда на телеграмму, посланную из Мариуполя в 
Екатеринослав получался бы ответ через час или два часа; 
%page41
теперь в подобном случае, мы должны ожидать ответа два или три
дня. Тем не менее, каковы бы ни были наши представления,
факт остается фактом, и в описываемое время, получить
в Мариуполь ответную телеграмму через час или два,
считалось нормальным. Поэтому пусть читатель не удивляется 
тому обстоятельству, что предупрежденный Хартахаем
начальник телеграфной конторы, принес ответную телеграмму 
губернатора в дом Попова, между тремя и четырьмя
часами пополудни, когда все власти и именитые граждане
были в сборе для шествия к высокому начальнику с
представлением.

\par\noindent\rule{\textwidth}{0.4pt}

Собравшееся общество оказалось в превеликом затруднении, вследствие того, 
что кучер Попова не узнал, где квартира начальника ревизора. Все по этому поводу охали,
торячились, спорили, и единогласно критиковали несообразительность кучера. 
Но последний не был виноват; он не мог
узнать, где квартира начальника, благодаря хитроумным действиям этого последнего. Выехав за город, на
гору, к тому месту, где теперь тюрьма, Ильяшенко остановил кучера, 
дал ему на воду и приказал повернуть
и уехать назад в город. Ослушаться кучер не посмел
и шагом поехал обратно, рассчитывая оглянуться и заметить,
хотя направление, куда направил свои стопы начальник. 
Но этот последний, вероятно, быстро прилег в
первой попавшейся рытвине, потому, что кучер, оглянувшись
никого не видел и только про себя заметил:
\emph{``канул, как в воду''}. Вследствие такого оборота вещей, все общество, 
собравшееся у Попова, не знало что делать и какими
путями разыскать начальство. Из затруднения всех вывела телеграмма, вскрытая Хартахаем;
%page42
в ней значилось: 
\emph{``Приказаний уполномоченного не исполнять, немедленно его арестовать''.}
Следовала подпись губернатора. \emph{``Д—ч угадал!''} —
перепуганным голосом возопил стряпчий Хартахай, — \emph{``мы
все пойдем на каторгу''}... У капитана Лисенко первый
раз в жизни подкосились ноги и он грузно опустился
на стул. Остальные разом заохали, зачмокали, застонали.
Хартахай раньше других постиг произведенную телеграммой губернатора новую \textbf{\em mise en scene} 
и у него сейчас же проснулось присущее стряпчему, прокурорскому оку,
усердие найти автора преступления и повергнуть оного под
карающую длань неумолимого закона. \emph{``Господа''}, — первый
заговорил он, — \emph{``нельзя упустить преступника!''} — \emph{``Да какой он преступник, он невинно сидит в тюрьме''}, —
послышалось в ответ несколько голосов. Все так освоились с мыслью осуждения Логафетова, что почитали слова
стряпчаго относящимся к Логафетову. Хартахай взбесился от этой
неповоротливости сообразительности и на кончике языка висел у него ответ:
``бараны''! Но лица, к коим должно было приложиться сие название, уже поняли
свое \textbf{\em qui pro quo}, и стали совместно и громко говорить что-то в
свое оправдание, причем все усердно уверяли друг друга, что они с первого
момента, как увидели Ильяшенко, были уверены, что он шарлатан.  Почему раньше
никто не открыл такой своей уверенности, об этом никто не сказал ни одного
слова.  Все тем не менее говорили много с явной тенденцией превознесть
собственную свою проницательность.  В общем поднялся неимоверный гвалт: все
представители греческого общества жестикулировали почти всеми органами своего
тела, хватали друг друга руками за ворот сюртуков, тыкали пальцами друг другу в
лицо и все кричали, как будто соперничали между собою голосами, подобно оперным
певцам.

%page43
Бедный Хартахай надорвал свои голосовые связки, прежде чем ему удалось
восстановить кой какую тищину. Не теряя ни минуты, он сейчас же выработал план
поимки преступника. Решено было собравшихся у Попова именитых города Мариуполя
граждан направить за поисками лжеуполномоченного, (так с этого момента Хартахай
называл Ильяшенко), в предместье города, на Марьинскую сторону, куда всего
удобнее было укрыться с того места, где Ильяшенко исчез с глаз кучера Попова.
Заседатели греческого суда должны были набрать сторожей разных присутственных
мест, кликнуть волонтеров и с этими соединенными силами пуститься в обход всего
города. Наконец капитан Лисенко командировалъ 3 солдат и одного унтер-офицера,
тоже на Марьинскую сторону в подкрепление к именитым гражданам. Сам же
председатель Попов, стряпчий Хартахай и капитан Лисенко должны были оставаться
недвижимо в квартире Попова и ждать известий. Их назначение состояло в том,
чтобы по открытии местонахождения лжеуполномоченного, немедленно отправиться к
этому последнему и объявить его арестованным.

Поиски именитых граждан и соединенного отряда, руководимого заседателями были безуспешны;
на все их расспросы им отвечали, что никакой начальник нигде не появлялся. Они ошибочно искали
чрезвычайных знамений чрезвычайного уполномоченного в виде развевающегося флага
над домом, занимаемом этим последним, или в виде полосатой будки с заключенным в ней полицейским и т. п.
Но таких знамений не дано было им открыть. Именитые граждане ходили толпой, 
горячо разговаривали, спорили и тоже ничего знаменательного не открыли. Посланные
же капитаном солдатики заметили возле одной из хат
на Марьинской стороне ту самую одноконную повозку, на
%page44
которой Ильяшенко и его товарищи накануне въехали в
г. Мариуполь. Солдатики уклонились от сложных рассуждений
и непосредственно решили, что эта повозка никого
иного, как только начальника, вероятно в пользу этого
быстрого заключения говорило то обстоятельство, что местное
население не пользовалось одноконными телегами, а употребляло двухконные, 
похожие на современные фургоны. Один
солдатик, подкравшись, заглянул в окно и увидел человека, 
растянувшагося на лавке; как заглянувший в окно,
так и его товарищи опять непосредственно решили, что
никому иному не растягиваться на лавке, как только начальнику и 
побежали отрапортовать о своей находке капитану Лисенко.

\par\noindent\rule{\textwidth}{0.4pt}

Вперед каюсь; на одну минуту я изменю своей роли
летонисца-протоколиста и не воздержусь от упрека, который
мне хочется бросить Ильяшенко. Мне хочется ему сказать:

\begin{quote}
\em\bfseries
	
Ильяшенко, Ильяшенко, почто ты, нагрузившись достаточно за обедом у Попова,
напился окончательно пьян, придя домой. Зачем ты, спровадив кучера Попова, не
запряг своей лошади и вместе с твоими товарищами не переехал через реку
Кальмиус на Донскую сторону. Для этого надо было потерять менее полчаса времени
и через каких нибудь полчаса никто никогда тебя не отыскаль бы, и никто никогда
не узнал, откуда, с неба или с земли, появился этоть ошеломлящий, властный
уполномоченный. Какое бы осталось широкое поле любителям мистики излагать
теорию о навождении, любителям точного знания - о границах самообмана чувств,
коллективного внушения; а людям благонамеренным, трепет носящим в душ,
представился бы случай поговорить на тему, что никто не знает ни дня, ни часа,
ниже образа и вида, в котором может появиться высшее начальство. О, Ильяшенко,
зачем ты напился пьян!

\end{quote}

\par\noindent\rule{\textwidth}{0.4pt}
%page45

Представители разнородных властей: председатель Попов, стряпчий Хартахай и
капитан Лисенко стояли в нерешительности перед дверьми того дома, в котором
солдатики выследили предполагаемого начальника. Никто не решался войти.
Председатель говорил в том смысле, что бесстрашие военных людей обязывает
капитана показать пример доблести. Капитан ничего не говорил, но решительно и
отрицательно качал головой. Старожилы утверждают, что капитан, жестоко
перепугавшись после получения телеграммы губернатора, впал в нервное состояние,
что стал всего бояться и даже ``сам себя боялся''.

Хартахай намекнул, что Попов первое лицо в городе,
но на это, в виде возражения, получил ответ, что \textbf{стряпчий - око закона}.
Наконец после долгих колебаний все три представителя власти решили войти вместе.
Набравшись духу, они влетели в землянку, где почивал опьяневший Ильяшенко.
Он проснулся, протеръ глаза и, услышав что он
арестован, таким громовым голосом разнес вошедших,
грозя каторгой и всеми ужасами Сибири, что вошедшие не
выдержали отпора и бежали из хаты, вновь пораженные
тревожною мыслью: а вдруг перед ними персона. Так
разносить, кричать, ругать можеть только начальство — в
этом все трое были глубоко убеждены. Но поглощенная
Ильяшенко водка опять ему повредила. Пока Ильяшенко
разносил, явившиеся к нему власти задыхались не только
от волнения, но и от отвратительного запаха сивухи, которая 
исходила от Ильяшенко, как из разлившейся бочки.
%page46

Благодаря этому Хартахай и смог сообразить, что настоящий начальник не будет
преисполняться исключительно столь омерзительным питьем, как простая сивуха: по
рассуждению Хартахая, настоящая персона можеть только для начала отведать
простой водки, — для открытия, так сказать, выпивки, а излишеству предастся
более благородными питиями. С этим справедливым рассуждением стряпчего его
спутники вполне согласились. Все трое опять вошли. Хартахай решительно заявил,
что Ильяшенко именем закона арестован и что, в случае сопротивления, капитан
пустит в дело стоявших у дверей солдат.  Серьезность положения отрезвила
Ильяшенка; он понял, что доведение дело до рукопашной было бы фатально для его
престижа - и он избрал другой способ сопротивления. \emph{``Хорошо''}, -
ответил он ворвавшимся к нему властям, - \emph{``я пойду за вами, только
знайте. что я вам покажу, кто я; весь век будете плакаться!''} Ильяшенко,
действительно, последовал за Хартахаем, Поповым и Лисенко.

Было более 5 часов пополудни; склонявшееся к западу, яркое, 
весеннее солнце выглянуло из за разорвавшихся туч и мягким, радующим взор светом залило
невзрачные Мариупольские домики, грязную немощеную улицу,
Ильяшенко, Попова, Хартахая, капитана Лисенко и большую
сопровождавшую их толпу людей. Все эти люди направлялись в присутствие греческого суда, 
куда они скоро и прибыли без всяких инцидентов. Отсюда, первым делом, председатель суда Попов
отправил в тюрьму приказ, который я копирую с сохранением не только выражений, но и орфографии. Вот этот приказ:

\begin{quote}
\em\bfseries
Смотрителю Тюремного замка. Сейчас освободить из тюрьмы и 
прислать в суд содержащегося Николая Логафетова. Апрель
6 дня 1863 года. Председатель К. Попов.
\end{quote}
%page47

Согласно этого приказа Логафетов был доставлен в суд в том самом виде, в каком
утром его препроводили в тюрьму, т. е. в кандалах и арестантском одеянии.

Логафетов продолжал пребывать в состоянии угнетения, доведшего его до полной
апатии; он молчал, не выражал радости по поводу возвращенной ему свободы, тупо
и медленно озирался, и только по временам тяжело со стоном вздыхал.

Капитан Лисенко был ни жив, ни мертв, преследуемый следующими безысходными думами:

\begin{quote}
\em\bfseries
	
если Ильяшенко шарлатан, капитану не избежать суда за содеянное над
Логафетовым; если же Ильяшенко таинственный и чрезвычайный начальник,
как в этом он продолжал упорно уверять, постоянно твердя: ``попомните
меня, никото не забуду''... то совсем не трудно угодить в каторгу, если
не на виселицу...

\end{quote}

Хартахай, председатель, заседатель и наводнившие присутствие суда именитые граждане пылали
злобой против Ильяшенко, обиженные посмеянием над их родным учреждением и над всеми ими. 

Сняв с Лотафетова арестантское одеяние, эти рассерженные люди 
со злобой одели в него лжеуполномоченнаго.

Но заключить Ильяшенко в кандалы греки не посмели; они все-таки по инерции
продолжали чувствовать совершенно ни на чем определенном не основанный страх;
вероятно, просто на их впечатлительность действовало то обстоятельство, что
Ильяшенко, не падая духом, продолжал их разносить и стращать всякими ужасами; а
также вероятно им импонировала сохраненная арестованным манера держать себя с
неукоризненной величественностью настоящей персоны.
%page48

Но так как надо было что нибудь делать с опасным узником, 
то в конце концов по наставлению Хартахая, 
суд в полном составе и присоединившаяся толпа
именитых граждан, все вместе, повели Ильяшенко в
тюрьму. Половину путешествия совершили безпрепятственно.

Только со стороны русских поселенцев, находившихся в толпе, было проявлено
нечто, похожее на манифестацию в пользу Ильяшенко. Эти русские люди открыто
уверяли, что греки влекут в тюрьму Великого Князя.  Конечно, если бы таких
протестантов находилось несколько сот человек, шествию не сдобровать; оно было
бы силою остановлено и узник получил бы свободу, но на несчастье Ильяшенко
таких русских людей не было и десятка; их протест не мог иметь реального
значения; на них никто не обращал внимания. Шествие же, когда половина пути
была пройдена, приостановилось благодаря решительному
действию одного из именитых граждан, содействовавшего
властям и греческому суду в арестовании лжеуполномоченного. 
Подобно доброму коню, получившему шпоры, он
стремительно выскочил из толпы, стал на ее дороге и,
растопырив руки, во всю глотку заорал:

— Стой, стой, стой!... — произнося это слово то по русски, 10 по турецки.
Задыхаясь, он стал затем говорить скоро, крича, размахивая руками и часто
отплевываясь. Смысль его обильных спешно сыпавшихся слов сводился к тому, что
все греки тотовят себе Сибирь, каторгу, что все они \emph{``дураки и бараны''},
не могли понять, что они сажают в тюрьму не только Ильяшенко, но и царский
портрет, висящий на шее арестованного.

Многие из слушавших воскликнули, мноме ударяли себя собственною дланью по лбу, все остановились
на минуту как вкопанные, и затем все шествие повернуло
%page49
обратно в греческий суд.  Здесь люди, погалдев в четверть
часа, поспорив и в нескольких отдельных случаях обругав друг друга, сняли с шеи Ильяшенко медальон с портретом
Государя и опять повели узника в тюрьму,
куда на этот раз доставили его благополучно и безпрепятственно.

Так кончились начальственные похождения Ильяшенко из г. Мариуполя. Как видит читатель, они
ограничились одним днем: в 10 час. утра Ильяшенко объявился в канцелярии начальника команды, в 12 часов дня
свершился суд и исполнение его решения над Логафетовым, между часом и двумя состоялся обед на манер банкета, а к 
5 часам пополудни Ильяшенко уже был арестован.

\par\noindent\rule{\textwidth}{0.4pt}

Эпилог мог бы составить новую историю, открывающую одну из обычных страничек дореформенного суда,
отживавшего в то время свои последние дни. Порядки эти всем хорошо известны и мы будем кратки.

Логафетов, обрев свободу и возвратившись домой, долгое время оставался в
раздумьи: обрить ли ему вторую половину головы или ждать, пока бритая половина
отрастет.  Несколько месяцев ходил он с повязанным на голову платком. Он кипел
злобой против всего состава греческого суда — бывших приятелей, и, пользуясь
услугами какого-то грамотея, писал бесконечные жалобы на все местные власти и
особенно на председателя Попова и стряпчего Хартахая, усматривая в действиях
этих последних лиц злые, корыстные намерения против своей личности.

Все эти жалобы, несмотря на многословие, заключали только одну улику против обвиняемых
Логафетовым лиц, состоявшую в том, что Ильяшенко был в простом
%page50
старом пальто и весь наружный вид его ни мало не соответствовал настоящему начальнику.

Такая шаткая улика оказалась, понятно, недостаточной
и жалобы Лотафетова по бездоказательности, оставлены без
последствий.

Но хуже всего для жалобщика было то обстоятельство, что, поразившее его в
момент приведения над ним приговора в исполнение, угнетенное состояние духа не
только не ослаблялось, но с каждым днем усиливалоеь. Он осунулся, избегал
людей, потерял аппетит, словом быстро шел на убыль. Протянув в таком виде года
полтора, Логафетов умер.

Так относительно него, еще в земных условиях,
исполнился закон воздаяния злом за зло.

Капитан Лисенко, Хартахай, Попов и члены греческого суда 
около двух лет пребывали в неизреченном страхе. Кроме Лисенко, все остальные лица, ища спасения
от ответственности за содеянное, обнаружили не особенно
высокую культуру душевных свойств. Попов, Хартахай
и вообще весь состав греческого суда прилагали все усилия, 
чтобы вызвать подозрение, будто капитан Лисенко находился 
в предварительном заговоре с Ильяшенко.

Следы таких злокозненных действий сохранились в
некоторых документах.

Такъ, напр., в журнале ``входящим и исходящим
мариупольской команды на 1865 год'' есть указание, что
уже 229 апреля мариупольский греческий суд отношением 
за № 1598 коварно просил капитана Лисенко уведомить,
на каком основании начальник команды прибыл с людьми в сей суд 6 апреля для 
арестования мещанина Логафетова.
%page51

Капитан Лисенко не кривил душою и отношением от 95 апреля, за № 207, напрямик
отвечал, 

\begin{quote}
\em\bfseries
что прибытие начальника команды с конвойными людьми в
присутствие сего суда 6 апреля состоялось по полученному
приказу, подписанному именующим себя полномочным
Государя Императора — Ильяшенко.
\end{quote}

Затем, в судебном деле есть особый следственный
акт от 14 сентября 1863 года, возникший 
\begin{quote}
\em\bfseries
по поволу неоднократных ссылок председателя Попова и Хартахая
на то обстоятельство, что невероятно, чтобы лжеупономоченный 
Ильяшенко не имел свидания с капитаном Лисенко до прибытия в греческий суд.
\end{quote}

Но все эги хитрые маневры, долженствовавшие направить следствие в желательном для Попова и Хартахая
направлении не увенчались успехом: следствие воочию доказало, что капитан Лисенко все время действовал, как бравый
солдат, по совести, \textbf{\em bona fide}.

Добродетель не пострадала и капитан Лисенко не подвергся никакому
преследованию. Года через три он совсем освободился от страха ответственности
за содеянное над Логафетовым.  Когда ему приходилось разсказывать о
знаменательном дне 6 апреля, он всегда и неизменно впадал в энтузиазм.
Несколько лет тому назад, в глубокой старости, капитан скончался. 

Немного сложнее и длиннее должно быть повествование о дальнейшей судьбе Ильяшенко.

В то время, как все участники настоящей эпопеи
пребывали в страхе и трепете, боясь ответственности, один
Ильяшенко, которого судьба висела на волоске в течение
почти трех лет, не падал духом. Заключенный в мариупольскую тюрьму, он сохранил такой начальственный вид,
держал себя так свысока, то порицая, то одобряя,
%page52
то награждая окружающих будущими обещаниями, 
что не только низшие чины тюремной администрации, но и сам смотритель тюрьмы подвергся действию
превеликого сомнения и смущения. Как скажет бывало Ильяшенко 
с величественной ввушительностью: \emph{``Попомнишь меня! увидишь скоро,
одобряю, награжу''}, — то и пойдет в тюрьме шопот,
отчего и происходило смятения тюремных начальственных душ.
Омущало всех, что такого преступника раньше не
бывало, и поэтому неудивительно, если все оканчивали свой
шопот заключеникм: \emph{``нет, тут что-то неспроста''.}

Этому смятению душ содействовало и то обстоятельство, что вне тюрьмы
разростался, как вкруг от брошенного в воду камня, пущенный несколькими
русскими людьми нелепый слух, будто Ильяшенко никто иной, как Великий Князь.
Авторы этого слуха основывали догадку на том, казавшимся многим убедительном
соображении, что ни у кого на шее не могло быть портрета Государя, как только у
Великого Князя — брата Государя.

Как бы то ни было, через несколько дней, по заключении Ильяшенко в
мариупольскую тюрьму, не было в городе такого человека, который бы не слышал
объяснения, что греки посадили в тюрьму Великого Князя. Этот слух не замедлил
выйти из пределов города Мариуполя, и, окрепши, стал распространяться по
окрестным селам, деревням, и даже проник в соседние города.

Достоверно известно, что весною 1563 года, в селе Максимилиановке, отстоящем от
города Мариуполя более чем на сто верст, сельский сход, собравшись в
присутствии старосты, имел совещание о своих нуждах и 
\begin{quote}
\em\bfseries
между прочим, имел суждение о том, что господа помещики посадили в тюрьму Брата
Государя, а потому, поговоря между собою, постановили...
\end{quote}
%page53

Так записывал малограмотный сельский писарь сущность того, что составляло
предмет рассуждений на сходе. По обыкновению, он писал постановление схода,
после того, как этот последний разошелся. Писарь мало думал о том, что хотел
постановить сельский парламент; по опыту он хорошо знал, что все им написанное
сойдет за постановление схода; для этого ведь стоило ему только вписать фамилии
неграмотных крестьян села и затем призвать одного грамотнаго крестьянина и
предложить ему расписаться за себя и за неграмотных; на практике никогда не
было примера отказа в учинении такого пустяка, как расписаться, хотя ни один из
рукоприкладчиков никогда не интересовался знать, в чем состояло постановление
схода и вообще, что он подписывал.

Писаря затрудняло другое обетоятельство: он никак
не мог сообразить, что писать после слова: \emph{``постановили''}.
Во всех приговорах, написанных им раньше, после этого слова писали:
распределить землю, нанять пастуха, обложить платежем и т. д.

Никакие такие слова в данном случае не подходили. Писарь тщетно морщил чело,
вертелся на табурете, погружал гусиное перо в чернильницу.  Ничего не выходило.

В конце концов, писарь скомкал в руке начатое им постановление сельского схода.

Косность деревни препятствовала таинственным слухам, которым все безусловно
верили, вызвать какой-нибудь значительный эффект. Люди удивлялись коварству
помещиков (которых, к слову сказать, в мариупольском уезде теперь совсем нет, а
в то время было не более 5 или 6 человек), и затем, пошумев иногда на сходе,
входили в спокойную колею деревенской жизни, не предпринимая ничего
действительного и целесообразного относительно возмутивших их покой событий.
%page54

Но зато мариупольские дамы, почуяв таанственное, воспламенились к Ильяшенко и
старались поддержать его репутацию. Он потянули в тюрьму как на богомолье.
Благо в то время никаких тюремных строгостей и формализма не практиковалось:
всякий кто хотел мог, без всякого разрешения, посещать тюрьму, когда ему было
угодно и сколько раз ему было угодно.

Боюсь, что недоверчивый читатель сейчас спросит: как же, при таком отсутствии
надзора и строгостей, все арестанты не убегали из тюрьмы. Замечу на это
попрошающему, что и я ставил старожилам такой же вопрос; они отвечали мне, что
не полагалось бегать и что в то время арестантов бежало не больше, чем теперь.
Всякий, мол, знал, что начальство не дозволяет бежать.

Благодаря такимъ удобным условиям для посещения тюрьмы, дамы стали самым
щедрым образом наносить визиты узнику. Ильяшенко их принимал, ни на минуту не
изменяя своей начальственной величественности, и наградил многих из
посетительниц обещанием, что он их в надежное время \textbf{\em вспомнит.}

Один из достовернейших старожилов объяснил мне, что дамы в общем для ухаживания
за Ильяшенко не ссорились между собою, ибо в это ухаживание романтические
чувства не входили.  Даже, напротив, проявляли самую усердную, солидарную
заботливость об Ильяшенко, он с общего всех согласия, ввели принцип разделения
труда: барыни из уезда узлами привозили для питания узника фрукты, овощи,
масло, молоко и всякую живность, городские же дамы ежедневно приносили в тюрьму
горящие пирожки, обеды, сладости и т. п. Но особенно трогательно выразилась
заботливость дам в доставлении узнику всякого потребного белья и костюмов. В
этом отношении щедрость их была так велика, что по свидетельству старожилов,
Ильяшенко, каждый раз, вызываемый к следователю на допрос, являлся в новом
костюме.
%page55

Паломничество дам в тюрьму не особенно импонировало тюремной администрации и
начальнику тюрьмы. Последний смотрел на дам, как на народ несерьезный, большей
частью не понимающий того, что делает.  Но зато посещение разных неизвестных
господ, приезжавших издалека и усердно большей частью не понимают того, что
делают. Но зало стремившихся и добивающихся аудиенции у заключенного,
продолжавшего себя держать с величественностью персоны наполняло ядом тревог и
сомнений бедную душу начальника тюрьмы.

Окончательно его добил некий полковник и некий генерал, посетившие Ильяшенко в
один и тот же день.

Полковник, после аудиенции, быстро вышел, встряхнув всем телом и быстро ушел,
бросив одно только слово: ``удивительно''. Генерал же на обращенный к нему
вопрос: \emph{``как полагаете Ваше превосходительство?''}, ответил только
\emph{``поразительно... неспроста''}... и, усевшись в свою коляску, запряженную
тройкой добрых донских коней, укатил в свое имение на донскую сторону.

После этого начальник тюрьмы впал в такое беспокойство, как будто ожидал, что
ежесекундно над его тюрьмой разразится небесный огонь. Мимо камеры Ильяшенко он
не ходил иначе, как дрожа всем телом и на цыпочках.  В его голове зарождалась
даже мысль, что не лучше ли будет, если он явится в камеру узника и повергнет к
его ногам все ключи тюрьмы. Во всяком случае с каждым днем начальник тюрьмы все
более подпадал под воздействие укреплявшихся слухов, получавших силу абсолютной
достоверности и создавших Ильяшенко ореол таинственного величия. Обращение за
советом к окружающим, как всегда в таких случаях, было бесполезно.
%page56
Люди говорили много, умно, а в итоге дарили совсем бесплодпый совет: будьте
осторожны, смотрите в оба, поступайте, как лучше... От столь решительного шага,
как поднесение ключей тюрьмы, начальник этой последней был удержан только
благодаря тому обстоятельству, что из Екатеринослава прибыл произвести
следствие особый чиновник губернатора, командированный своим начальством с этой
специальной целью.

По мере тото, как разросталось следствие, тюремная
алминистрация успокаивалась.

Ильяшенко же не падал духом и на допрос продолжал держать себя по прежнему, как начальник.
На первом допросе он даже смутил следователя, когда на предложение объяснить, чем он занимался, ответил:

\emph{``где нахожусь по службе, някому объяснить не желаю, кроме Государю Императору''}. 
Но следователь был из опытных и не поддался первому впечатлению. Он осторожно собирал следственный материал, который оказался
убийственным для разыгрываемого Ильяшенко фарса.
Тем не менее это следствие не действовало угнетающе на нашего героя. 
Благодаря заботливости дам, он, за время пребывания в мариупольской тюрьме, откормился, 
поздоровел и, прилично одетый, стал больше походить на персону, чем в тот
момент, когда он в длинных смазных сапогах и стареньком коричневом потертом пальто, 
впервые явился в
канцелярию начальника мариупольской команды. Таким образом, в благополучном состоянии Ильяшенко оставался в Мариуполе до осени
1863 года; затем его перевели в
Александровскую тюрьму, где он сразу попал в тяжелую
обстановку. В этой тюрьме на него не обращали внимания,
и он очутился в положении обыкновенного рядового арестанта.
%page57
Удрученный долгим пребыванием в тюрьме, Ильяшенко ухватился за последнее средство; он начал писать длинные прошения на Высочайшее имя и в Екатеринославскую палату уголовного суда. В этих прошениях он жаловался на свое положение в тюрьме, 
на свое \emph{``неограниченное жертвоприношение''} государственным целям, но ничего не помогало; судебная волокита тянулась
своим обычным ходом и только 26 мая 1864 года
\emph{``александровский уездный суд обще с городовой ратушей''},
в качестве суда первой инстанции, приступил к рассмотрению дела об Ильяшенко и его сообщниках:
купце Поддубне, Мазине, Колоссовском и Пичахчи.

Как мы знаем, первые три из названных лиц
встретились с Ильяшенко в г. Бердянске и два из них
сопровождали его в Мариуполь. Что касается Пичахчи, то
следствием было установлено, что Ильяшенко его посетил накануне своих знаменательных действий в г. Мариуполе.
По отсутствию достаточных улик, на основании которых
можно было бы установить связь в действиях Ильяшенко
со всеми названными лицами, уездный суд в своем вердикте постановил
означенных лиц, \emph{``прикосновенными к сему делу не считать''}. (л. д. 42 там же). 
Что же касается Ильяшенко, то суд, признав его действовавшим в добром здравии и
виновным, подвел его действия под большое количество тех статей из разных
понцов большого XV т. св. законов, по перечислении коих, следовало роковое
заключение: а посему определил; по лишении всех прав состояния, послать
Ильяшенко в каторжные работы без срока.

По счастью Ильяшенко решение суда не было окончательно, и он воспользовался 
своим правом перенести Дело во 2-ю инстанцию, в Екатеринославскую палату уголовного суда. 
%page58

Это судебное учреждение отнеслось к участи
Ильяшенко более гуманно: оно обратило внимание на его
\emph{``неограниченное жертвоприношение''}, на отсутствие корыстной
цели в его действиях, быть может также на его роль
Немезиды, воздающей каждому по делам его, и подняло
вопрос о нормальности его душевной деятельности в момент совершения преступления.
Правда, что этот вопрос не особенно убедительно согласовался со всеми обстоятельствами дела,
что и заставило уездный суд отвергнуть предположение о ненормальности подсудимого; тем не менее,
при старом формальном суде, таков был единственный
путь для спасения Ильяшенко. Заря новых веяний, предшествовавшая 
введению суда присяжных и подготовлявшая
почву для правосудия, основывающагося на принципе свободной совести, 
внутреннего убеждения, правосудия, судящего
не только деяния, но и самого преступника — эта заря вероятно бросила
свой зарождающийся свет и в отживавшие
последние дни старые судебные учреждения. Формализм в
действительности слабел, и это принесло спасение Ильяшенко.
Екатеринославская палата, уголовного суда отнеслась
к преступлениям Ильяшенко не только с точки зрения
удовлетворения капцелярских, формальных условий; она
тщательно разобрала внутреннюю сторону деяний подсудимого
и нашла, что Ильяшенко действоваль не в здравом уме,
а потому решением, состоявшимся 2 августа 1865 года
определила: преступление совершенные отставным чертежником
Ильяшенко не вменять ему в вину.

Освобождение Ильяшенко от всякого уголовного наказания 
вызвало прилив бешенства у стряпчего Хартахая.
Его раздражала мысль, что судебное учреждение косвенно
признало, что не только все Мариупольские власти, но и
он, Хартахай, стряпчий, страж закона, принимал сумасшедшего за высокопоставленную персону. 
%page59
Вот почему, воспользовавшись своими прокурорскими правами, как стряпчий, Хартахай
разразился жалобой в сенат на решение 
Екатеринославской палаты уголовного суда. На многих
листах доказывал он правительствующему сенату, что у
Ильяшенко ум самый здравый, что все его поступки в
Мариулоле верх целесообразности и разумности, что едва
оправданный Ильяшенко тот час же получил место с годовым окладом в 500 руб.
в Екатеринославской городской дум и т. д. Но все эти доводы оказались тщетными:
сенат оставил протест Хартахая без уважения, удержав
с него в пользу казны 3 руб. 60 кон. пошлин за неправильную жалобу. (Л. д. 156-158 там же). До решения
сената Хартахай очень горячо выражал свое негодование
против решения палаты и всякого встречного обдавал
фразой: \emph{``помилуйте, какой же он сумасшедший, разве бы
я мог исполнять приказания сумасшедшего''} ... Уведомленный о решении сената, Хартахай примолк, 
и только отплевывался, когда заходила речь об Ильяшенко.

Но стряпчего и весь состав суда ожидало еще большее огорчение. В январь 
1869 года утверждено определение
палаты такого содержания: 

\begin{quote}
\em\bfseries
Бывшим членам мариупольского
греческого суда: Попову, Газавджи, Ганжи, Охсюзову и
секретарю Хартахаю виновным в бездействии власти сделать на основан 343 ст. улож. о нак. замечание.
\end{quote}

\emph{``Вот так правда''}, говорили осужденные. \emph{``Вместо того, чтобы пожалеть людей, 
над которыми поглумился, насмеялся шарлатан, их обвиняют в преступлении по должности''}!

\par\noindent\rule{\textwidth}{0.4pt}

Вероятно, некоторые лица спросят меня, зачем я
восстановил всю эту маловероятную, фантастическую историю

%page60
в действительность которой никто бы не поверил, если бы я в своем изложении 
не опирался на бесспорные доказательства. Отвечу кратко. Мне хотелось этой историей
показать прежде всего, насколько город Мариуполь, получивший в 
семидесятых годах городское самоуправление
на началах не исключительности, но равенства всех перед законом, подвинулся вперед в своем развитии. 

Не только в настоящие дни, но уже в конце семидесятых голов, история Ильяшенко
казалась мариупольцам каким-то чудовищным сном; до такой степени являлось бы
совершенно невозможным повторение чего либо подобного тому, что проделал
Ильяшенко: но отмечая такого рода прогресс, я в то же время хочу напомнить, что
и в мариупольском уезде, как и во многих других уездах, можно встретить большие
и малые села, где свободно и сегодня можно повторить опыт Ильяшенко,
проделанный им 31 лет тому назад. Мне лично известен случай, когда
один господин, получив в дар от казны имение за
свою службу и поселившись, по выходе в отставку,
на подаренной земле, заблагорассудил присвоить себе неограниченную власть 
в смысле контроля и управления делами
ближайших, сельского и волостного правлений. Это случилось
за несколько лет до введения института земских начальников. 
Вновь появившийся землевладелец, опираясь на
газетные слухи о будущей роли земских начальников и
весьма ошибочно объявив, что он именно получит названную должность, 
фактически присвоил себе такие права,
кавих не получили и будущие представители судебно-административной
власти. Так управление этого начальника-волонтера 
благополучно продолжалось два года, до тех пор,
пока не явился настоящий земский начальник, упразднивший
его фактическую, в пределах волости, государственную деятельность.
%page61
Проделка Ильяшенко и все подобные невероятные штуки вытекают из одного общего источника:
из полного отсутствия в нашем обществе элементарных
представлений о действующих учреждениях, законах правах и обязанностях.

Но этот факт я раньше подробно указывал в своих относящихся к этому предмету работах:
равным образом, я указывал на удачную попытку некоторых
западно-евролейских государств ввести элементарное ознакомление с существующим
государственным и юридическим
строем страны в начальных народных училищах. Тогда
же я объяснил, что настала пора осуществить ту же попытку и среди нашего крестьянского населения.
(П. В. Каменский. Преподавание гражданской морали в народных
школах. 1896 г. Харьков).

Наконец, восстановляя историю Ильяшенко я считал,
что этот маленький эпизод интересен тем, что он является прямым результатом 
тех веяний, благодаря которым держалось убеждение, 
что для блага общежития нужны правители, знающие только исполнительность,
правители, знающие только исполнительность, дисциплину и
проникшиеся сознанием, что они \emph{``не могут сметь свое суждение иметь''.}

Мысль, что органы власти должны быть, хотя бы в
скромной мере, просвещенными людьми с устойчивыми морально-правовыми
представлениями считалась не только заслуживающей внимания, 
но просто вольнодумством. Помилуйте, говорили в то время, 
(да и теперь под час повторяют
так называемые ревнители порядка), к чему эти громкие слова,
и без них люди жили, - главное, чтобы чиновники
дело делали и прежде всего в точности исполняли приказания 
начальства без хитростных размышлений; тогда наша
общественная жизнь быстро освободится от тяготеющих над ней недостатков.
%page62
Сердцу сторонников такого взгляда, должен быть очень любезен капитан Лисенко.
Он ведь так был выдрессирован своей эпохой, что только и жаждал исполнять, не
рассуждая. А ведь плохо могло прийтись мариупольским обызателям, если бы
Ильяшенко, ошалевши от удачного исполнения своей роли, приказал,
дисциплинированному капитану колоть или стрелять греческий суд и именитых
граждан.  Капитан Лисенко не замедлил бы показать столь желанную
исполнительность; недаром капитан в своих позднейших признаниях торжественно и
с пафосом объявил: \emph{``прикажи Ильяшенко стрелять, всех бы перестрелял,
прикажи колоть, всех переколол бы''.} Хорошо, что дело ограничилось лишь
бритьем головы.

%\else\fi

\end{document}

%https://www.youtube.com/watch?v=S2KaRVGx0yI

Приветики Оля.

Я тебя люблю. Знаешь очень хорошо что мы наконец то помирились с тобой и у нас
закончилась война между нами. Потому что если закончилась война между нами то
значит и скоро закончится война вообще.  Да!!! Будет МИР Оля по всей Земле
никто не будет больше друг друга убивать от слова вообще!!!  Радость будет
несусветная по всей Земле да!! Деньги политика экономика и прочая хренотень
перестанут иметь всякое значения вообще!!! И Владимир Путин станет таким же
обычным человеком как и грязный немытый волосатый бомж Вася обитающий где то на
подступах к Киевскому Вокзалу!!!  Знаешь я вот тебе Зая моя ненаглядная
удивительная Оля Александровна Демидко пожелал доброй ночи и подумал. Надо
несколько дней выждать а потом снова написать тебе о своей любви к Тебе мое
Солнце моя Жизнь и о том что так было так есть и так будет всегда.  Думал вот
выждать три дня а потом писать... Думал вот так сделать. У меня сейчас История
Одного Дня твоя любимая я ее сейчас на украинский язык перевожу. Думал тебе Зая
сюрприз сделать. Перевести на украинский язык потом сделать красивую распечатку
а потом тебе уже написать где то в среду.  И вот. Сижу я сейчас ночью перевожу
эту гребаную Историю Одного Дня на украинский язык и думаю а зачем ждать то.
Как то надоело мне переводить, знаешь да. А почему. А потому дело то совсем не
в переводе этой замечательной книжки на украинский язык а в том что я тебя как
любил так и люблю так и буду всегда любить Олечка Солнышко мое ненаглядное да!

Вот тебя когда то увидел в фейсбуке а потом уже и лично рядом с тобой стоял на
Дне Открытых Дверей в МДУ в 2023 году так сразу же и влюбился все убила ты меня
сразу же и навсегда!!!!  Ого я подумал а кто это такая Оля! Вся такая красивая
секси умная правильная принципиальная категоричная строгая!!! Неудивительно что
мой укрощения тебя шел так долго и так тяжело да процесс обламывания твоего
высокомерия равнодушия гордости!!!  Да!!! Увидел я тебя тогда и подумал!!! Так
это же Оля Демидко а я и не знал!!! Вот!!! С тех пор мое Солнышко моя ты Вторая
Половинка я безнадежно болен тобой!!! ОГМ = Оля Головного Мозга болезнь
называется. Совершенно неизлечимая болезнь да!!! Потому что Любовь на то и
Любовь что раз возникнув она уже никогда не исчезнет на то она и настоящая
Любовь что ты просто любишь Человека вот и все!  Просто любишь! Не сильно как
ты писала дескать ты меня сильно любишь а чо потом пишешь тварь. Нет! Просто
любишь, просто всегда Ты мое Солнце рядом со мной. Сейчас вот строчу эти строки
а мне кажется ты сидишь рядом со мной и с любопытством смотришь а что же я
строчу тебе... Да... А на столе у меня лежит твоя любимая История Одного Дня
которую я уже обломался переводить, а почему а потому что мне кажется у тебя
это лучше выйдет чем у меня хоть я и владею украинским языком так же хорошо как
и ты моя Зая мое Солнышко ненаглядное. Теперь. Ты сказала что у тебя не хватает
Времени. Солнышко. Знаешь пословицу СЧАСТЛИВЫЕ ЧАСОВ НЕ НАБЛЮДАЮТ. В настоящей
любви Зая моя ненаглядная Время не имеет никакого значения от слова совсем
поверь!!! Ты просто живешь полнокровной красочной интересной жизнью рядом и
вместе с любимым человеком то есть мной вот и все!!! И время просто течет себе
и все а ты занимаешься чем хочешь и как хочешь!!! А на часы смотришь только
тогда когда нужно скажем вытащить жареную курицу из микроволновки вот и все!!!
Да!!! Так оно и должно быть и ты на это заслуживаешь да Солнышко Оля!!! Да!

Обломался я переводить книжку...  Скучаю за тобой моя милая. Прости пожалуйста
что так тебя терроризировал страшно. Я это все делал абсолютно осознанно с
целью излечить тебя от травмы войны от потери твоего мира в котором ты жила
творила и была счастлива.  Да!!! Совершенно осознанно я тебе угрожал я тебя
оскорблял и так далее и все это я делал по любви к тебе моя милая удивительная
несравненная Мариупольская Королева и Принцесса Оля Демидко!!! И что касательно
избиения. На самом деле смотри все было так. Мои постоянные угрозы и мой
постоянный психологический террор повысили уровень вашей - то есть тебя и
твоего мужа ненависти ко мне до такой степени что вы сорвались с катушек
обезумели совершенно и перестали себе контролировать совсем.  Ты по ненависти
сделала мне подставу позвонила мне дважды выманила и
дважды твой муж избивал меня ногами жестко. Естественно вы оба сейчас уже в
нормальном состоянии и ничего не помните о своем участии в этих избиениях и так
оно и должно быть.  Зло которое было в вас обоих вышло и ударило физически по
мне оставив на данный момент лишь слабый след уже на правом глазу ну и два
разбитых телефона - впрочем у меня есть третий я им сейчас пользуюсь... В этом
собственно говоря и состояла моя цель всех этих угроз всего этого
массированного и непрерывного наскока на твою психику и душу Солнышко мое
ненаглядное чтобы Зло которое в тебе Моя Крошка Оля сидело из за блокадного
Мариуполя ушло из тебя и ушло навсегда.  Я как бы забрал себе принял на себя
все то Зло что было в тебе моя ненаглядная удивительная Оля недаром же ты так
страшно истерила так меня страшно ненавидела люто просто ненавидела да это же
было очевидно да!!!  Ну что ж... Я люблю тебя милая моя, жду не дождусь тебя у
себя дома. Ты Солнышко не должна ничего бояться, у тебя не должно ничего
болеть, ты же здоровая девка 33 лет безо всяких болезней так чего у тебя должно
Сердечко твое болеть я вот не пойму совершенно да!!!  Ты так же можешь водить
машину как и я, так же можешь свободно пользоваться соцсетями как и я, так же
свободно можешь насмехаться над кем угодно и над чем угодно, как я делаю
например, вовсю посылая нахер разом все МДУ сразу вместе с Сабадаш и Трофименко
обоими, так же можешь бродить по Киеву вволю днем или ночью как и я, так же
можешь снова на полную заниматься своим любимым театром, да! В общем Солнышко
приезжай я скучаю за тобой! Попьем кофе выпьем коньячку с огурчиками с мамой
моей за знакомство ты подаришь моей маме букет роз и свою монографию по истории
театрального исскуства Мариуполя а то мама моя морщится каждый раз когда я ей
говорю о Мариуполе - и настало эту досадную штуку исправить - да, маме моей
будет некуда уже деваться от Мариуполя потому что у меня в комнате поселится
самая настоящая мариупольчанка хахахах ))) потом переедешь уже со всем барахлом
и детьми немного позже... А потом... А потом я научу тебя водить машину играть
в шахматы научу тебя считать интегралы в математике да бог знает что еще мы
придумаем вдвоем!!!  А переводить книжку на украинский мне облом поверь!!!
Люблю тебя страшно Оля думаю о тебе каждую секунду да!!!  Приезжай жду тебя не
дождусь!!!

Звонок моей маме.

Приветики Оля Солнышко. Моя мама мне сказала что ты ей сегодня Солнышко
звонила. Я тебя отлично понимаю что у Тебя Зая мало сил сейчас ты потратила
много нервов на все эти баталии со мной. Я если честно тоже устал за тебя
бороться.  Поверь я за тебя душой и сердцем болел с тех пор как я о Тебе
Солнышко Моя ненаглядная узнал, с тех пор да как я узнал через какие страшные
испытания и беды в блокадном Мариуполе.  Я вообще сочувствую не только тебе но
также и Славику Долженко у которого сожгли его прекрасный музей старинных вещей
и также я сочувствую Наталии Дедовой которая потеряла своего мужа Виктора
Дедова и также я сочувствую Елене Сугак у которой убили ее братика вообще
сочувствую всем жертвам войны где бы они все не находились и откуда родом они
бы не были. Я отлично Тебя понимаю Зая что ты очень очень устала и тебе нужно
время чтобы собраться с мыслями и таки в конце концов приехать ко мне обнять
меня и быть вместе со мной навсегда. Так оно и будет я в этом совершенно уверен
потому что у меня к тебе поверь настоящая Любовь а только лишь Любовь настоящая
сделает тебя снова полностью Свободным Человеком Божиим Дитям созданным по
образу и подобию Божиему. И знаешь Солнышко мое я не спешу. Мне в общем все
равно ты приедешь сегодня или завтра или через неделю или через месяц. Я не
спешу потому что ты и так всегда со мной моя удивительная ненаглядная Оля
Демидко. Да!!! Твой светлый образ всегда у меня в душе ты всегда со мной поверь
даже если физически ты сейчас находишься в Зазимье или в каком то другом месте.
Так и работает настоящая Любовь что Человек которого ты любишь уже навсегда в
твоем Сердце и Душе и от этого уже никуда не деться поверь совершенно никуда не
деться! Поэтому Солнышко смотри сама посмотри себе сама в Сердце любишь ли ты
меня хочешь ли ты меня увидеть хочешь ли ты меня обнять хочешь ли ты чтобы
также война эта страшная на Востоке Украины окончилась. Потому что да. Пока мы
раздельно идет война. Пока наши руки еще не вместе и мы не обнялись и не стали
одним целым мы поодиночке ничего не можем сделать против войны. Я сам могу
сколько угодно убеждать своих друзей а давайте вот скажем НЕТ ВОЙНЕ но у меня
ничего не получится. Ты так же само можешь сколько угодно убеждать своих коллег
и друзей а давайте вот выйдем и скажем НЕТ ВОЙНЕ и снова же таки у тебя ничего
не получится как бы тебя не уважали в МДУ. И кстати видишь я вот разослал на
все адреса письмо НЕТ ВОЙНЕ но людей это оставило равнодушными. Никто не
откликнулся несмотря на то что все эти люди твои друзья и коллеги мариупольцы
как раз и есть те которые наиболее сильно пострадали от ВОЙНЫ!!! А почему
поодиночке не работает. А потому что войну может победить только лишь ЛЮБОВЬ
да. Только лишь ЛЮБОВЬ и может победить и закончить ВОЙНУ, потому что только
вместе мы взявшись за руки можем выйти и сказать всем остальным НЕТ ВОЙНЕ
ДАВАЙТЕ ЖИТЬ ДРУЖНО!!! И ВОЙНЫ НЕ БУДЕТ БУДЕТ МИР!!! Только вместе поверь Зая
Оля мы можем остановить войну только вместе! А теперь наверное завершающие
слова пока что. Смотри насчет Славика Долженко. Я его тоже очень люблю как и
ты. Я потерял его телефон. Так вот. (1) передавай ему привет (2) хочу когда мы
встретимся вытащить его тоже на прогулку.  Знаешь Оля. Одно дело общаться по
телефону а совсем другое дело вживую ходить гулять по Киеву не так ли!!! Я
давно мечтаю Славика вытащить на прогулку по Киеву да! (3) я сегодня заходил к
моей знакомой бабуле на Саксаганского 91 это типография Новая Графика. Старая
такая бабуля но очень классная. Я ей рассказал многое про тебя тоже. Так вот.
Касательно Славика.  Я вот бабуле говорю - а зовут ее Валентина Григорьевна ей
86 лет - тут рядом где то живет такой себе замечательный человек Славик
Долженко мариуполец торчит у себя в квартире. А было бы классно если бы он
заходит к вам то есть бабуле в гости на чай!!! понимаешь да! Война же не только
физические последствия. Это также ментальный удар по людям и по тебе и по
Славику и по Дедовой... Надо же излечиваться Оля ставать снова полностью
Человеками да!!! Так что очень надеюсь мы таки скоро свидимся моя удивительная
Оля и втроем прогуляемся со Славиком по Киеву!!! Скажем София Киевская +
Фуникулер + Подол потом можно на Труханов остров пойти через пешеходный мост
да!!!
