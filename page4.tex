\ifrus
ибо они сводились на повторение одних и тех же слов: каторжные греческие порядки,
проклятое греческое царство и т. п. Понятно, что эти повторные фразы,
переплетаемые с ругательствами, никак не знакомили 
с учреждениями, которым подчинялось в то время греческое население.
На самом же деле, ознакомление с этими учреждениями не представляется делом сложным.

Переселившиеся в конце прошлого столетия из Крымского полуострова греки
основали город Мариуполь и 353 греческих села.  По позднейшим законодательным
актам они составили особый греческий округ, и населявшие его греки подчинялись
исключительно созданному для них учреждению, называвшемуся греческим судом. Это
было учреждение одновременно судебное, административное и полицейское.  Его
компетенция распространялась только на греков, другие национальности ведению
этого суда не подлежали. Состав суда состоял из председателя, и трех членов,
которые назывались заседателями, секретаря и подчиненных ему столоначальников.
Все дела решались коллегиально, и только по обычаю, но не по закону;
практиковалось, что на одного заседателя возлагались полицейские обязанности,
на другого обязанности следователя, а третий заведовал хозяйственной частью.
Исключая секретаря и подчиненных ему столоначальников, остальной состав
избирался греческими поселенцами на трехлетний период.

Для этого один раз в трехлётний период каждое
из 28-х греческих поселений посылало в гор. Мариуполь
двух уполномоченных, которые, вместе с представителями города, избирали часть состава
греческого суда, т. е. председателя и трех заседателей. По обычаю, принявшему вследствие своей неизменной
повторности силу закона, избирался всегда тот кандидат, который ставил избирателям больше вина.
\else\fi
\ifukr 
\else\fi
