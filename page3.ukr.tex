\ifpages\section{Сторінка 3}\else\fi
Історія Одного Дня 

(Достовірна Розповідь)

Біля 20-х чисел березня 1868 року у місті Бердянську, у готелі ``Білий Лебідь'' орехівський
третьої гільдії купець Поддубня, який торгував чаєм, дьогтем та салом, розповідав двум своїм
знайомим, Мазіну та Колосовському, про зловісні пригоди, які стались із ним у місті Маріуполь. Серед
слухачів був також третій чоловік, невідомий Поддубні, що також слухав
розповідь; його звали \textbf{\em Григорій Власов Ільяшенко}. Він був молодий чоловік років 33, блондин,
зовнішньо нічим особливо собою не помітний, і ніяких, як мовиться у паспортах, особливих прикмет не мавший.
Родом він був із міста Миколаїв: у час коли трапилась ця історія він жив разов із своєю жінкою в місті Бердянськ, 
де займався приватним чином креслярськими роботами у місцевого архітектора. До того як він прибув у Бердянськ Ільяшенко
знаходився на службі в севастопольській інженерній команді морської будівельної частини як кресляр та був нагороджений,
як це було підтверджено офіційними документами, бронзовою медаллю на Андріївскій ленті.

Розповідь Поддубні не була зв'язною. Він почав із пояснення, у чому полягало у місті Маріуполь особливе
грецьке управління, але його пояснення були мало зрозумілі, 
