\ifrus
\ifpages\section{Страница 3}\else\fi
ИСТОРИЯ ОДНОГО ДНЯ

(ДОСТОВЕРНОЕ СКАЗАНИЕ)

Около 20-х чисел марта 1868 года в г. Бердянске,
в гостинице ``Белого Лебедя'' ореховский 3-й гильдии купец Поддубня,
торговавший чаем, дегтем и салом, рассказывал двум своим знакомым, Мазину и Колосовскому,
о приключившихся ему злоключениях в г. Мариуполе. В числе слушателей
находился еще третий слушатель, неизвестный Поддубне, но также внимавший рассказчику;
назывался он \textbf{\em Григорий Власов Ильяшенко}. Это был молодой человек 33 лет, блондин, внешней 
фигурой ничего из себя особенного не представлявший, никаких, как говорится в паспортах, особых примет не имевший.
Он был уроженец гор. Николаева: в описываемое время проживал с женой в г. Бердянске, где занимался частно
чертежными работами у местного архитектора. До прибытия в Бердянск Ильяшенко
находился на службе в севастопольской инженерной команде морской строительной части
чертежником и был награжден, как это удостоверяется
официальными документами, бронзовой медалью на Андреевской ленте.

Рассказ Поддубни не был связный. Он начал с объяснения, в чем состояло в г. Мариуполе
особое греческое управление, но его объяснения были мало уяснительны,
\else\fi

\ifukr
\else\fi
